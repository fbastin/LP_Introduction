\documentclass[usepdftitle=false]{beamer}
\usepackage[utf8]{inputenc}
\usetheme{Singapore}
\usepackage{xcolor}
% \setbeamercovered{transparent}
%\usecolortheme{crane}
\title[IFT2505]{IFT 2505\\Programmation Linéaire\\Introduction}
\author[Fabian Bastin]{Fabian Bastin\\DIRO\\Université de Montréal}
\date{}

\usepackage[french]{babel}

\setbeamertemplate{footline}[frame number]

\usepackage{ulem}
\usepackage{tikz}

\def\bb{\boldsymbol{b}}
\def\bc{\boldsymbol{c}}
\def\bx{\boldsymbol{x}}
\def\bA{\boldsymbol{A}}
\def\bB{\boldsymbol{B}}
\def\bD{\boldsymbol{D}}

\newtheorem{exe}{Exemple}

\begin{document}
\frame{\titlepage}

% ------------------------------------------------------------------------------------------------------------------------------------------------------
\begin{frame}
\frametitle{Objectifs du cours}

\begin{enumerate}
\item
Bases de la programmation mathématique:
\begin{itemize}
\item
objectif, contraintes;
\item
solution(s) réalisable(s).
\item
solution(s) optimale(s), valeur optimale;
\end{itemize}
\item
Comprendre les principes de base de la programmation linéaire (PL), et de ses méthodes de résolution. La méthode du simplexe sera étudiée avec un niveau élevé de détails.
\item
Comprendre la notion de dualité.
\item
Principes de base des méthodes de points intérieurs.
\item
Introduction à la programmation en nombres entiers.
\item
Aspects numériques de la PL (utilisation de Julia).
\end{enumerate}

\end{frame}

\begin{frame}
\frametitle{Ouvrages de références}

\begin{itemize}
	\item
	Luenberger et Ye, Linear and Nonlinear Programming, 4e édition, Springer, 2016, \url{https://link.springer.com/book/10.1007\%2F978-3-319-18842-3}
	\item
	Chan, Batson, Dang, Applied Integer Programming: Modeling and Solution, Wiley, 2011,
	\url{https://onlinelibrary.wiley.com/doi/book/10.1002/9781118166000}
\end{itemize}

\end{frame}

\begin{frame}
\frametitle{Modalités}

Évaluations:
\begin{itemize}
	\item 5 devoirs, 10\% chaque;
	\item Examen intra: 20\%, une page de notes autorisée;
	\item Examen final: 30\%, deux pages de notes autorisées.
\end{itemize}
{\bf Le règlement sur le plagiat sera de stricte application!}

\mbox{}

Démonstrateur:\\
 Jean Laprés-Chartrand (\url{jl_chartrand@hotmail.com})\\

\mbox{}

\begin{itemize}
	\item 
Notebooks: \url{https://github.com/fbastin/optim};
\item
Slides: \url{https://github.com/fbastin/LP_Introduction};
\item
Canal discord: \url{https://discord.gg/j7FBs9V6}.
\end{itemize}

\end{frame}

\begin{frame}
\frametitle{PL: forme standard}

\begin{align*}
\min_{\bx}\ z &= c_1x_1+c_2x_2+\ldots +c_nx_n \\
\mbox{sujet à } &
a_{11}x_1 + a_{12}x_2 + \ldots + a_{1n}x_n = b_1 \\
&a_{21}x_1 + a_{22}x_2 + \ldots + a_{2n}x_n = b_2 \\
& \qquad \vdots \qquad \vdots \\
& a_{m1}x_1 + a_{m2}x_2 + \ldots + a_{mn}x_n = b_m \\
& x_1 \geq 0,\ x_2 \geq 0, \ldots, x_n \geq 0 \\
\end{align*}

\mbox{}
\end{frame}

\begin{frame}
\frametitle{Ecriture matricielle}

\begin{align*}
\min_{\bx} \ z &= \bc^T x \\
\mbox{sujet à } \bA x &= \bb \\
\bx & \geq 0
\end{align*}

\end{frame}

\begin{frame}
\frametitle{Retour à la forme standard}

{\bf Variables d'écart}

\begin{align*}
& a_{j1}x_1 + a_{j2}x_2 + \ldots + a_{jn}x_n \leq b_j \\
\rightarrow & a_{j1}x_1 + a_{j2}x_2 + \ldots + a_{jn}x_n + s_j = b_j \end{align*}

\mbox{}

{\bf Variables de surplus}

\begin{align*}
& a_{j1}x_1 + a_{j2}x_2 + \ldots + a_{jn}x_n \geq b_j \\
\rightarrow & a_{j1}x_1 + a_{j2}x_2 + \ldots + a_{jn}x_n - u_j = b_j \end{align*}

\end{frame}

\begin{frame}
\frametitle{Retour à la forme standard (suite)}

{\bf Variables libres}

$x_j$ non contraint: \sout{$x_j \geq 0$}

\mbox{}

{\bf Première approche: décomposition}

\[
x_j = s_j - u_j,\quad s_j \geq 0,\ u_j \geq 0.
\]

\mbox{}

{\bf Deuxième approache: substitution}

Supposons $a_{kj} \ne 0$
\[
x_j = \frac{b_k}{a_{kj}} - \frac{a_{k1}}{a_{kj}}x_1 - \frac{a_{k2}}{a_{kj}}x_2 - \ldots - \frac{a_{k,j-1}}{a_{kj}}x_{j-1} - \frac{a_{k,j+1}}{a_{kj}}x_{j+1} -  \ldots - \frac{a_{kn}}{a_{kj}}x_n.
\]
On remplace dans les autres contraintes.

\end{frame}

\begin{frame}
\frametitle{Maximisation vs minimisation}

Le problème
\begin{align*}
\max_{\bx} \ & \bc^T x \\
\mbox{sujet à } & \bA x = \bb \\
& \bx \geq 0
\end{align*}
est équivalent à
\begin{align*}
- \min_{\bx} \ & -\bc^T x \\
\mbox{sujet à } &  \bA x = \bb \\
& \bx \geq 0
\end{align*}

\end{frame}

\begin{frame}
\frametitle{Réalisabilité -- optimalité}

$\bx$ est une solution réalisable si $\bA\bx = \bb$ et $\bx \geq 0$.

\mbox{}

$\bx$ est une solution optimale si $\bx$ est réalisable et il n'existe pas de de $\bx'$ réalisable tel que $\bc^T \bx' < \bc^T \bx$.

\mbox{}

Supposons $m \leq n$ et rang($\bA$) = $m$.
Sans perte de généralité, supposons les $m$ premières colonnes de $\bA$ indépendantes, et formons
\[
\bA =
\begin{pmatrix}
 \bB & \bD
\end{pmatrix}
\]
Solution de base: $\bx = (\bx_b \ 0)$, avec $\bB\bx_b = \bb$.

\end{frame}

\begin{frame}
\begin{exe}[Wyndor Glass]
\label{ex:wyndor}
La compagnie Wyndor Glass Co. produit des produits verriers de haute qualité, incluant des fenêtres et des portes vitrées.
Elle dispose à cette fin de trois usines (usine 1, usine 2, usine 3), qui ont chacune une capacité de production limitée.
Les châssis en aluminium et les matériaux sont produits dans l'usine 1, les châssis en bois sont fabriqués dans l'usine 2, et l'usine 3 produit le verre et assemble les produits.
La compagnie a décidé de mettre en place de ligne de production:
\begin{itemize}
\item
produit 1: une porte vitrée avec un châssis d'aluminium;
\item
produit 2: une fenêtre double-vitrage avec châssis en bois.
\end{itemize}

Un lot de 20 unités donne lieu à un profit de \$3000 et \$5000, respectivement pour le produit 1 et le produit 2.
\end{exe}

\end{frame}

\begin{frame}

\begin{exe}[Wyndor Glass]

\begin{center}
\begin{tabular}{|c|c|c|c|}
\hline
& Produit 1 & Produit 2 & Capacité de \\
& Tps de prod. (h) & Tps de prod. (h)& production (h)\\
\hline
Usine 1 & 1 & 0 & 4 \\
\hline
Usine 2 & 0 & 2 & 12 \\
\hline
Usine 3 & 3 & 2 & 18 \\
\hline
\end{tabular}
\end{center}

Chaque lot d'un produit est le résultat combiné de la production dans les trois usines.
Nous souhaitons déterminer le taux de production pour chaque produit (nombre de lots par semaine) de façon à maximiser le profit total.

Les variables de décision sont
\begin{itemize}
\item
$x_1$, le nombre de lots du produit 1;
\item
$x_2$, le nombre de lots du produit 2.
\end{itemize}
Le fonction objectif est le profit total, qui vaut $3x_1 +5x_2$, en l'exprimant en miller de dollars. Nous voulons maximiser ce profit.
\end{exe}

\end{frame}

\begin{frame}

\begin{exe}[Wyndor Glass]
En résumé, nous avons le problème d'optimisation suivant:
\begin{align*}
\max_x\ & z = 3x_1 + 5x_2 \\
\mbox{t.q. } & x_1\qquad & \leq 4 & \mbox{ (usine 1)} \\
& 2x_2 & \leq 12 & \mbox{ (usine 2)} \\
& 3x_1 + 2x_2 & \leq 18 & \mbox{ (usine 3)} \\
& x_1 \geq 0,\ x_2 \geq 0 & \mbox{ (non-négativité)}
\end{align*}

Dans l'exemple
\[
z = 3x_1+5x_2 \Leftrightarrow x_2 = -\frac{3}{5}x_1 + \frac{1}{5}z.
\]
L'ordonnée à l'origine, dépendant de la valeur de $z$, est $\frac{1}{5}z$, et la pente vaut $-\frac{3}{5}$.
Maximiser revient à augmenter $z$.
\end{exe}

\end{frame}

\begin{frame}
\frametitle{Solution graphique}

\begin{center}
\begin{tikzpicture}[scale=0.7]
\draw[->] (0,0) -- (10,0) node[below,right] {$x_1$};
\draw[->] (0,0) -- (0,10) node[above,left] {$x_2$};
\draw (6,0) -- (0,9) node[above,right] {$3x_1 + 2x_2 \leq 18$};
\draw (0,6) -- (8,6) node[right] {$2x_2 \leq 12$};
\draw (4,0) -- (4,8) node[above,right] {$x_1 \leq 4$};

\foreach \x in {0,1,2,3,4,5,6,7,8,9}
  \draw (\x,1pt) -- (\x,-1pt) node[anchor=north] {$\x$};
\foreach \y in {0,1,2,3,4,5,6,7,8,9}
  \draw (1pt,\y) -- (-1pt,\y) node[anchor=east] {$\y$};

\filldraw[fill=yellow]
  (0,0) -- (0,6) -- (2,6) -- (4,3) -- (4,0) -- (0,0);
  
\draw (2,3) node[text width=2cm,text ragged] {domaine réalisable};

\end{tikzpicture}
\end{center}

\end{frame}

\begin{frame}
\frametitle{Solution graphique}

\begin{center}
\begin{tikzpicture}[scale=0.65]
\draw[->] (0,0) -- (10,0) node[below,right] {$x_1$};
\draw[->] (0,0) -- (0,10) node[above,left] {$x_2$};

\foreach \x in {0,1,2,3,4,5,6,7,8,9}
  \draw (\x,1pt) -- (\x,-1pt) node[anchor=north] {$\x$};
\foreach \y in {0,1,2,3,4,5,6,7,8,9}
  \draw (1pt,\y) -- (-1pt,\y) node[anchor=east] {$\y$};

\filldraw[fill=yellow]
  (0,0) -- (0,6) -- (2,6) -- (4,3) -- (4,0) -- (0,0);

\draw (5.0,-1) -- (-1,13.0/5) node[above,left] {$10=3x_1+5x_2$};
\draw (25.0/3,-1) -- (-1,23.0/5) node[above,left] {$20=3x_1+5x_2$};
\draw (38.5/3,-0.5) -- (-1,39.0/5) node[above,left] {$36=3x_1+5x_2$};

\draw (2,6) node[above] {$(2,6)$};
\fill (2,6) circle (2 pt);

\end{tikzpicture}
\end{center}

\end{frame}

\begin{frame}[fragile]
\frametitle{JuMP}

JuMP: Julia for Mathematical Optimization

\begin{verbatim}
using JuMP
using Clp

m = Model(with_optimizer(Clp.Optimizer))

@variable(m, 0 <= x <= 4)
@variable(m, 0 <= y <= 6)

@constraint(m, 3x + 2y <= 18.0)

@objective(m, Max, 3x+5y)

status = optimize!(m)
\end{verbatim}

\end{frame}

\end{document}

