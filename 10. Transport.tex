\documentclass[usepdftitle=false, aspectratio=169]{beamer}

\usepackage[utf8]{inputenc}

\usetheme{Singapore}
\usepackage{xcolor}

\setbeamertemplate{footline}[frame number]

\usepackage[frenchb]{babel}

\usepackage{xcolor}
% \setbeamercovered{transparent}
%\usecolortheme{crane}
\title[Transport]{Programmation Linéaire\\Problèmes de transport}
\author[Fabian Bastin]{Fabian Bastin\\DIRO\\Université de Montréal}
\date{}

\usepackage{ulem}
\usepackage{cancel}

\usepackage{tikz}
\newcommand*\circled[1]{\tikz[baseline=(char.base)]{
    \node[shape=circle,draw,inner sep=2pt] (char) {#1};}}

% We add the following commonly used macros:

% Note: \bsc is already in babel-french; the macros below override it

% vectors as boldsymbols:
\newcommand{\bsa}{{\boldsymbol{a}}}
\newcommand{\bsb}{{\boldsymbol{b}}}
\providecommand{\bsc}{} % if \bsc is not defined, define it
\renewcommand{\bsc}{{\boldsymbol{c}}}
\newcommand{\bsd}{{\boldsymbol{d}}}
\newcommand{\bse}{{\boldsymbol{e}}}
\newcommand{\bsf}{{\boldsymbol{f}}}
\newcommand{\bsg}{{\boldsymbol{g}}}
\newcommand{\bsh}{{\boldsymbol{h}}}
\newcommand{\bsi}{{\boldsymbol{i}}}
\newcommand{\bsj}{{\boldsymbol{j}}}
\newcommand{\bsk}{{\boldsymbol{k}}}
\newcommand{\bsl}{{\boldsymbol{l}}}
\newcommand{\bsm}{{\boldsymbol{m}}}
\newcommand{\bsn}{{\boldsymbol{n}}}
\newcommand{\bso}{{\boldsymbol{o}}}
\newcommand{\bsp}{{\boldsymbol{p}}}
\newcommand{\bsq}{{\boldsymbol{q}}}
\newcommand{\bsr}{{\boldsymbol{r}}}
\newcommand{\bss}{{\boldsymbol{s}}}
\newcommand{\bst}{{\boldsymbol{t}}}
\newcommand{\bsu}{{\boldsymbol{u}}}
\newcommand{\bsv}{{\boldsymbol{v}}}
\newcommand{\bsw}{{\boldsymbol{w}}}
\newcommand{\bsx}{{\boldsymbol{x}}}
\newcommand{\bsy}{{\boldsymbol{y}}}
\newcommand{\bsz}{{\boldsymbol{z}}}
\newcommand{\bsA}{{\boldsymbol{A}}}
\newcommand{\bsB}{{\boldsymbol{B}}}
\newcommand{\bsC}{{\boldsymbol{C}}}
\newcommand{\bsD}{{\boldsymbol{D}}}
\newcommand{\bsE}{{\boldsymbol{E}}}
\newcommand{\bsF}{{\boldsymbol{F}}}
\newcommand{\bsG}{{\boldsymbol{G}}}
\newcommand{\bsH}{{\boldsymbol{H}}}
\newcommand{\bsI}{{\boldsymbol{I}}}
\newcommand{\bsJ}{{\boldsymbol{J}}}
\newcommand{\bsK}{{\boldsymbol{K}}}
\newcommand{\bsL}{{\boldsymbol{L}}}
\newcommand{\bsM}{{\boldsymbol{M}}}
\newcommand{\bsN}{{\boldsymbol{N}}}
\newcommand{\bsO}{{\boldsymbol{O}}}
\newcommand{\bsP}{{\boldsymbol{P}}}
\newcommand{\bsQ}{{\boldsymbol{Q}}}
\newcommand{\bsR}{{\boldsymbol{R}}}
\newcommand{\bsS}{{\boldsymbol{S}}}
\newcommand{\bsT}{{\boldsymbol{T}}}
\newcommand{\bsU}{{\boldsymbol{U}}}
\newcommand{\bsV}{{\boldsymbol{V}}}
\newcommand{\bsW}{{\boldsymbol{W}}}
\newcommand{\bsX}{{\boldsymbol{X}}}
\newcommand{\bsY}{{\boldsymbol{Y}}}
\newcommand{\bsZ}{{\boldsymbol{Z}}}
% other commonly used boldsymbols:
\newcommand{\bsell}{{\boldsymbol{\ell}}}
\newcommand{\bszero}{{\boldsymbol{0}}} % vector of zeros
\newcommand{\bsone}{{\boldsymbol{1}}}  % vector of ones
% boldsymbol greeks:
\newcommand{\bsalpha}{{\boldsymbol{\alpha}}}
\newcommand{\bsbeta}{{\boldsymbol{\beta}}}
\newcommand{\bsgamma}{{\boldsymbol{\gamma}}}
\newcommand{\bsdelta}{{\boldsymbol{\delta}}}
\newcommand{\bsepsilon}{{\boldsymbol{\epsilon}}}
\newcommand{\bsvarepsilon}{{\boldsymbol{\varepsilon}}}
\newcommand{\bszeta}{{\boldsymbol{\zeta}}}
\newcommand{\bseta}{{\boldsymbol{\eta}}}
\newcommand{\bstheta}{{\boldsymbol{\theta}}}
\newcommand{\bsvartheta}{{\boldsymbol{\vartheta}}}
\newcommand{\bskappa}{{\boldsymbol{\kappa}}}
\newcommand{\bslambda}{{\boldsymbol{\lambda}}}
\newcommand{\bsmu}{{\boldsymbol{\mu}}}
\newcommand{\bsnu}{{\boldsymbol{\nu}}}
\newcommand{\bsxi}{{\boldsymbol{\xi}}}
\newcommand{\bspi}{{\boldsymbol{\pi}}}
\newcommand{\bsvarpi}{{\boldsymbol{\varpi}}}
\newcommand{\bsrho}{{\boldsymbol{\rho}}}
\newcommand{\bsvarrho}{{\boldsymbol{\varrho}}}
\newcommand{\bssigma}{{\boldsymbol{\sigma}}}
\newcommand{\bsvarsigma}{{\boldsymbol{\varsigma}}}
\newcommand{\bstau}{{\boldsymbol{\tau}}}
\newcommand{\bsupsilon}{{\boldsymbol{\upsilon}}}
\newcommand{\bsphi}{{\boldsymbol{\phi}}}
\newcommand{\bsvarphi}{{\boldsymbol{\varphi}}}
\newcommand{\bschi}{{\boldsymbol{\chi}}}
\newcommand{\bspsi}{{\boldsymbol{\psi}}}
\newcommand{\bsomega}{{\boldsymbol{\omega}}}
\newcommand{\bsGamma}{{\boldsymbol{\Gamma}}}
\newcommand{\bsDelta}{{\boldsymbol{\Delta}}}
\newcommand{\bsTheta}{{\boldsymbol{\Theta}}}
\newcommand{\bsLambda}{{\boldsymbol{\Lambda}}}
\newcommand{\bsXi}{{\boldsymbol{\Xi}}}
\newcommand{\bsPi}{{\boldsymbol{\Pi}}}
\newcommand{\bsSigma}{{\boldsymbol{\Sigma}}}
\newcommand{\bsUpsilon}{{\boldsymbol{\Upsilon}}}
\newcommand{\bsPhi}{{\boldsymbol{\Phi}}}
\newcommand{\bsPsi}{{\boldsymbol{\Psi}}}
\newcommand{\bsOmega}{{\boldsymbol{\Omega}}}

% Roman fonts:
\newcommand{\rma}{{\mathrm{a}}}
\newcommand{\rmb}{{\mathrm{b}}}
\newcommand{\rmc}{{\mathrm{c}}}
\newcommand{\rmd}{{\mathrm{d}}}
\newcommand{\rme}{{\mathrm{e}}}
\newcommand{\rmf}{{\mathrm{f}}}
\newcommand{\rmg}{{\mathrm{g}}}
\newcommand{\rmh}{{\mathrm{h}}}
\newcommand{\rmi}{{\mathrm{i}}}
\newcommand{\rmj}{{\mathrm{j}}}
\newcommand{\rmk}{{\mathrm{k}}}
\newcommand{\rml}{{\mathrm{l}}}
\newcommand{\rmm}{{\mathrm{m}}}
\newcommand{\rmn}{{\mathrm{n}}}
\newcommand{\rmo}{{\mathrm{o}}}
\newcommand{\rmp}{{\mathrm{p}}}
\newcommand{\rmq}{{\mathrm{q}}}
\newcommand{\rmr}{{\mathrm{r}}}
\newcommand{\rms}{{\mathrm{s}}}
\newcommand{\rmt}{{\mathrm{t}}}
\newcommand{\rmu}{{\mathrm{u}}}
\newcommand{\rmv}{{\mathrm{v}}}
\newcommand{\rmw}{{\mathrm{w}}}
\newcommand{\rmx}{{\mathrm{x}}}
\newcommand{\rmy}{{\mathrm{y}}}
\newcommand{\rmz}{{\mathrm{z}}}
\newcommand{\rmA}{{\mathrm{A}}}
\newcommand{\rmB}{{\mathrm{B}}}
\newcommand{\rmC}{{\mathrm{C}}}
\newcommand{\rmD}{{\mathrm{D}}}
\newcommand{\rmE}{{\mathrm{E}}}
\newcommand{\rmF}{{\mathrm{F}}}
\newcommand{\rmG}{{\mathrm{G}}}
\newcommand{\rmH}{{\mathrm{H}}}
\newcommand{\rmI}{{\mathrm{I}}}
\newcommand{\rmJ}{{\mathrm{J}}}
\newcommand{\rmK}{{\mathrm{K}}}
\newcommand{\rmL}{{\mathrm{L}}}
\newcommand{\rmM}{{\mathrm{M}}}
\newcommand{\rmN}{{\mathrm{N}}}
\newcommand{\rmO}{{\mathrm{O}}}
\newcommand{\rmP}{{\mathrm{P}}}
\newcommand{\rmQ}{{\mathrm{Q}}}
\newcommand{\rmR}{{\mathrm{R}}}
\newcommand{\rmS}{{\mathrm{S}}}
\newcommand{\rmT}{{\mathrm{T}}}
\newcommand{\rmU}{{\mathrm{U}}}
\newcommand{\rmV}{{\mathrm{V}}}
\newcommand{\rmW}{{\mathrm{W}}}
\newcommand{\rmX}{{\mathrm{X}}}
\newcommand{\rmY}{{\mathrm{Y}}}
\newcommand{\rmZ}{{\mathrm{Z}}}
% also commonly defined
\newcommand{\rd}{{\mathrm{d}}}
\newcommand{\ri}{{\mathrm{i}}}

% blackboards:
\newcommand{\bbA}{{\mathbb{A}}}
\newcommand{\bbB}{{\mathbb{B}}}
\newcommand{\bbC}{{\mathbb{C}}}
\newcommand{\bbD}{{\mathbb{D}}}
\newcommand{\bbE}{{\mathbb{E}}}
\newcommand{\bbF}{{\mathbb{F}}}
\newcommand{\bbG}{{\mathbb{G}}}
\newcommand{\bbH}{{\mathbb{H}}}
\newcommand{\bbI}{{\mathbb{I}}}
\newcommand{\bbJ}{{\mathbb{J}}}
\newcommand{\bbK}{{\mathbb{K}}}
\newcommand{\bbL}{{\mathbb{L}}}
\newcommand{\bbM}{{\mathbb{M}}}
\newcommand{\bbN}{{\mathbb{N}}}
\newcommand{\bbO}{{\mathbb{O}}}
\newcommand{\bbP}{{\mathbb{P}}}
\newcommand{\bbQ}{{\mathbb{Q}}}
\newcommand{\bbR}{{\mathbb{R}}}
\newcommand{\bbS}{{\mathbb{S}}}
\newcommand{\bbT}{{\mathbb{T}}}
\newcommand{\bbU}{{\mathbb{U}}}
\newcommand{\bbV}{{\mathbb{V}}}
\newcommand{\bbW}{{\mathbb{W}}}
\newcommand{\bbX}{{\mathbb{X}}}
\newcommand{\bbY}{{\mathbb{Y}}}
\newcommand{\bbZ}{{\mathbb{Z}}}
% commonly used shortcuts:
\newcommand{\C}{{\mathbb{C}}} % complex numbers
\newcommand{\F}{{\mathbb{F}}} % field, finite field
\newcommand{\N}{{\mathbb{N}}} % natural numbers {1, 2, ...}
\newcommand{\Q}{{\mathbb{Q}}} % rationals
\newcommand{\R}{{\mathbb{R}}} % reals
\newcommand{\Z}{{\mathbb{Z}}} % integers
% more commonly used shortcuts:
\newcommand{\CC}{{\mathbb{C}}} % complex numbers
\newcommand{\FF}{{\mathbb{F}}} % field, finite field
\newcommand{\NN}{{\mathbb{N}}} % natural numbers {1, 2, ...}
\newcommand{\QQ}{{\mathbb{Q}}} % rationals
\newcommand{\RR}{{\mathbb{R}}} % reals
\newcommand{\ZZ}{{\mathbb{Z}}} % integers
% more commonly used shortcuts:
\newcommand{\EE}{{\mathbb{E}}}
\newcommand{\PP}{{\mathbb{P}}}
\newcommand{\TT}{{\mathbb{T}}}
\newcommand{\VV}{{\mathbb{V}}}
% and even more commonly used shortcuts:
\newcommand{\Complex}{{\mathbb{C}}}
\newcommand{\Integer}{{\mathbb{Z}}}
\newcommand{\Natural}{{\mathbb{N}}}
\newcommand{\Rational}{{\mathbb{Q}}}
\newcommand{\Real}{{\mathbb{R}}}

%%% Issue with miktex
% indicator boldface 1:
%\DeclareSymbolFont{bbold}{U}{bbold}{m}{n}
%\DeclareSymbolFontAlphabet{\mathbbold}{bbold}
%\newcommand{\ind}{{\mathbbold{1}}}
%\newcommand{\bbone}{{\mathbbold{1}}}


% calligraphic letters:
\newcommand{\cala}{{\mathcal{a}}}
\newcommand{\calb}{{\mathcal{b}}}
\newcommand{\calc}{{\mathcal{c}}}
\newcommand{\cald}{{\mathcal{d}}}
\newcommand{\cale}{{\mathcal{e}}}
\newcommand{\calf}{{\mathcal{f}}}
\newcommand{\calg}{{\mathcal{g}}}
\newcommand{\calh}{{\mathcal{h}}}
\newcommand{\cali}{{\mathcal{i}}}
\newcommand{\calj}{{\mathcal{j}}}
\newcommand{\calk}{{\mathcal{k}}}
\newcommand{\call}{{\mathcal{l}}}
\newcommand{\calm}{{\mathcal{m}}}
\newcommand{\caln}{{\mathcal{n}}}
\newcommand{\calo}{{\mathcal{o}}}
\newcommand{\calp}{{\mathcal{p}}}
\newcommand{\calq}{{\mathcal{q}}}
\newcommand{\calr}{{\mathcal{r}}}
\newcommand{\cals}{{\mathcal{s}}}
\newcommand{\calt}{{\mathcal{t}}}
\newcommand{\calu}{{\mathcal{u}}}
\newcommand{\calv}{{\mathcal{v}}}
\newcommand{\calw}{{\mathcal{w}}}
\newcommand{\calx}{{\mathcal{x}}}
\newcommand{\caly}{{\mathcal{y}}}
\newcommand{\calz}{{\mathcal{z}}}
\newcommand{\calA}{{\mathcal{A}}}
\newcommand{\calB}{{\mathcal{B}}}
\newcommand{\calC}{{\mathcal{C}}}
\newcommand{\calD}{{\mathcal{D}}}
\newcommand{\calE}{{\mathcal{E}}}
\newcommand{\calF}{{\mathcal{F}}}
\newcommand{\calG}{{\mathcal{G}}}
\newcommand{\calH}{{\mathcal{H}}}
\newcommand{\calI}{{\mathcal{I}}}
\newcommand{\calJ}{{\mathcal{J}}}
\newcommand{\calK}{{\mathcal{K}}}
\newcommand{\calL}{{\mathcal{L}}}
\newcommand{\calM}{{\mathcal{M}}}
\newcommand{\calN}{{\mathcal{N}}}
\newcommand{\calO}{{\mathcal{O}}}
\newcommand{\calP}{{\mathcal{P}}}
\newcommand{\calQ}{{\mathcal{Q}}}
\newcommand{\calR}{{\mathcal{R}}}
\newcommand{\calS}{{\mathcal{S}}}
\newcommand{\calT}{{\mathcal{T}}}
\newcommand{\calU}{{\mathcal{U}}}
\newcommand{\calV}{{\mathcal{V}}}
\newcommand{\calW}{{\mathcal{W}}}
\newcommand{\calX}{{\mathcal{X}}}
\newcommand{\calY}{{\mathcal{Y}}}
\newcommand{\calZ}{{\mathcal{Z}}}

% Euler fraks:
\newcommand{\fraka}{{\mathfrak{a}}}
\newcommand{\frakb}{{\mathfrak{b}}}
\newcommand{\frakc}{{\mathfrak{c}}}
\newcommand{\frakd}{{\mathfrak{d}}}
\newcommand{\frake}{{\mathfrak{e}}}
\newcommand{\frakf}{{\mathfrak{f}}}
\newcommand{\frakg}{{\mathfrak{g}}}
\newcommand{\frakh}{{\mathfrak{h}}}
\newcommand{\fraki}{{\mathfrak{i}}}
\newcommand{\frakj}{{\mathfrak{j}}}
\newcommand{\frakk}{{\mathfrak{k}}}
\newcommand{\frakl}{{\mathfrak{l}}}
\newcommand{\frakm}{{\mathfrak{m}}}
\newcommand{\frakn}{{\mathfrak{n}}}
\newcommand{\frako}{{\mathfrak{o}}}
\newcommand{\frakp}{{\mathfrak{p}}}
\newcommand{\frakq}{{\mathfrak{q}}}
\newcommand{\frakr}{{\mathfrak{r}}}
\newcommand{\fraks}{{\mathfrak{s}}}
\newcommand{\frakt}{{\mathfrak{t}}}
\newcommand{\fraku}{{\mathfrak{u}}}
\newcommand{\frakv}{{\mathfrak{v}}}
\newcommand{\frakw}{{\mathfrak{w}}}
\newcommand{\frakx}{{\mathfrak{x}}}
\newcommand{\fraky}{{\mathfrak{y}}}
\newcommand{\frakz}{{\mathfrak{z}}}
\newcommand{\frakA}{{\mathfrak{A}}}
\newcommand{\frakB}{{\mathfrak{B}}}
\newcommand{\frakC}{{\mathfrak{C}}}
\newcommand{\frakD}{{\mathfrak{D}}}
\newcommand{\frakE}{{\mathfrak{E}}}
\newcommand{\frakF}{{\mathfrak{F}}}
\newcommand{\frakG}{{\mathfrak{G}}}
\newcommand{\frakH}{{\mathfrak{H}}}
\newcommand{\frakI}{{\mathfrak{I}}}
\newcommand{\frakJ}{{\mathfrak{J}}}
\newcommand{\frakK}{{\mathfrak{K}}}
\newcommand{\frakL}{{\mathfrak{L}}}
\newcommand{\frakM}{{\mathfrak{M}}}
\newcommand{\frakN}{{\mathfrak{N}}}
\newcommand{\frakO}{{\mathfrak{O}}}
\newcommand{\frakP}{{\mathfrak{P}}}
\newcommand{\frakQ}{{\mathfrak{Q}}}
\newcommand{\frakR}{{\mathfrak{R}}}
\newcommand{\frakS}{{\mathfrak{S}}}
\newcommand{\frakT}{{\mathfrak{T}}}
\newcommand{\frakU}{{\mathfrak{U}}}
\newcommand{\frakV}{{\mathfrak{V}}}
\newcommand{\frakW}{{\mathfrak{W}}}
\newcommand{\frakX}{{\mathfrak{X}}}
\newcommand{\frakY}{{\mathfrak{Y}}}
\newcommand{\frakZ}{{\mathfrak{Z}}}
% sets as Euler fraks:
\newcommand{\seta}{{\mathfrak{a}}}
\newcommand{\setb}{{\mathfrak{b}}}
\newcommand{\setc}{{\mathfrak{c}}}
\newcommand{\setd}{{\mathfrak{d}}}
\newcommand{\sete}{{\mathfrak{e}}}
\newcommand{\setf}{{\mathfrak{f}}}
\newcommand{\setg}{{\mathfrak{g}}}
\newcommand{\seth}{{\mathfrak{h}}}
\newcommand{\seti}{{\mathfrak{i}}}
\newcommand{\setj}{{\mathfrak{j}}}
\newcommand{\setk}{{\mathfrak{k}}}
\newcommand{\setl}{{\mathfrak{l}}}
\newcommand{\setm}{{\mathfrak{m}}}
\newcommand{\setn}{{\mathfrak{n}}}
\newcommand{\seto}{{\mathfrak{o}}}
\newcommand{\setp}{{\mathfrak{p}}}
\newcommand{\setq}{{\mathfrak{q}}}
\newcommand{\setr}{{\mathfrak{r}}}
\newcommand{\sets}{{\mathfrak{s}}}
\newcommand{\sett}{{\mathfrak{t}}}
\newcommand{\setu}{{\mathfrak{u}}}
\newcommand{\setv}{{\mathfrak{v}}}
\newcommand{\setw}{{\mathfrak{w}}}
\newcommand{\setx}{{\mathfrak{x}}}
\newcommand{\sety}{{\mathfrak{y}}}
\newcommand{\setz}{{\mathfrak{z}}}
\newcommand{\setA}{{\mathfrak{A}}}
\newcommand{\setB}{{\mathfrak{B}}}
\newcommand{\setC}{{\mathfrak{C}}}
\newcommand{\setD}{{\mathfrak{D}}}
\newcommand{\setE}{{\mathfrak{E}}}
\newcommand{\setF}{{\mathfrak{F}}}
\newcommand{\setG}{{\mathfrak{G}}}
\newcommand{\setH}{{\mathfrak{H}}}
\newcommand{\setI}{{\mathfrak{I}}}
\newcommand{\setJ}{{\mathfrak{J}}}
\newcommand{\setK}{{\mathfrak{K}}}
\newcommand{\setL}{{\mathfrak{L}}}
\newcommand{\setM}{{\mathfrak{M}}}
\newcommand{\setN}{{\mathfrak{N}}}
\newcommand{\setO}{{\mathfrak{O}}}
\newcommand{\setP}{{\mathfrak{P}}}
\newcommand{\setQ}{{\mathfrak{Q}}}
\newcommand{\setR}{{\mathfrak{R}}}
\newcommand{\setS}{{\mathfrak{S}}}
\newcommand{\setT}{{\mathfrak{T}}}
\newcommand{\setU}{{\mathfrak{U}}}
\newcommand{\setV}{{\mathfrak{V}}}
\newcommand{\setW}{{\mathfrak{W}}}
\newcommand{\setX}{{\mathfrak{X}}}
\newcommand{\setY}{{\mathfrak{Y}}}
\newcommand{\setZ}{{\mathfrak{Z}}}

% other commonly defined commands:
\newcommand{\wal}{{\rm wal}}
\newcommand{\floor}[1]{\left\lfloor #1 \right\rfloor} % floor
\newcommand{\ceil}[1]{\left\lceil #1 \right\rceil}    % ceil
\DeclareMathOperator{\cov}{Cov}
\DeclareMathOperator{\var}{Var}
\providecommand{\argmin}{\operatorname*{argmin}}
\providecommand{\argmax}{\operatorname*{argmax}}

\makeatletter
\renewcommand*\env@matrix[1][c]{\hskip -\arraycolsep
  \let\@ifnextchar\new@ifnextchar
  \array{*\c@MaxMatrixCols #1}}
\makeatother

\usepackage[frenchb]{mathlist}

\begin{document}
\frame{\titlepage}

% ------------------------------------------------------------------------------------------------------------------------------------------------------\begin{frame}

\begin{frame}
\frametitle{Problèmes de transport}

Beaucoup de problèmes linéaires présentent certaines structures qui simplifient grandement leur résolution.

\mbox{}

Supposons que nous avons $m$ origines contenant certains quantités d'une marchandise qui doit etre transportée à $n$ destinations pour satisfaires certaines demandes:
\begin{itemize}
\item
origine $i$: contient la quantité $a_i$;
\item
destination $j$: présente une demande $b_j$.
\end{itemize}

\mbox{}

Nous supposons le problème équilibré, i.e. l'offre totale est égale à la demande totale:
\[
\sum_{i = 1}^m a_i = \sum_{j = 1}^n b_j.
\]

\end{frame}

\begin{frame}
	\frametitle{Illustration}
	
	\begin{center}
		\begin{tikzpicture}[scale=0.75,->]
			\node[circle,draw] (A1) at (1,8) {$a_1$};
			\node[circle,draw] (A2) at (1,6) {$a_2$};
			\node[circle,draw] (A3) at (1,4) {$a_3$};
			\node[circle,draw] (A4) at (1,2) {$a_4$};
			\node[circle,draw] (A5) at (1,0) {$a_5$};
			\node[circle,draw] (B1) at (12,7) {$b_1$};
			\node[circle,draw] (B2) at (12,5) {$b_2$};
			\node[circle,draw] (B3) at (12,3) {$b_3$};
			\node[circle,draw] (B4) at (12,1) {$b_4$};
			\path
			(A1) edge node [above,near start] {{\tiny $c_{11}$}} (B1)
			(A1) edge node [above,near start] {{\tiny $c_{12}$}} (B2)
			(A1) edge node [above,near start] {{\tiny $c_{13}$}} (B3)
			(A1) edge node [above,near start] {{\tiny $c_{14}$}} (B4)
			(A2) edge node [above,near end] {{\tiny $c_{21}$}} (B1)
			(A2) edge node [above,near start] {{\tiny $c_{22}$}} (B2)
			(A2) edge node [above,near start] {{\tiny $c_{23}$}} (B3)
			(A2) edge node [above,near start] {{\tiny $c_{24}$}} (B4)
			(A3) edge node [above,near end] {{\tiny $c_{31}$}} (B1)
			(A3) edge node [above,near start] {{\tiny $c_{32}$}} (B2)
			(A3) edge node [above,near start] {{\tiny $c_{33}$}} (B3)
			(A3) edge node [above,near end] {{\tiny $c_{34}$}} (B4)
			(A4) edge node [above,near start] {{\tiny $c_{41}$}} (B1)
			(A4) edge node [above,near start] {{\tiny $c_{42}$}} (B2)
			(A4) edge node [above,near start] {{\tiny $c_{43}$}} (B3)
			(A4) edge node [above,near end] {{\tiny $c_{44}$}} (B4)
			(A5) edge node [above,near start] {{\tiny $c_{51}$}} (B1)
			(A5) edge node [above,near start] {{\tiny $c_{52}$}} (B2)
			(A5) edge node [above,near start] {{\tiny $c_{53}$}} (B3)
			(A5) edge node [above,near start] {{\tiny $c_{54}$}} (B4)
			;
%			\draw [->] (D) to [bend right=25] (E);
%			\draw [->] (E) to [bend right=25] (D);
		\end{tikzpicture}
	\end{center}
	
\end{frame}

\begin{frame}
\frametitle{Formulation}

Les nombres $a_i$ et $b_j$, $i = 1,2,\ldots,m$, $j = 1,2,\ldots,n$, sont supposés non-négatifs, et de plus, souvent entiers.

\mbox{}

$c_{ij}$: coût de transport d'une unité de marchandise de l'origine $i$ à la destination $j$.

\mbox{}

On veut déterminer les quantitiés à transporter pour chaque paire $(i,j)$.

\end{frame}

\begin{frame}
\frametitle{Formulation}

Programme mathématique:
\begin{align*}
\min_x \ & \sum_{i = 1}^m \sum_{j = 1}^n c_{ij} x_{ij} \\
\mbox{s.à. } & \sum_{j = 1}^n x_{ij} = a_i, \ i = 1,2,\ldots,m \\
& \sum_{i = 1}^m x_{ij} = b_j,\ j = 1,2,\ldots,n \\
& x_{ij} \geq 0,\ \forall\, i, j.
\end{align*}

\end{frame}

\begin{frame}
\frametitle{Formulation}

Réécrivons les contraintes d'égalité:
\[
\begin{matrix}[l]
x_{11} + \ldots + x_{1n} & & & & = & a_1 \\
& x_{21} + \ldots + x_{2n} & & & = &  a_2 \\
& & & & \vdots \\
& & \mbox{ } & x_{m1} + \ldots + x_{mn} & = &  a_m \\
x_{11} & + x_{21} & & + x_{m1} & = & b_1 \\
\quad \quad x_{12} & \quad \quad + x_{22} & & \quad \quad + x_{m2} & = & b_2 \\
& & & & \vdots \\
\quad \quad \quad \qquad x_{1n} &  \quad \quad \quad \qquad +x_{2n} & &  \quad \quad \quad \qquad +x_{mn} & = & b_n 
\end{matrix}
\]

\end{frame}

\begin{frame}
\frametitle{Formulation}

En d'autres termes, la matrice $\bA$ a la structure
\[
A =
\begin{pmatrix}
\bone^T \\
& \bone^T \\
& & \vdots \\
& & & \bone^T \\
\bI & \bI & \ldots & \bI
\end{pmatrix}
\]
où $I$ est la matrice identité ($n \times n$).

\mbox{}

Notation plus compacte:
\[
\begin{matrix}
\begin{matrix}
\ba = (a_1, a_2,\ldots, a_m) \\
\bb = (b_1, b_2,\ldots, a_n) \\
\end{matrix}
&
\bC = 
\begin{pmatrix}
c_{11} & c_{12} & \ldots & c_{1n} \\
c_{21} & c_{22} & \ldots & c_{2n} \\
\vdots & \vdots & \ddots & \vdots \\
c_{m1} & c_{m2} & \ldots & c_{mn} \\
\end{pmatrix}
\end{matrix}
\]

\end{frame}

\begin{frame}
\frametitle{Exemple}

\[
\begin{matrix}
\begin{matrix}
\ba = (30, 80, 10, 60) \\
\bb = (10, 50, 20, 80, 20) \\
\end{matrix}
&
\bC = 
\begin{pmatrix}
3 & 4 & 6 & 8 & 9 \\
2 & 2 & 4 & 5 & 5 \\
2 & 2 & 2 & 3 & 2 \\
3 & 3 & 2 & 4 & 2
\end{pmatrix}
\end{matrix}
\]

\mbox{}

La somme de l'offre, ainsi que de la demande, est 180.

\end{frame}

\begin{frame}
\frametitle{Généralisation}

Il est facile d'étendre la formulation au cas plus réaliste où
$$
\sum_{j = 1}^n x_{ij} \leq a_i,\ i = 1,\ldots,m, \quad
\sum_{i = 1}^m x_{ij} \geq b_j,\ j = 1,\ldots,n,
$$
donnant lieu au problème
\begin{align*}
	\min_x \ & \sum_{i = 1}^m \sum_{j = 1}^n c_{ij} x_{ij} \\
	\mbox{s.à. } & \sum_{j = 1}^n x_{ij} \leq a_i, \ i = 1,2,\ldots,m \\
	& \sum_{i = 1}^m x_{ij} \geq b_j,\ j = 1,2,\ldots,n \\
	& x_{ij} \geq 0,\ \forall\, i, j.
\end{align*}

\end{frame}

\begin{frame}
\frametitle{Généralisation}

Nous pouvons considérons un excédent de production au travers de variables d'écart qui vont désservir une destination virtuelle, à coût nul, permettant de tenir compte de ce surplus, en considérant une demande $b_{n+1} = \sum_{i = 1}^m a_i - \sum_{j = 1}^n b_j$:
\begin{align*}
	\min_x \ & \sum_{i = 1}^m \sum_{j = 1}^n c_{ij} x_{ij} \\
	\mbox{s.à. } & \sum_{j = 1}^n x_{ij} + s_{i,n+1} = a_i, \ i = 1,2,\ldots,m \\
	& \sum_{i = 1}^m x_{ij} = b_j,\ j = 1,2,\ldots,n \\
	& \sum_{i = 1}^m s_{i,n+1} = b_{n+1} \\
	& x_{ij} \geq 0,\ \forall\, i, j,\ \quad s_{i,n+1} \geq 0,\ \forall\, i.
\end{align*}

\end{frame}

\begin{frame}
\frametitle{Réalisabilité}

Soit $S = \sum_{i = 1}^m a_i = \sum_{j = 1}^n b_j$ la demande totale (et donc, l'offre totale).
La solution
$$
x^0_{ij} = \frac{a_ib_j}{S},\ i=1,2,\ldots,m,\ j = 1,2,\ldots,n,
$$
est réalisable. En effet,
\begin{align*}
\forall i,\ & \sum_{j = 1}^n x^0_{ij} = \sum_{j = 1}^n \frac{a_ib_j}{S} = a_i, \\
\forall j,\ & \sum_{i = 1}^m x^0_{ij} = \sum_{i = 1}^m \frac{a_ib_j}{S} = b_j, \\
\forall i,j,\ & x^0_{ij} \geq 0.
\end{align*}

\end{frame}

\begin{frame}
\frametitle{0ptimalité}

De plus, $\forall i, j$,
\begin{align*}
0 \leq x_{ij} \leq a_i, \\
0 \leq x_{ij} \leq b_j.
\end{align*}
Un programme avec un ensemble réalisable non vide fermé et borné (i.e. compact) a toujours une solution optimale.

\mbox{}

Plus généralement, une fonction continue sur un compact atteint son maximum et son minimum sur cet ensemble compact.

\mbox{}

Dès lors, un problème de transport a toujours une solution optimale.

\end{frame}

\begin{frame}
\frametitle{Redondance}

Nous avons un ensemble de $m+n$ contraintes linéaires.
\begin{itemize}
	\item Considérons les contraintes
	$$
	\sum_{j = 1}^n x_{ij} = a_i,\ \forall i.
	$$
	En les sommant, nous avons
	$$
	\sum_{i = 1}^m \sum_{j = 1}^n x_{ij} = \sum_{i = 1}^m a_i.
	$$
	\item Considérons à présent les contraintes
	$$
	\sum_{i = 1}^m x_{ij} = b_j,\ \forall j.
	$$
	Leur somme donne
	$$
	\sum_{i = 1}^m \sum_{j = 1}^n x_{ij} = \sum_{j = 1}^n b_j.
	$$
\end{itemize}

\end{frame}

\begin{frame}
\frametitle{Redondance}

On a formé deux combinaisons linéaires distinctes des contraintes, pour former des termes de gauche identiques (et les termes de droite sont également identiques en vertu de l'hypothèse de départ).

Prenons les $n+m-1$ dernières contraintes:
$$
\begin{cases}
\sum_{j = 1}^m x_{ij} = a_i,\ i = 2,\ldots,m,\\
\sum_{i = 1}^n x_{ij} = b_j,\ j = 1,\ldots,n.
\end{cases}
$$
La combinaison linéaire obtenue en sommant les $n$ dernières contraintes et en soustrayant les $m-1$ premières contraintes ainsi sélectionnées donne
$$
\sum_{i = 1}^m \sum_{j = 1}^n x_{ij} - 
\sum_{i = 2}^m \sum_{j = 1}^n x_{ij}
= \sum_{j =1}^n b_j - \sum_{i = 2}^m a_i
$$
ce qui est équivalent à
$$
\sum_{j = 1}^n x_{1j} = a_1.
$$

\end{frame}

\begin{frame}
\frametitle{Redondance}

Autrement dit, nous avons pu réécrire la première contrainte comme une combinaison linéaire des autres contraintes.

\mbox{}

On pourrait faire la même chose avec n'importe quelle contrainte. Il y a donc une contrainte redondante.

\mbox{}

On va établir qu'on ne peut trouver qu'une redondance, et donc ramener le problème à un ensemble de $m + n - 1$ vecteurs. Une solution de base réalisable non-dégénérée consistera de $m + n -1$ variables de base.

\end{frame}

\begin{frame}
\frametitle{Théorème}

Un problème de transport a toujours une solution, mais il y a exactement une constrainte d'égalité redondante.
Quand on retire n'importe laquelle des contraintes d'égalité, le système restant de $n+m-1$ contraintes d'égalité est linéairement indépendant.

\mbox{}

\emph{Preuve}

\mbox{}

L'existence d'une solution et la redondance ont déjà été établis.
La somme de toutes les contraintes d'origine moins la somme de toutes les contraintes de destination est égale à zéro, et n'importe quelle contrainte peut être exprimée comme une combinaison linéaire des autres. On peut donc retirer n'importe laquelle de ces contraintes. Supposons qu'on retire la dernière.

\end{frame}

\begin{frame}
\frametitle{Théorème}

Nous avons donc le système d'équations
\begin{footnotesize}
\begin{align*}
\sum_{j = 1}^n x_{1j} - a_1 & = 0  \\
\sum_{j = 1}^n x_{2j} - a_2 & = 0  \\
\quad \vdots & \\
\sum_{j = 1}^n x_{mj} - a_m & = 0  \\
\sum_{i = 1}^m x_{i1} - b_1 & = 0  \\
\quad \vdots & \\
\sum_{i = 1}^m x_{i,{n-1}} - b_{n-1} & = 0
\end{align*}
\end{footnotesize}

\end{frame}

\begin{frame}
\frametitle{Théorème}

Supposons par l'absurde qu'il existe une combinaison linéaire des équations restantes qui soit nulle.

Notons les coefficients d'une telle combinaison $\alpha_i$, $i = 1,2,\ldots,m$, et $\beta_j$, $j = 1,2,\ldots,n-1$:
\[
\sum_{i = 1}^m \alpha_i \left( \sum_{j = 1}^n x_{ij} - a_i \right) +
\sum_{j = 1}^{n-1} \beta_j \left( \sum_{i = 1}^m x_{ij} - b_j \right) = 0.
%\left( \sum_{i = 1}^m \sum_{j = 1}^n x_{ij} \right).
\]

\mbox{}

Comme nous avons écarté la dernière contrainte, $x_{in}$, $i = 1,2,\ldots,m$ apparaît seulement dans la $i^e$ équation, de sorte que
\[
\sum_{i = 1}^m \alpha_i x_{in} + \sum_{i = 1}^m \alpha_i \left( \sum_{j = 1}^{n-1} x_{ij} - a_i \right) +
\sum_{j = 1}^{n-1} \beta_j \left( \sum_{i = 1}^m x_{ij} - b_j \right) = 0.
\]
Dès lors, $\alpha_i = 0$, $i = 1, 2,\ldots,m$.

\end{frame}

\begin{frame}
\frametitle{Théorème}

Il reste
\[
\sum_{j = 1}^{n-1} \beta_j \left( \sum_{i = 1}^m x_{ij} - b_j \right) = 0.
\]

Autrement dit, chaque $x_{ij}$ n'apparaît que dans une équation (jamais si $j=n$), aussi $\beta_j = 0$, $j = 1,2,\ldots,n-1$.

\mbox{}

Dès lors, l'ensemble d'équations est linéairement indépendant.

\end{frame}

\begin{frame}
\frametitle{Découverte d'une solution réalisable de base}

Du théorème précédent, on voit qu'une base pour le problème de transport consiste de $m+n-1$ vecteurs, et qu'une solution réalisable de base-non dégénérée consiste de $m+n-1$ variables.

\mbox{}

Tableau de solution:
\begin{center}
\begin{tabular}{|c|c|c|c|c|c|}
\hline
$x_{11}$ & $x_{12}$ & $x_{13}$ & \ldots & $x_{1n}$ & $a_1$ \\
\hline
$x_{21}$ & $x_{22}$ & $x_{23}$ & \ldots & $x_{2n}$ & $a_2$ \\
\hline
$\vdots$ & $\vdots$ & $\vdots$ & $\ddots$ & $\vdots$ & $\vdots$ \\
\hline
$x_{m1}$ & $x_{m2}$ & $x_{m3}$ & \ldots & $x_{mn}$ & $a_m$ \\
\hline
$b_1$ & $b_2$ & $b_3$ & \ldots & $b_n$ & \\
\hline
\end{tabular}
\end{center}

\mbox{}

Les éléments individuels du tableau apparaissent dans des cellules et représentent une solution. Une cellule vide dénote une valeur nulle.

\end{frame}

\begin{frame}
\frametitle{Règle du coin Nord-Ouest}

\begin{description}
\item[Étape 0.]
Le tableau est créé, avec toutes les cellules vides.
\item[Étape 1.]
On sélectionne la cellule dans le coin supérieur gauche (d'où le nom de la méthode).
\item[Étape 2.]
On alloue le montant maximum réalisable compatible avec les exigences de sommes sur la ligne et la colonne impliquant cette colonne (au moins une de ces exigences sera remplie).
\item[Étape 3.]
On se déplace d'une cellule vers la droite s'il reste des exigences de ligne à satisfaire (offre). Autrement, on se déplace d'une cellule vers le bas. Si toutes les exigences sont remplies, arrêt. Sinon, retour à l'étape 2.
\end{description}

\end{frame}

\begin{frame}
\frametitle{Règle du coin Nord-Ouest: exemple}

\[
\begin{matrix}
\ba = (30, 80, 10, 60) \\
\bb = (10, 50, 20, 80, 20) \\
\end{matrix}
\]

\begin{center}
\begin{tabular}{|c|c|c|c|c|c}
\hline
10 & 20 & & & & 30 \\
\hline
& 30 & 20 & 30 & & 80 \\
\hline
& & & 10 & & 10 \\
\hline
& & & 40 & 20 & 60 \\
\hline
10 & 50 & 20 & 80 & 20 & \\
\end{tabular}
\end{center}

\end{frame}

\begin{frame}
\frametitle{Règle du coin Nord-Ouest: dégénérescence}

Il existe la possibilité qu'à un certain point, les exigences de ligne et de colonne correspondant à une cellule soient toutes deux remplies.

\mbox{}

La prochaine entrée sera alors un zéro, indiquant une solution de base dégénérée. Dans pareil cas, il y a un choix à faire quand à l'endroit où place le zéro: à droite ou en-dessous.

\mbox{}

\begin{center}
\begin{minipage}{.4\linewidth}
\begin{tabular}{|c|c|c|c|c}
\hline
30 & & & & 30 \\
\hline
20 & 20 & & & 40 \\
\hline
& 0 & 20 & & 20 \\
\hline
& & 20 & 40 & 60 \\
\hline
50 & 20 & 40 & 40 & \\
\end{tabular}
\end{minipage}
\begin{minipage}{.4\linewidth}
\begin{tabular}{|c|c|c|c|c}
\hline
30 & & & & 30 \\
\hline
20 & 20 & 0 & & 40 \\
\hline
& & 20 & & 20 \\
\hline
& & 20 & 40 & 60 \\
\hline
50 & 20 & 40 & 40 & \\
\end{tabular}
\end{minipage}
\end{center}

\end{frame}

\begin{frame}
\frametitle{Solutions de base}

\begin{itemize}
\item 
La règle du coin Nord-Ouest permet de trouver facilement une solution de base réalisable.
\item
Comment caractériser les autres solutions de base?
\item
Le simplexe peut-il se réinterpréter plus simplement dans ce cadre?
\end{itemize}

\end{frame}

\begin{frame}
\frametitle{Matrices triangulaires}

{\bf Définition.}
Une matrice carrée $M$ non singulière est dite triangulaire si elle
peut être mise sous la forme d'une matrice triangulaire inférieure au moyen d'une permutation de ses lignes et colonnes.

\mbox{}

$$
\begin{pmatrix}
1 & 2 & 0 & 1 & 0 & 2 \\
4 & 1 & 0 & 5 & 0 & 0 \\
0 & 0 & 0 & 4 & 0 & 0 \\
2 & 1 & 7 & 2 & 1 & 3 \\
2 & 3 & 2 & 0 & 0 & 3 \\
0 & 2 & 0 & 1 & 0 & 0
\end{pmatrix}
\rightarrow
\begin{pmatrix}
4 & 0 & 0 & 0 & 0 & 0 \\
1 & 2 & 0 & 0 & 0 & 0 \\
5 & 1 & 4 & 0 & 0 & 0 \\
1 & 2 & 1 & 2 & 0 & 0 \\
0 & 3 & 2 & 3 & 2 & 0 \\
2 & 1 & 2 & 3 & 7 & 1
\end{pmatrix}
$$

\end{frame}

\begin{frame}
\frametitle{Théorème de triangularité de base}

\textcolor{red}{\sl Chaque base du problème de transport est triangulaire.}

\mbox{}

Pour le voir, repartons du système de contraintes
\[
\begin{matrix}[l]
x_{11} + \ldots + x_{1n} & & & & = & a_1 \\
& x_{21} + \ldots + x_{2n} & & & = &  a_2 \\
& & & & \vdots \\
& & \mbox{ } & x_{m1} + \ldots + x_{mn} & = &  a_m \\
x_{11} & + x_{21} & & + x_{m1} & = & b_1 \\
\quad \quad x_{12} & \quad \quad + x_{22} & & \quad \quad + x_{m2} & = & b_2 \\
& & & & \vdots \\
\quad \quad \quad \qquad x_{1n} &  \quad \quad \quad \qquad +x_{2n} & &  \quad \quad \quad \qquad +x_{mn} & = & b_n 
\end{matrix}
\]

\end{frame}

\begin{frame}
\frametitle{Théorème de triangularité de base}

Changeons le signe de la demi-partie supérieure du système; la matrice de coefficients consiste d'entrées égales à +1, -1 ou 0.
\[
\begin{matrix}[l]
-x_{11} - \ldots - x_{1n} & & & & = & -a_1 \\
& -x_{21} - \ldots - x_{2n} & & & = &  -a_2 \\
& & & & \vdots \\
& & \mbox{ } & - x_{m1} - \ldots - x_{mn} & = &  -a_m \\
x_{11} & + x_{21} & & + x_{m1} & = & b_1 \\
\quad \quad x_{12} & \quad \quad + x_{22} & & \quad \quad + x_{m2} & = & b_2 \\
& & & & \vdots \\
\quad \quad \quad \qquad x_{1n} &  \quad \quad \quad \qquad +x_{2n} & &  \quad \quad \quad \qquad +x_{mn} & = & b_n 
\end{matrix}
\]

On peut également supprimer n'importe quelle de ces équations pour éliminer la redondance.

\end{frame}

\begin{frame}
\frametitle{Théorème de triangularité de base}

De la matrice de coefficients résultante, on forme une base $\bB$ en sélectionnant un sous-ensemble non singulier de $m+n-1$ colonnes.

\mbox{}

Chaque colonne de $\bB$ contient au plus deux entrées non-nulles: un +1 et un -1. Dès lors, il y a au plus $2(m+n-1)$ entrées non nulles dans la base. 

\mbox{}

Cependant, si chaque colonne contient deux entrées non nulles, alors la somme de toutes les lignes serait zéro, contredisant la non-singularité de $\bB$.

\mbox{}

Dès lors, au moins une colonne de $\bB$ doit contenir seulement une seule entrée non nulle. Ceci signifie que le nombre totale d'entrées non nulles dans $\bB$ est inférieur à $2(m+n-1)$.

\end{frame}

\begin{frame}
\frametitle{Théorème de triangularité de base}

Dès lors, il y a au moins une ligne avec seulement une entrée non-nulle, que l'on peut isoler pour créer la première ligne de la matrice triangulaire.

\mbox{}

Un argument similaire peut être appliqué à la sous-matrice de $\bB$ obtenue en supprimant la ligne contenant une seule entrée non nulle et la colonne correspondant à cette entrée. Cette sous-matrice doit également contenir une ligne avec une seule entrée non-nulle. On repète l'argument jusqu'à obtenir $\bB$ triangulaire.

\end{frame}

\begin{frame}
\frametitle{Théorème de triangularité de base: illustration}

Considérons la solution réalisable
\begin{center}
\begin{tabular}{|c|c|c|c|c|c}
\hline
10 & 20 & & & & 30 \\
\hline
& 30 & 20 & 30 & & 80 \\
\hline
& & & 10 & & 10 \\
\hline
& & & 40 & 20 & 60 \\
\hline
10 & 50 & 20 & 80 & 20 & \\
\end{tabular}
\end{center}

Nous allons réexprimer ce système en triangularisation la matrice associée.

\end{frame}

\begin{frame}
\frametitle{Procédure de triangularisation}

\begin{description}
	\item[\textcolor{blue}{Étape 1.}]
Trouver une ligne avec exactement une entrée différente de zéro.
	\item[\textcolor{blue}{Étape 2.}]
Former une sous-matrice de la matrice utilisée à l'étape 1 en barrant la ligne
trouvée à l'étape 1 et la colonne correspondant à l'entrée différente de zéro dans cette ligne.
Revenir à l'étape 1 avec cette sous-matrice.
\end{description}

\end{frame}

\begin{frame}
\frametitle{Théorème de triangularité de base: illustration}

Sont dans la base: $x_{11}, x_{12}, x_{22}, x_{23}, x_{24}, x_{34}, x_{44}, x_{45}$, pour le système d'équations
\begin{align*}
x_{11} + x_{12} &= 30 \\
x_{22} + x_{23} + x_{24} &= 80 \\
x_{34} &= 10 \\
x_{44} + x_{45} &= 60 \\
x_{11} & = 10 \\
x_{12} + x_{22} &= 50 \\
x_{23} &= 20 \\
x_{24} + x_{34} + x_{44} &= 80 \\
\cancel{x_{45}} &\cancel{= 20}
\end{align*}

\end{frame}

\begin{frame}
\frametitle{Théorème de triangularité de base: illustration}

Sous forme matricielle, en laissant tomber la dernière contrainte, nous avons
\[
\begin{pmatrix}
 1 & 1 & 0 & 0 & 0 & 0 & 0 & 0 \\
 0 & 0 & 1 & 1 & 1 & 0 & 0 & 0 \\
 0 & 0 & 0 & 0 & 0 & 1 & 0 & 0 \\
 0 & 0 & 0 & 0 & 0 & 0 & 1 & 1 \\
 1 & 0 & 0 & 0 & 0 & 0 & 0 & 0 \\
 0 & 1 & 1 & 0 & 0 & 0 & 0 & 0 \\
 0 & 0 & 0 & 1 & 0 & 0 & 0 & 0 \\
 0 & 0 & 0 & 0 & 1 & 1 & 1 & 0
\end{pmatrix}
\begin{pmatrix}
x_{11} \\
x_{12} \\
x_{22} \\
x_{23} \\
x_{24} \\
x_{34} \\
x_{44} \\
x_{45}
\end{pmatrix}
=
\begin{pmatrix}
30 \\
80 \\
10 \\
60 \\
10 \\
50 \\
20 \\
80
\end{pmatrix}
\]

\end{frame}

\begin{frame}
\frametitle{Théorème de triangularité de base: illustration}

La cinquième ligne a un seul élément non-nul. Nous la permutons avec la première ligne, ce qui donne
\[
\begin{pmatrix}
 1 & 0 & 0 & 0 & 0 & 0 & 0 & 0 \\
 0 & 0 & 1 & 1 & 1 & 0 & 0 & 0 \\
 0 & 0 & 0 & 0 & 0 & 1 & 0 & 0 \\
 0 & 0 & 0 & 0 & 0 & 0 & 1 & 1 \\
 1 & 1 & 0 & 0 & 0 & 0 & 0 & 0 \\
 0 & 1 & 1 & 0 & 0 & 0 & 0 & 0 \\
 0 & 0 & 0 & 1 & 0 & 0 & 0 & 0 \\
 0 & 0 & 0 & 0 & 1 & 1 & 1 & 0
\end{pmatrix}
\begin{pmatrix}
x_{11} \\
x_{12} \\
x_{22} \\
x_{23} \\
x_{24} \\
x_{34} \\
x_{44} \\
x_{45}
\end{pmatrix}
=
\begin{pmatrix}
10 \\
80 \\
10 \\
60 \\
30 \\
50 \\
20 \\
80
\end{pmatrix}
\]

\end{frame}

\begin{frame}
\frametitle{Théorème de triangularité de base: illustration}

La première ligne est fixée, et nous devons identifier une ligne avec un seul élément non nul dans les colonnes 2 à 8.
C'est le cas de la septième ligne, que nous permutons avec la première.
\[
\begin{pmatrix}
 1 & 0 & 0 & 0 & 0 & 0 & 0 & 0 \\
 0 & 0 & 0 & 1 & 0 & 0 & 0 & 0 \\
 0 & 0 & 0 & 0 & 0 & 1 & 0 & 0 \\
 0 & 0 & 0 & 0 & 0 & 0 & 1 & 1 \\
 1 & 1 & 0 & 0 & 0 & 0 & 0 & 0 \\
 0 & 1 & 1 & 0 & 0 & 0 & 0 & 0 \\
 0 & 0 & 1 & 1 & 1 & 0 & 0 & 0 \\
 0 & 0 & 0 & 0 & 1 & 1 & 1 & 0
\end{pmatrix}
\begin{pmatrix}
x_{11} \\
x_{12} \\
x_{22} \\
x_{23} \\
x_{24} \\
x_{34} \\
x_{44} \\
x_{45}
\end{pmatrix}
=
\begin{pmatrix}
10 \\
20 \\
10 \\
60 \\
30 \\
50 \\
80 \\
80
\end{pmatrix}
\]

\end{frame}

\begin{frame}
\frametitle{Théorème de triangularité de base: illustration}

Permutons maintenant la deuxième et la quatrième colonnes pour allumer le 1 sur la deuxième colonne. L'ordre des variables s'en trouve aussi modifiée.
\[
\begin{pmatrix}
 1 & 0 & 0 & 0 & 0 & 0 & 0 & 0 \\
 0 & 1 & 0 & 0 & 0 & 0 & 0 & 0 \\
 0 & 0 & 0 & 0 & 0 & 1 & 0 & 0 \\
 0 & 0 & 0 & 0 & 0 & 0 & 1 & 1 \\
 1 & 0 & 0 & 1 & 0 & 0 & 0 & 0 \\
 0 & 0 & 1 & 1 & 0 & 0 & 0 & 0 \\
 0 & 1 & 1 & 0 & 1 & 0 & 0 & 0 \\
 0 & 0 & 0 & 0 & 1 & 1 & 1 & 0
\end{pmatrix}
\begin{pmatrix}
x_{11} \\
x_{23} \\
x_{22} \\
x_{12} \\
x_{24} \\
x_{34} \\
x_{44} \\
x_{45}
\end{pmatrix}
=
\begin{pmatrix}
10 \\
20 \\
10 \\
60 \\
30 \\
50 \\
80 \\
80
\end{pmatrix}
\]

\end{frame}

\begin{frame}
\frametitle{Théorème de triangularité de base: illustration}

Nous continuons en permutant la troisième et la sixième colonne.
\[
\begin{pmatrix}
 1 & 0 & 0 & 0 & 0 & 0 & 0 & 0 \\
 0 & 1 & 0 & 0 & 0 & 0 & 0 & 0 \\
 0 & 0 & 1 & 0 & 0 & 0 & 0 & 0 \\
 0 & 0 & 0 & 0 & 0 & 0 & 1 & 1 \\
 1 & 0 & 0 & 1 & 0 & 0 & 0 & 0 \\
 0 & 0 & 0 & 1 & 0 & 1 & 0 & 0 \\
 0 & 1 & 0 & 0 & 1 & 1 & 0 & 0 \\
 0 & 0 & 1 & 0 & 1 & 0 & 1 & 0
\end{pmatrix}
\begin{pmatrix}
x_{11} \\
x_{23} \\
x_{34} \\
x_{12} \\
x_{24} \\
x_{22} \\
x_{44} \\
x_{45}
\end{pmatrix}
=
\begin{pmatrix}
10 \\
20 \\
10 \\
60 \\
30 \\
50 \\
80 \\
80
\end{pmatrix}
\]

\end{frame}

\begin{frame}
\frametitle{Théorème de triangularité de base: illustration}

Permutons à présent la cinquième et la quatrième ligne.
\[
\begin{pmatrix}
 1 & 0 & 0 & 0 & 0 & 0 & 0 & 0 \\
 0 & 1 & 0 & 0 & 0 & 0 & 0 & 0 \\
 0 & 0 & 1 & 0 & 0 & 0 & 0 & 0 \\
 1 & 0 & 0 & 1 & 0 & 0 & 0 & 0 \\
 0 & 0 & 0 & 0 & 0 & 0 & 1 & 1 \\
 0 & 0 & 0 & 1 & 0 & 1 & 0 & 0 \\
 0 & 1 & 0 & 0 & 1 & 1 & 0 & 0 \\
 0 & 0 & 1 & 0 & 1 & 0 & 1 & 0
\end{pmatrix}
\begin{pmatrix}
x_{11} \\
x_{23} \\
x_{34} \\
x_{12} \\
x_{24} \\
x_{22} \\
x_{44} \\
x_{45}
\end{pmatrix}
=
\begin{pmatrix}
10 \\
20 \\
10 \\
30 \\
60 \\
50 \\
80 \\
80
\end{pmatrix}
\]

\end{frame}

\begin{frame}
\frametitle{Théorème de triangularité de base: illustration}

Permutons à présent la sixième et la cinquième ligne.
\[
\begin{pmatrix}
 1 & 0 & 0 & 0 & 0 & 0 & 0 & 0 \\
 0 & 1 & 0 & 0 & 0 & 0 & 0 & 0 \\
 0 & 0 & 1 & 0 & 0 & 0 & 0 & 0 \\
 1 & 0 & 0 & 1 & 0 & 0 & 0 & 0 \\
 0 & 0 & 0 & 1 & 0 & 1 & 0 & 0 \\
 0 & 0 & 0 & 0 & 0 & 0 & 1 & 1 \\
 0 & 1 & 0 & 0 & 1 & 1 & 0 & 0 \\
 0 & 0 & 1 & 0 & 1 & 0 & 1 & 0
\end{pmatrix}
\begin{pmatrix}
x_{11} \\
x_{23} \\
x_{34} \\
x_{12} \\
x_{24} \\
x_{22} \\
x_{44} \\
x_{45}
\end{pmatrix}
=
\begin{pmatrix}
10 \\
20 \\
10 \\
30 \\
50 \\
60 \\
80 \\
80
\end{pmatrix}
\]

\end{frame}

\begin{frame}
\frametitle{Théorème de triangularité de base: illustration}

Nous devons permuter la cinquième et la sixième colonnes.
\[
\begin{pmatrix}
 1 & 0 & 0 & 0 & 0 & 0 & 0 & 0 \\
 0 & 1 & 0 & 0 & 0 & 0 & 0 & 0 \\
 0 & 0 & 1 & 0 & 0 & 0 & 0 & 0 \\
 1 & 0 & 0 & 1 & 0 & 0 & 0 & 0 \\
 0 & 0 & 0 & 1 & 1 & 0 & 0 & 0 \\
 0 & 0 & 0 & 0 & 0 & 0 & 1 & 1 \\
 0 & 1 & 0 & 0 & 1 & 1 & 0 & 0 \\
 0 & 0 & 1 & 0 & 0 & 1 & 1 & 0
\end{pmatrix}
\begin{pmatrix}
x_{11} \\
x_{23} \\
x_{34} \\
x_{12} \\
x_{22} \\
x_{24} \\
x_{44} \\
x_{45}
\end{pmatrix}
=
\begin{pmatrix}
10 \\
20 \\
10 \\
30 \\
50 \\
60 \\
80 \\
80
\end{pmatrix}
\]

\end{frame}

\begin{frame}
\frametitle{Théorème de triangularité de base: illustration}

Permutons à présent la sixième et la septième lignes.
\[
\begin{pmatrix}
 1 & 0 & 0 & 0 & 0 & 0 & 0 & 0 \\
 0 & 1 & 0 & 0 & 0 & 0 & 0 & 0 \\
 0 & 0 & 1 & 0 & 0 & 0 & 0 & 0 \\
 1 & 0 & 0 & 1 & 0 & 0 & 0 & 0 \\
 0 & 0 & 0 & 1 & 1 & 0 & 0 & 0 \\
 0 & 1 & 0 & 0 & 1 & 1 & 0 & 0 \\
 0 & 0 & 0 & 0 & 0 & 0 & 1 & 1 \\
 0 & 0 & 1 & 0 & 0 & 1 & 1 & 0
\end{pmatrix}
\begin{pmatrix}
x_{11} \\
x_{23} \\
x_{34} \\
x_{12} \\
x_{22} \\
x_{24} \\
x_{44} \\
x_{45}
\end{pmatrix}
=
\begin{pmatrix}
10 \\
20 \\
10 \\
30 \\
50 \\
80 \\
60 \\
80
\end{pmatrix}
\]

\end{frame}

\begin{frame}
\frametitle{Théorème de triangularité de base: illustration}

Enfin, nous permutons à présent la septième et la dernière lignes.
\[
\begin{pmatrix}
 1 & 0 & 0 & 0 & 0 & 0 & 0 & 0 \\
 0 & 1 & 0 & 0 & 0 & 0 & 0 & 0 \\
 0 & 0 & 1 & 0 & 0 & 0 & 0 & 0 \\
 1 & 0 & 0 & 1 & 0 & 0 & 0 & 0 \\
 0 & 0 & 0 & 1 & 1 & 0 & 0 & 0 \\
 0 & 1 & 0 & 0 & 1 & 1 & 0 & 0 \\
 0 & 0 & 1 & 0 & 0 & 1 & 1 & 0 \\
 0 & 0 & 0 & 0 & 0 & 0 & 1 & 1
\end{pmatrix}
\begin{pmatrix}
x_{11} \\
x_{23} \\
x_{34} \\
x_{12} \\
x_{22} \\
x_{24} \\
x_{44} \\
x_{45}
\end{pmatrix}
=
\begin{pmatrix}
10 \\
20 \\
10 \\
30 \\
50 \\
80 \\
80 \\
60 \\
\end{pmatrix}
\]
Comme nous n'avons fait que des permutations, nous avons le même système d'équations qu'au départ.

\end{frame}

\begin{frame}
\frametitle{Solutions entières}

Puisque n'importe quelle matrice de base est triangulaire, il suit que le processus de substitution en arrière impliquera simplement des additions et des soustractions de lignes et de colonnes. Aucune multiplication n'est requise.

\mbox{}

Il suit que si les lignes et les colonnes originales sont entières, les valeurs de toutes les variables de base sont entières.

\end{frame}

\begin{frame}
\frametitle{Application du simplexe}

{\sl Idée}: exploiter le fait que les bases sont triangulaires.

\mbox{}

On va décomposer le vecteur des multiplicateurs du simplexe en écrivant
\[
\lambda = (u,v)
\]
$u_i$ est associé à la $i^e$ contrainte de ligne, et $v_j$ à la $j^e$ contrainte de colonne. Une contrainte étant redondante, on peut associer une valeur arbitraire à un des multiplicateurs. On va poser $v_n = 0$.

\mbox{}

Étant donnée une base $B$, nous allons développer l'équation
\[
\lambda^T B = c_B^T.
\]
Si $x_{ij}$ est dans la base, la colonne correspondante de $A$ (et donc de $B$) aura exactement deux entrées égales à +1 (les autres sont nulles), à la $i^e$ place de la portion supérieure, et la $j^e$ place de la portion inférieure.

\end{frame}

\begin{frame}
\frametitle{Application du simplexe}

Dès lors, le produit
\[
\lambda^T a_i = c_{ij}
\]
va donner
\[
u_i + v_j = c_{ij}.
\]
Rappelons que nous avons posé $v_n = 0$. Partant de là, nous pouvons déterminer toutes les valeurs $u_i$ et $v_j$ par substitution. De plus, nous avons le résultat suivant.

\begin{mltheorem}
Si les coûts $c_{ij}$ d'un problème de transport sont tous entiers, alors (en supposant qu'un multiplicateur du simplexe est posé arbitrairement comme égal à un entier fixé), les multiplicateurs du simplexe associés à n'importe quelle base sont entiers.
\end{mltheorem}

\end{frame}

\begin{frame}
\frametitle{Multiplicateurs du simplexe}

De
\[
r_D^T = c_D^T - \lambda^T D,
\]
nous avons
\[
r_{ij} =
\begin{cases}
c_{ij} - u_i - v_j &  \text{ si $x_{ij}$ est hors base,} \\
0 & \text{ si $x_{ij}$ est variable de base.}
\end{cases}
\]
On peut facilement calculer les coûts réduits en considérant le tableau
\begin{center}
\begin{tabular}{|c|c|c|c|c|c|}
\hline
$c_{11}$ & $c_{12}$ & $c_{13}$ & \ldots & $c_{1n}$ & $u_1$ \\
\hline
$c_{21}$ & $c_{22}$ & $c_{23}$ & \ldots & $c_{2n}$ & $u_2$ \\
\hline
$\vdots$ & $\vdots$ & $\vdots$ & $\ddots$ & $\vdots$ & $\vdots$ \\
\hline
$c_{m1}$ & $c_{m2}$ & $c_{m3}$ & \ldots & $c_{mn}$ & $u_m$ \\
\hline
$v_1$ & $v_2$ & $v_3$ & \ldots & $v_n$ & \\
\hline
\end{tabular}
\end{center}

\end{frame}

\begin{frame}
\frametitle{Multiplicateurs du simplexe}

\begin{description}
\item[Étape 1]
Affecter une valeur arbitraire à un des multiplicateurs.
\item[Étape 2]
Sélectionner un $c_{ij}$ associé à une variable de base $x_{ij}$ pour lequel soit $u_i$, soit $v_j$ n'a pas encore été determiné, mais l'un des deux multiplicateurs a une valeur connue.
\item[Étape 3]
Calculer le multiplicateur inconnu à partir de l'équation
\[
c_{ij} = u_i + v_j.
\]
Si tous les multiplicateurs ont été déterminés, arrêt. Sinon, retour à l'étape 2.
\end{description}

\end{frame}

\begin{frame}
\frametitle{Exemple}

En reprenant l'exemple précédant, et en posant (arbitrairement) $v_5 = 0$
\begin{center}
	\begin{tabular}{ccccc|c}
		{\bf 3} & {\bf 4} & 6 & 8 & 9 & x \\
		2 & {\bf 2} & {\bf 4} & {\bf 5} & 5 & x \\
		2 & 2 & 2 & {\bf 3} & 2 & x \\
		3 & 3 & 2 & {\bf 4} & {\bf 2} & x \\
		\hline
		x & x & x & x & 0 & 
	\end{tabular}
\end{center}
On parcourt alors le tableau à la recherche d'un élément dont un seul multiplicateur est inconnu. Il s'agit ici de l'élément inférieur droit, ce qui donne
\begin{center}
	\begin{tabular}{ccccc|c}
		{\bf 3} & {\bf 4} & 6 & 8 & 9 & x \\
		2 & {\bf 2} & {\bf 4} & {\bf 5} & 5 & x \\
		2 & 2 & 2 & {\bf 3} & 2 & x \\
		3 & 3 & 2 & {\bf 4} & {\bf 2} & 2 \\
		\hline
		x & x & x & x & 0 & 
	\end{tabular}
\end{center}

\end{frame}

\begin{frame}
\frametitle{Exemple}

En continuant de la sorte, nous obtenons le tableau
\begin{center}
\begin{tabular}{ccccc|c}
{\bf 3} & {\bf 4} & 6 & 8 & 9 & 5 \\
2 & {\bf 2} & {\bf 4} & {\bf 5} & 5 & 3 \\
2 & 2 & 2 & {\bf 3} & 2 & 1 \\
3 & 3 & 2 & {\bf 4} & {\bf 2} & 2 \\
\hline
-2 & -1 & 1 & 2 & 0 & 
\end{tabular}
\end{center}

%Nous observons que certains coûts réduits sont négatifs, et par conséquent, nous allons devoir échanger une variable de base et une variable hors base.

\end{frame}

\begin{frame}
\frametitle{Expression des colonnes hors base}

\begin{mltheorem}
Soit $B$ une base de $A$ (en ignorant une ligne) et soit $d$ une colonne hors-base de $A$.
Si $y = B^{-1}d$, alors $y_i \in \{ -1, 0, +1 \}$, $i = 1,\ldots,n$.
\end{mltheorem}

\begin{itemize}
	\item 
Lorsqu'une unité d'une nouvelle variable est ajoutée, les variables de base actuelles changeront chacune de $+1$, $-1$, ou $0$.
	\item 
Si la nouvelle variable a une valeur $\theta$, les variables de base
change de +$\theta$, -$\theta$, ou 0. Il suffit donc de déterminer les signes de
changement.
	\item 
Cela revient à redistribuer les flux sur le réseau.
\end{itemize}

\end{frame}

\begin{frame}
\frametitle{Échange de variables: exemple}

Considérons le tableau où le + indique que nous faisons entrer la variable $x_{53}$.
$$
	\begin{array}{|c|c|c|c|c|c}
		\hline
		& & 10^0 & & & 10 \\
		\hline
		& & 20^- & & 10^+ & 30 \\
		\hline
		20^+ & 10^0 & & & 30^- & 60 \\
		\hline
		10^0 & & & & & 10 \\
		\hline
        10^- & & + & 40^0 & & 50 \\
		\hline
		40 & 10 & 30 & 40 & 40 & \\
	\end{array}
$$
En déterminant quelles lignes/colonnes n'ont qu'une entrée dont le signe de changement n'est pas déterminé, nous voyons que l'ordre de considération des variables est $x_{13}$, $x_{23}$, $x_{25}$, $x_{35}$, $x_{32}$, $x_{31}$, $x_{41}$, $x_{51}$, $x_{54}$.

La plus petite variable avec le signe moins est $x_{51} = 10$, aussi $\theta = 10$.
\end{frame}

\begin{frame}
	\frametitle{Algorithme}
	
Nous sommes à présent en mesure de décrire l'algorithme complet.

\begin{description}
\item[Étape 1]
Calculer une solution réalisable de base initiale à l'aide de la règle du coin nord-ouest ou une autre méthode.
\item[Étape 2]
Calculer les multiplicateurs du simplexe et les coefficients de coût relatifs.
Si les coefficients de coût relatifs sont non négatifs, arrêt: la solution est optimale. Autrement,
passer à l'étape 3.
\item[Étape 3]
Sélectionner une variable hors base de coût réduit négatif pour entrer dans la base.
Calculer le cycle de changement et poser la valeur de la variable à la plus petite variable de base avec un moins qui lui est attribuée.
Mettre à jour la solution. Retour à l'étape 2.
\end{description}

\end{frame}

\begin{frame}
\frametitle{Exemple: coûts réduits}

Repartons de l'exemple de départ. Le tableau des multiplicateurs du simplexe donnait
\begin{center}
	\begin{tabular}{ccccc|c}
		{\bf 3} & {\bf 4} & 6 & 8 & 9 & 5 \\
		2 & {\bf 2} & {\bf 4} & {\bf 5} & 5 & 3 \\
		2 & 2 & 2 & {\bf 3} & 2 & 1 \\
		3 & 3 & 2 & {\bf 4} & {\bf 2} & 2 \\
		\hline
		-2 & -1 & 1 & 2 & 0 & 
	\end{tabular}
\end{center}
Les coûts réduits sont obtenus en calculant $r_{ij} = c_{ij} - u_i - v_j$, menant au tableau.
\begin{center}
	\begin{tabular}{ccccc|c}
		{\bf 0} & {\bf 0} & 0 & 1 & 4 & 5 \\
		1 & {\bf 0} & {\bf 0} & {\bf 0} & 2 & 3 \\
		4 & 2 & 0 & {\bf 0} & 1 & 1 \\
		3 & 2 & -1 & {\bf 0} & {\bf 0} & 2 \\
		\hline
		-2 & -1 & 1 & 2 & 0 & 
	\end{tabular}
\end{center}

\end{frame}

\begin{frame}
\frametitle{Exemple: échange de variables}

Il y a un seul coût réduit négatif, et donc nous faisons entrer $x_{43}$ dans la base, en utilisant le tableau ci-dessous.
$$
\begin{array}{|c|c|c|c|c|c}
\hline
10 & 20 & & & & 30 \\
\hline
& 30 & 20^- & 30^+ & & 80 \\
\hline
& & & 10^0 & & 10 \\
\hline
& & + & 40^- & 20^0 & 60 \\
\hline
10 & 50 & 20 & 80 & 20 &
\end{array}
$$

Note: on peut s'arrêter dès qu'un cycle est déterminé.

\end{frame}

\begin{frame}
\frametitle{Exemple: échange de variables}

La plus petite variable de base avec le signe moins est $x_{23} = 20$. Nous échangeons donc $x_{43}$ et $x_{23}$ pour obtenir
$$
\begin{array}{|c|c|c|c|c|c}
	\hline
	10 & 20 & & & & 30 \\
	\hline
	& 30 & & 50 & & 80 \\
	\hline
	& & & 10 & & 10 \\
	\hline
	& & 20 & 20 & 20 & 60 \\
	\hline
	10 & 50 & 20 & 80 & 20 &
\end{array}
$$

\end{frame}

\begin{frame}
\frametitle{Exemple: coûts réduits}

Nous calculons d'abord les multiplicateurs associées à la nouvelle base.
\begin{center}
	\begin{tabular}{ccccc|c}
		{\bf 3} & {\bf 4} & 6 & 8 & 9 & 5 \\
		2 & {\bf 2} & 4 & {\bf 5} & 5 & 3 \\
		2 & 2 & 2 & {\bf 3} & 2 & 1 \\
		3 & 3 & {\bf 2} & {\bf 4} & {\bf 2} & 2 \\
		\hline
		-2 & -1 & 0 & 2 & 0 & 
	\end{tabular}
\end{center}
Les coûts réduits sont
\begin{center}
	\begin{tabular}{ccccc|c}
		{\bf 0} & {\bf 0} & 1 & 1 & 4 & 5 \\
		1 & {\bf 0} & 1 & {\bf 0} & 2 & 3 \\
		3 & 2 & 1 & {\bf 0} & 1 & 1 \\
		3 & 2 & {\bf 0} & {\bf 0} & {\bf 0} & 2 \\
		\hline
		-2 & -1 & 0 & 2 & 0 & 
	\end{tabular}
\end{center}
Aucun coûts réduits négatif: la solution est optimale.

\end{frame}

\end{document}
