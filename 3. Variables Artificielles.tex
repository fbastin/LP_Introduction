\documentclass[usepdftitle=false]{beamer}
\usepackage[utf8]{inputenc}
\usetheme{Singapore}
\usepackage{xcolor}
% \setbeamercovered{transparent}
%\usecolortheme{crane}

\title[Variables artificielles]{Programmation Linéaire\\Variables artificielles}
\author[Fabian Bastin]{Fabian Bastin\\DIRO\\Université de Montréal}
\date{}

\usepackage[normalem]{ulem}

\setbeamertemplate{footline}[frame number]

\def\ba{\boldsymbol{a}}
\def\bb{\boldsymbol{b}}
\def\bc{\boldsymbol{c}}
\def\be{\boldsymbol{e}}
\def\bx{\boldsymbol{x}}
\def\by{\boldsymbol{y}}
\def\bz{\boldsymbol{z}}
\def\bA{\boldsymbol{A}}
\def\bB{\boldsymbol{B}}
\def\bD{\boldsymbol{D}}
\def\bzero{\boldsymbol{0}}

\begin{document}
\frame{\titlepage}

% ------------------------------------------------------------------------------------------------------------------------------------------------------\begin{frame}

\begin{frame}
\frametitle{Motivation}

Pour débuter la méthode du simplexe, il faut une solution de base réalisable initiale.

\mbox{}

Comment trouver cette solution de base initiale?

\mbox{}

\textcolor{red}{Principe de base}
\begin{itemize}
	\item
	identifier les variables isolées;
	\item
compléter en introduisant des variables artificielles.
\end{itemize}

\end{frame}

\begin{frame}
\frametitle{Variable isolée}

Dans un système d'équations linéaires, une variable est {\em isolée} si son coefficient est égal à 1 dans une équation et à 0 dans les autres.

\mbox{}

Exemple:
$$
\begin{matrix}
x_1 & + & x_2 & + & x_3 & + &  x_4 & = & 100 \\
5x_1 & & & + & 4x_3 & & & = & 200 \\
x_1 & & & & & -  & x_4 & = & 0
\end{matrix}
$$
$x_2$ est la seule variable isolée dans ce système.

\mbox{}

Si des variables sont isolées, on pourra les identifier comme variables de base dans la base initiale. S'il y a moins de $m$ variables isolées, disons $p$, on devra identifier $m-p$ variables supplémentaires.
\end{frame}

\begin{frame}
\frametitle{Variables artificielles}

Reprenons le système
\[
\bA\bx = \bb, \ \bx \geq 0,
\]
où, s.p.d.g., $\bb \geq 0$, et supposons qu'il n'y a aucune variable isolée.

\mbox{}

Considérons le programme artificiel
\begin{align*}
\min_{\bx,\by} \ & \sum_{i = 1}^m y_i\\
\mbox{t.q. } & \bA\bx + \by = \bb,\\
& \bx \geq 0, \ \by \geq 0.
\end{align*}

$\by = (y_1, y_2,\ldots, y_m)$ est un vecteur de variables artificielles.

\end{frame}

\begin{frame}
\frametitle{Variables artificielles}

\begin{itemize}
\item 
Le programme artificiel est déjà sous forme canonique, avec comme solution de base $\by = \bb$.
\item 
Si le programme initial est réalisable, une solution optimale du programme artificiel est $\by = 0$.
En effet, nous avons alors qu'il existe un $\bx \geq 0$ tel que $\bA\bx = \bb$.
\item
Sinon, si le programme initial n'est pas réalisable, aucun $\bx \geq 0$ ne satisfait le système linéaire $\bA \bx = \bb$.
La valeur optimale du programme artificiel est $> 0$.
\item 
Si le problème est réalisable et si dans la solution finale certaines composantes de $\by$ sont toujours dans la base, nous avons une solution de base dégénérée. Si $A$ est de rang plein, on peut échanger chacun des $y_i$ avec une variable hors-base.
%Si la solution est non-dégénérée, aucune composante de $\by$ n'est encore dans la base.
\end{itemize}

\end{frame}

\begin{frame}
\frametitle{Variables artificielles: exemple}

Considérons le système
\begin{align*}
& 2x_1 + x_2 + 2x_3 = 4 \\
& 3x_1 + 3x_2 + x_3 = 3 \\
& x_1 \geq 0,\ x_2 \geq 0,\ x_3 \geq 0.
\end{align*}

\mbox{}

Variables artificielles: $x_4$, $x_5$. Le programme artificiel est
\begin{align*}
\min\ & x_4 + x_5 \\
\mbox{t.q. }
& 2x_1 + x_2 + 2x_3 + x_4 = 4 \\
& 3x_1 + 3x_2 + x_3 + x_5 = 3 \\
& x_1 \geq 0,\ x_2 \geq 0,\ x_3 \geq 0, \\
& x_4 \geq 0,\ x_5 \geq 0.
\end{align*}

\end{frame}

\begin{frame}
\frametitle{Variables artificielles: exemple}

Sous forme tableau:
\[
\begin{matrix}
& x_1 & x_2 & x_3 & x_4 & x_5 & \bb \\
& 2 & 1 & 2 & 1 & 0 & 4 \\
& 3 & 3 & 1 & 0 & 1 & 3 \\
c^T & 0 & 0 & 0 & 1 & 1 & 0
\end{matrix}
\]

Pour obtenir une solution de base réalisable, il faut annuler les coefficients de $x_4$ et $x_5$.
Nous obtenons le nouveau tableau
\[
\begin{matrix}
& x_1 & x_2 & x_3 & x_4 & x_5 & \bb \\
& 2 & 1 & 2 & 1 & 0 & 4 \\
& 3 & 3 & 1 & 0 & 1 & 3 \\
c^T & -5 & -4 & -3 & 0 & 0 & -7
\end{matrix}
\]

\mbox{}

On peut maintenant continuer la résolution du système en utilisant le simplexe, comme d'ordinaire.

\end{frame}

\begin{frame}
\frametitle{Variables artificielles: exemple}

Le premier pivotage donne
\[
\begin{matrix}
& x_1 & x_2 & x_3 & x_4 & x_5 & \bb \\
& 0 & -1 & 4/3 & 1 & -2/3 & 2 \\
& 1 & 1 & 1/3 & 0 & 1/3 & 1 \\
c^T & 0 & 1 & -4/3 & 0 & 5/3 & -2
\end{matrix}
\]

\mbox{}

Le second,
\[
\begin{matrix}
& x_1 & x_2 & x_3 & x_4 & x_5 & \bb \\
& 0 & -3/4 & 1 & 3/4 & -1/2 & 3/2 \\
& 1 & 5/4 & 0 & -1/4 & 1/2 & 1/2 \\
c^T & 0 & 0 & 0 & 1 & 1 & 0
\end{matrix}
\]

\mbox{}

Ceci donne la solution réalisable $(1/2,0,3/2)^T$.

\end{frame}

\begin{frame}
\frametitle{Méthode à deux phases}

\textbf{Phase I}\\
Recherche d'une solution de base réalisable par l'intermédiaire de variables artificielles.

\mbox{}

\textbf{Phase II}\\
Oubli des variables artificielles et résolution du programme original.

\end{frame}

\begin{frame}
\frametitle{Méthode à deux phases: exemple}

Prenons le programme
\begin{align*}
\min_x \ & 4x_1 + x_2 + x_3 \\
& 2x_1 + x_2 + 2x_3 = 4 \\
& 3x_1 + 3x_2 + x_3 = 3 \\
& x_1 \geq 0,\ x_2 \geq 0,\ x_3 \geq 0.
\end{align*}

\mbox{}

De ce qui précède, on part du tableau
\[
\begin{matrix}
& x_1 & x_2 & x_3  & \bb \\
& 0 & -3/4 & 1 & 3/2 \\
& 1 & 5/4 & 0 & 1/2 \\
c^T & 4 & 1 & 1 & 0
\end{matrix}
\]
\end{frame}

\begin{frame}
\frametitle{Méthode à deux phases: exemple}

Il faut faire apparaître des 0 comme coûts réduits associés à la base:
\[
\begin{matrix}
& x_1 & x_2 & x_3  & \bb \\
& 0 & -3/4 & 1 & 3/2 \\
& 1 & 5/4 & 0 & 1/2 \\
c^T & 0 & -13/4 & 0 & -7/2
\end{matrix}
\]

\mbox{}

Par après, on trouve
\[
\begin{matrix}
& x_1 & x_2 & x_3  & \bb \\
& 3/5 & 0 & 1 & 9/5 \\
& 4/5 & 1 & 0 & 2/5 \\
c^T & 13/5 & 0 & 0 & -11/5
\end{matrix}
\]

\mbox{}

La solution optimale est donc $(0, 2/5, 9/5)$.

\end{frame}

\begin{frame}
\frametitle{Alternative: méthode du ``big-$M$''}

Idée: plutôt que de créer un nouveau problème, on va directement intégrer les variables artificielles dans les contraintes initiales.

\mbox{}

Comment? On intègre les variables artificielles dans l'objectif initial, mais avec un coût si grand qu'elles ne peuvent qu'être mises à zéro dans une solution optimale. Pour ce faire, on les multiplie avec un certain $M$ \textsl{suffisamment} grand.

\end{frame}

\begin{frame}
\frametitle{Alternative: méthode du ``big-$M$''}

Si $\sum_{j = 1,\ldots,n} a_{ij}x_j = b_i$ et qu'aucune variable n'est isolée:
\begin{enumerate}
\item
nous ajoutons une variable artificielle $x_{n+1} \geq 0$;
\item
nous lui associons un profil très positif: $M$
\begin{align*}
\min\ & \sum_{j = 1}^{n} c_{j}x_j + M x_{n+i} \\
\mbox{t.q. } & \ldots \\
& \sum_{j = 1}^n a_{ij}x_j + x_{n+i} = b_i \\
& \ldots \\
& x_1 \geq 0, i=1,\ldots,n+1.
\end{align*}
\end{enumerate}
Si le problème est réalisable, nous devrions avoir $x_{n+i} = 0$.

\end{frame}

\end{document}
