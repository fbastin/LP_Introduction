\documentclass[usepdftitle=false]{beamer}
\usepackage[utf8]{inputenc}
\usetheme{Singapore}
\usepackage{xcolor}
% \setbeamercovered{transparent}
%\usecolortheme{crane}
\title[IFT2505]{IFT 2505\\Programmation Linéaire\\Simplexe: forme matricielle -- dictionnaires}
\author[Fabian Bastin]{Fabian Bastin\\DIRO\\Université de Montréal}
\date{}

\usepackage{ulem}
\usepackage{tikz}
\newcommand*\circled[1]{\tikz[baseline=(char.base)]{
    \node[shape=circle,draw,inner sep=2pt] (char) {#1};}}

\def\ba{\boldsymbol{a}}
\def\bb{\boldsymbol{b}}
\def\bc{\boldsymbol{c}}
\def\be{\boldsymbol{e}}
\def\br{\boldsymbol{r}}
\def\bu{\boldsymbol{u}}
\def\bx{\boldsymbol{x}}
\def\by{\boldsymbol{y}}
\def\bz{\boldsymbol{z}}
\def\bA{\boldsymbol{A}}
\def\bB{\boldsymbol{B}}
\def\bD{\boldsymbol{D}}
\def\bH{\boldsymbol{H}}
\def\bI{\boldsymbol{I}}
\def\bL{\boldsymbol{L}}
\def\bM{\boldsymbol{M}}
\def\bU{\boldsymbol{U}}
\def\bzero{\boldsymbol{0}}
\def\blambda{\boldsymbol{\lambda}}

\setbeamertemplate{footline}[frame number]

\begin{document}
\frame{\titlepage}

% ------------------------------------------------------------------------------------------------------------------------------------------------------\begin{frame}

\begin{frame}
\frametitle{Forme matricielle de la méthode du simplexe}

Utile pour mieux comprendre, et construire des variantes.

\mbox{}

Soit
\[
\bA = [ \bB\ \bD ]
\]
où nous supposons que $\bB$ est une base, et décomposons $\bx$ et $\bc$ de manière similaire:
\[
\bx = (\bx_{\bB}, \bx_{\bD}), \quad \bc = (\bc_{\bB}, \bc_{\bD}).
\]

\mbox{}

Le programme linéaire standard devient
\begin{align*}
\min_{\bx} \ & 
z = \bc_{\bB}^T\bx_{\bB} + \bc_{\bD}^T \bx_{\bD} \\
\mbox{t.q. } & \bB\bx_{\bB} + \bD\bx_{\bD} = b \\
& \bx_{\bB} \geq 0,\ \bx_{\bD} \geq 0.
\end{align*}

\end{frame}

\begin{frame}
\frametitle{Forme matricielle de la méthode du simplexe}

La solution de base associée, que nous supposons également réalisable, devient
\[
 \bx = (\bx_{\bB}, \bzero), \ \bx_{\bB} = \bB^{-1} b.
\]
Dès lors,
\[
 \bx_{\bD} = \bzero.
\]

\mbox{}

Plus généralement,
\[
  \bx_{\bB} = \bB^{-1} \bb - \bB^{-1}\bD\bx_{\bD}.
\]
et
\begin{align*}
z &= \bc_{\bB}^T\left(\bB^{-1} \bb - \bB^{-1}\bD\bx_{\bD}\right)
 + \bc_{\bD}^T \bx_{\bD} \\
 & = \bc_{\bB}^T \bB^{-1} \bb + \left( \bc_{\bD}^T - \bc_{\bB}^T \bB^{-1} \bD \right) \bx_{\bD}. 
 \end{align*}

\end{frame}

\begin{frame}
\frametitle{Dictionnaire}

L'écriture
$$
\begin{matrix}
  \bx_{\bB} & = & \bB^{-1} \bb & - \bB^{-1}\bD\bx_{\bD} \\
\hline
z & = & \bc_{\bB}^T \bB^{-1} \bb & + \left( \bc_{\bD}^T - \bc_{\bB}^T \bB^{-1} \bD \right) \bx_{\bD}. 
\end{matrix}
$$
est aussi appelée {\color{red} dictionnnnaire}.

\mbox{}

Avantages du dictionnaire par rapport aux tableaux:
\begin{itemize}
	\item forme plus compacte;
	\item implémentation numérique plus directe
\end{itemize}

\end{frame}

\begin{frame}
\frametitle{Forme matricielle de la méthode du simplexe}

Ceci permet s'exprimer n'importe quelle solution en termes de $\bx_{\bD}$. Dès lors,
\[
\br_{\bD}^T = \bc_{\bD}^T - \bc_{\bB}^T \bB^{-1} \bD
\]
est le vecteur des coûts réduits.

\mbox{}

En d'autres termes,
\[
\begin{pmatrix}
\bA & \bb \\
\bc^T & \bzero
\end{pmatrix}
=
\begin{pmatrix}
\bB & \bD & \bb \\
\bc_{\bB}^T & \bc_{\bD}^T & \bzero
\end{pmatrix}
\]

\mbox{}

Forme canonique: on multiplie la partie supérieure par $\bB^{-1}$ et on récupère l'expression de l'objectif en termes de coûts réduits pour la partie inférieure:
\[
\begin{pmatrix}
\bI & \bB^{-1}\bD & \bB^{-1}\bb \\
\bzero & \bc_{\bD}^T - \bc_{\bB}^T \bB^{-1} \bD
& -\bc_{\bB}^T \bB^{-1} \bb
\end{pmatrix}
\]
\end{frame}

\begin{frame}
\frametitle{Exemple}

Considérons le problème
\begin{align*}
\min\ & - x_1 - 2x_2 + x_3 \\
\mbox{s.à. } & 2x_1 + x_2 + x_3 \leq 14 \\
& 4x_1 + 2x_2 + 3x_3 \leq 28 \\
& 2x_1 + 5x_2 + 5x_3 \leq 30 \\
& x_1 \geq 0, x_2 \geq 0, x_3 \geq 0
\end{align*}

\end{frame}

\begin{frame}
\frametitle{Variables d'écart}

Mettons le système sous forme standard en ajoutant une variable d'écart à chacune des inégalités:
\begin{align*}
\min\ & - x_1 - 2x_2 + x_3 \\
\mbox{s.à. } & 2x_1 + x_2 + x_3 + s_1 = 14 \\
& 4x_1 + 2x_2 + 3x_3 + s_2 = 28 \\
& 2x_1 + 5x_2 + 5x_3 + s_3 = 30 \\
& x_1 \geq 0, x_2 \geq 0, x_3 \geq 0 \\
& s_1 \geq 0, s_2 \geq 0, s_3 \geq 0
\end{align*}
Le système est à présent sous forme canonique.

\end{frame}

\begin{frame}
\frametitle{Dictionnaire initial}

Nous réorganisons le système linéaire pour identifier les variables de bases ainsi que l'objectif.
\begin{align*}
\begin{array}{cccccc}
s_1 & = & 14 & - 2x_1 & - x_2 & - x_3 \\
s_2 & =	& 28 & - 4x_1 & - 2x_2 & - 3x_3 \\
s_3 & =	& 30 & - 2x_1 & - 5x_2 & - 5x_3 \\
z & = & 0 & - x_1 & - 2 x_2 & + x_3
\end{array}
\end{align*}
Un tel système est appelé {\bf dictionnaire}.

\mbox{}

Les règles de pivotage sont similaires.

\end{frame}

\begin{frame}
\frametitle{Variable entrante}

On se concentre sur la ligne de l'objectif
$$
z = 0 - x_1 - 2 x_2 + x_3
$$
Les variables $x_1$ et $x_2$ sont toutes deux associées à un coefficient négatif, et sont donc candidates pour entrer dans la base.

\mbox{}

En suivant la règle du coût réduit le plus négatif, nous sélectionnons $x_2$.

\end{frame}

\begin{frame}
\frametitle{Choix de la variable sortante}

Nous partons du dictionnaire en annulant $x_1$ et $x_3$ qui restent hors base.
\begin{align*}
\begin{array}{cccc}
s_1 & = & 14 & - x_2 \\
s_2 & =	& 28 & - 2x_2 \\
s_3 & =	& 30 & - 5x_2 \\
z & = & 0 & - 2 x_2
\end{array}
\end{align*}
Comme nous devons garder $s_1$, $s_2$, $s_3$ non-négatifs, nous avons
\begin{align*}
\begin{array}{cccc}
0 & \leq & 14 & - x_2 \\
0 & \leq & 28 & - 2x_2 \\
0 & \leq	& 30 & - 5x_2
\end{array}
\end{align*}
Dès lors, $x_2 \leq \min \{ 14, 28/2, 30/5 \} = 6$. Ainsi, $s_3$ sort de la base.

\end{frame}

\begin{frame}
\frametitle{Nouveau dictionnaire}

Le pivot s'effectue en échangeant $s_3$ et $x_2$ dans l'équation du dictionnaire correspondant à $s_3:$
\begin{align*}
s_3 & = 30 - 2x_1 - 5x_2 - 5x_3 \\
\Leftrightarrow x_2 & = 6 - \frac{2}{5}x_1 - x_3 - \frac{1}{5}s_3.
\end{align*}

Nous remplaçons ensuite $x_2$ par son expression en termes des variables hors base dans les autres équations du dictionnaire, ce qui donne
\begin{align*}
\begin{array}{cccccc}
x_2 & = & 6 & - \frac{2}{5}x_1 & - x_3 &- \frac{1}{5}s_3 \\
s_1 & = & 8 & - \frac{8}{5}x_1 &  & + \frac{1}{5}s_3 \\
s_2 & =	& 16 & - \frac{16}{5}x_1 & - x_3 & + \frac{2}{5}s_3 \\
z & = & -12 & - \frac{1}{5}x_1 & + 3 x_3 & + \frac{2}{5}s_3
\end{array}
\end{align*}

\end{frame}

\begin{frame}
\frametitle{Nouveau dictionnaire}

Seul $x_1$ a un coût réduit négatif et donc est choisi pour entrer dans la base.
En prenant les rapports entre la colonne de $x_1$ et celle du terme indépendant, nous voyons que tant $s_1$ que $s_2$ peuvent sortir de la base. Choisissons $s_1$. Le nouveau dictionnaire est
\begin{align*}
\begin{array}{cccccc}
x_1 & = & 5 & & - \frac{5}{8}s_1 & + \frac{1}{8}s_3 \\
x_2 & = & 4 & - x_3 & + \frac{1}{4}s_1 & - \frac{1}{4}s_3 \\
s_2 & =	& 0 & - x_3 & + 2s_1 &  \\
z & = & -13 & + 3 x_3 & + \frac{1}{8}s_1 & + \frac{3}{8}s_3
\end{array}
\end{align*}

Comme il n'y a plus de coût réduit négatif, nous avons convergé. La solution optimale est $x_1 = 5$, $x_2 = 4$, $x_3 = 0$.

\mbox{}

Nous allons automatiser l'approche.

\end{frame}

\begin{frame}
\frametitle{Méthode du simplexe révisée}

Converge souvent en $O(m)$.

\mbox{}

La méthode revisée ordonne les calculs afin d'éviter les opérations inutiles, en particulier pour les variables non concernées par les pivotages.

\mbox{}

Soit $\bB^{-1}$ l'inverse de la base actuelle, et la solution actuelle
\[
\bx_{\bB} = \by_0 = \bB^{-1}b.
\]

\mbox{}

{\bf Etape 1} Calculer les coefficients de coûts réduits actuels
\[
\br_{\bD}^T = \bc_{\bD}^T - \bc_{\bB}^T \bB^{-1} \bD
\]
On calcule d'abord
\[
\blambda^T = \bc^T_{\bB}\bB^{-1}
\]
puis
\[
\br_{\bD}^T = \bc_{\bD}^T - \blambda^T \bD.
\]
\end{frame}

\begin{frame}
\frametitle{Méthode du simplexe révisée}

{\bf Etape 2} Déterminer le vecteur $\ba_q$ qui va entrer dans la base en sélectionnant le coût réduit le plus négatif, et calculer
\[
 \by_q = \bB^{-1}\ba_q,
\]
donnant l'expression de $\ba_q$ en termes de la base actuelle.

\mbox{}

{\bf Etape 3} 
Si aucun $y_{iq}$ n'est $> 0$, arrêt: le problème n'est pas borné. Sinon, calculer les rapports $y_{i0}/y_{iq}$ pour $y_{iq} > 0$ pour déterminer le vecteur qui va quitter la base.

\mbox{}

{\bf Etape 4}
Mettre à jour $\bB^{-1}$ et la solution actuelle $\bB^{-1}b$. Retour à l'étape 1.

La mise à jour de $\bB^{-1}$ se fait en effectuant l'opération classique de pivotage, constituée de $\bB^{-1}$ et $\by_q$, où le pivot est l'élément approprié dans $y_q$. On en profite pour mettre à jour $\bB^{-1}\bb$.
\end{frame}

\begin{frame}
\frametitle{Exemple}

\begin{align*}
\max_x\ & 3x_1 + x_2 + 3x_3 \\
\mbox{t.q. } & 2x_1 + x_2 + x_3 \leq 2 \\
& x_1 + 2x_2 + 3x_3 \leq 5 \\
& 2x_1 +2x_2 + x_3 \leq 6 \\
& x_1 \geq 0,\ x_2 \geq 0,\ x_3 \geq 0 
\end{align*}

\mbox{}

Après ajout des variables d'écarts:
\begin{align*}
\max_x\ & 3x_1 + x_2 + 3x_3 \\
\mbox{t.q. } & 2x_1 + x_2 + x_3 + x_4 = 2 \\
& x_1 + 2x_2 + 3x_3 + x_5 = 5 \\
& 2x_1 +2x_2 + x_3 + x_6 = 6 \\
& x_1 \geq 0,\ x_2 \geq 0, x_3 \geq 0
\end{align*}

\end{frame}

\begin{frame}
\frametitle{Exemple}

Tableau:
\[
\begin{matrix}
& \ba_1 & \ba_2 & \ba_3 & \ba_4 & \ba_5 & \ba_6 & \bb \\
& 2 & 1 & 1 & 1 & 0 & 0 & 2 \\
& 1 & 2 & 3 & 0 & 1 & 0 & 5 \\
& 2 & 2 & 1 & 0 & 0 & 1 & 6 \\
\bc^T & -3 & -1 & -3 & 0 & 0 & 0 & 0
\end{matrix}
\]

\mbox{}

On se limite à
\[
\begin{matrix}
\mbox{Var} & & & & \bx_{\bB} \\
x_4 & 1 & 0 & 0 & 2 \\
x_5 & 0 & 1 & 0 & 5 \\
x_6 & 0 & 0 & 1 & 6
\end{matrix}
\]

\mbox{}

Nous avons
\[
\blambda^T = \bc_ {\bB}^T \bB^{-1} = \begin{pmatrix} 0 & 0 & 0 \end{pmatrix}\bB^{-1} =
 \begin{pmatrix} 0 & 0 & 0 \end{pmatrix}
\]
et
\[
\br_D^T = \bc_D^T - \blambda^T\bD =  \begin{pmatrix} -3 & -1 & -3 \end{pmatrix}.
\]

\end{frame}

\begin{frame}
\frametitle{Dictionnaire}

\[
\begin{matrix}
x_4 = & 2 & - 2 x_1 - x_2 - x_3 \\
x_5 = & 5 & - x_1 - 2 x_2 - 3 x_3 \\
x_6 = & 6 & - 2 x_1 - 2 x_2 - x_3 \\
\hline
z = & 0 & - 3x_1 - x_2 - 3 x_3
\end{matrix}
\]

\end{frame}

\begin{frame}
\frametitle{Exemple}

On fait entrer $\ba_2$ (violant la règle du coût le plus négatif)

\[
\begin{matrix}
\mbox{Var} & & & & x_{\bB} & y_2 \\
x_4 & 1 & 0 & 0 & 2 & \circled{1} \\
x_5 & 0 & 1 & 0 & 5 & 2 \\
x_6 & 0 & 0 & 1 & 6 & 2
\end{matrix}
\]

\mbox{}

\[
\begin{matrix}
\mbox{Var} & & & & x_{\bB} \\
x_2 & 1 & 0 & 0 & 2 \\
x_5 & -2 & 1 & 0 & 1 \\
x_6 & -2 & 0 & 1 & 2
\end{matrix}
\]

\mbox{}

Nous avons
\[
\bB^{-1} =
\begin{pmatrix}
1 & 0 & 0 \\
-2 & 1 & 0 \\
-2 & 0 & 1
\end{pmatrix}
\]

\end{frame}

\begin{frame}
\frametitle{Exemple}

Nous avons également
\[
\bc_{\bB}^T =
\begin{pmatrix}
 -1 & 0 & 0
\end{pmatrix},
\]
et dès lors
\begin{align*}
\blambda^T &= \bc_{\bB}^T \bB^{-1} \\
&=
\begin{pmatrix}
 -1 & 0 & 0
\end{pmatrix}
\begin{pmatrix}
1 & 0 & 0 \\
-2 & 1 & 0 \\
-2 & 0 & 1
\end{pmatrix}
=
\begin{pmatrix}
-1 & 0 & 0
\end{pmatrix}
\end{align*}

\mbox{}

Les coûts reduits se calculent de manière similaire
\[
\begin{pmatrix}
-3 & -3 & 0
\end{pmatrix}
- \begin{pmatrix}
-1 & 0 & 0
\end{pmatrix}
\begin{pmatrix}
2 & 1 & 1 \\
1 & 3 & 0 \\
2 & 1 & 0
\end{pmatrix}
=
\begin{pmatrix}
-1 & -2 & 1
\end{pmatrix}
\]
En d'autres termes,
\[
r_1 = -1,\ r_3 = -2,\ r_4 = 1.
\]

\end{frame}

\begin{frame}
\frametitle{Exemple}

\[
y_3 =
\begin{pmatrix}
1 & 0 & 0 \\
-2 & 1 & 0 \\
-2 & 0 & 1
\end{pmatrix}
\begin{pmatrix}
1 \\
3 \\
1
\end{pmatrix}
=
\begin{pmatrix}
1 \\
1 \\
-1
\end{pmatrix}
\]
Le variable entrante retenue est $x_3$, et on construit le tableau
\[
\begin{matrix}
\mbox{Var} & & & & x_{\bB} & y_3 \\
x_2 & 1 & 0 & 0 & 2 & 1 \\
x_5 & -2 & 1 & 0 & 1 & \circled{1} \\
x_6 & -2 & 0 & 1 & 2 & -1
\end{matrix}
\]

\mbox{}

Après le pivot:
\[
\begin{matrix}
\mbox{Var} & & & & x_{\bB} \\
x_2 & 3 & -1 & 0 & 1 \\
x_3 & -2 & 1 & 0 & 1 \\
x_6 & -4 & 1 & 1 & 3
\end{matrix}
\]

\end{frame}

\begin{frame}
\frametitle{Exemple}

\[
\blambda^T =
\begin{pmatrix}
-1 & -3 & 0
\end{pmatrix}
\begin{pmatrix}
3 & -1 & 0 \\
-2 & 1 & 0 \\
-4 & 1 & 1
\end{pmatrix}
=
\begin{pmatrix}
3 & -2 & 0
\end{pmatrix}
\]

\mbox{}

\begin{align*}
\br_{\bD}^T &=
\begin{pmatrix}
-3 & 0 & 0
\end{pmatrix}
- \begin{pmatrix}
3 & -2 & 0
\end{pmatrix}
\begin{pmatrix}
2 & 1 & 0 \\
1 & 0 & 1 \\
2 & 0 & 0
\end{pmatrix} \\
&=
\begin{pmatrix}
-3 & 0 & 0
\end{pmatrix}
- \begin{pmatrix}
4 & 3 & -2
\end{pmatrix}
=
\begin{pmatrix}
-7 & -3 & 2
\end{pmatrix}
\end{align*}

\mbox{}

On fait entrer $\ba_1$.
\[
\by_1 = 
\begin{pmatrix}
  3 & -1 & 0 \\
 -2 & 1 & 0 \\
 -4 & 1 & 1
\end{pmatrix}
\begin{pmatrix}
 2 \\
 1 \\
 2 
\end{pmatrix}
=
\begin{pmatrix}
  5 \\
 -3 \\
 -5 
\end{pmatrix}
\]

\end{frame}

\begin{frame}
\frametitle{Exemple}

\[
\begin{matrix}
\mbox{Var} & & & & x_{\bB} & y_1 \\
x_2 & 3 & -1 & 0 & 1 & \circled{5} \\
x_3 & -2 & 1 & 0 & 1 & -3 \\
x_6 & -4 & 1 & 1 & 3 & -5
\end{matrix}
\]

\mbox{}

\[
\begin{matrix}
x_1 & \frac{3}{5} & -\frac{1}{5} & 0 & \frac{1}{5} \\
x_3 & -\frac{1}{5} & \frac{2}{5} & 0 & \frac{8}{5} \\
x_6 & -1 & 0 & 1 & 4 
\end{matrix}
\]

\mbox{}

\[
\blambda^T =
\begin{pmatrix}
-3 & -3 & 0
\end{pmatrix}
\bB^{-1}
=
\begin{pmatrix}
-\frac{6}{5} & -\frac{3}{5} & 0
\end{pmatrix}
\]

\end{frame}

\begin{frame}
\frametitle{Exemple}

\begin{align*}
\br_{\bD}^T & =
\begin{pmatrix}
-1 & 0 & 0
\end{pmatrix}
-
\begin{pmatrix}
-\frac{6}{5} & -\frac{3}{5} & 0
\end{pmatrix}
\begin{pmatrix}
 1 & 1 & 0 \\
 2 & 0 & 1 \\
 2 & 0 & 0
\end{pmatrix} \\
& =
\begin{pmatrix}
-1 & 0 & 0
\end{pmatrix}
-
\begin{pmatrix}
-\frac{12}{5} & -\frac{6}{5} & -\frac{3}{5}
\end{pmatrix} \\
& =
\begin{pmatrix}
\frac{7}{5} & \frac{6}{5} & \frac{3}{5}
\end{pmatrix}
\end{align*}

\mbox{}

$\bx = ( 1/5, 0, 8/5, 0, 0, 4)$ est une solution optimale.

\end{frame}

\begin{frame}
\frametitle{Simplexe et décomposition LU}

$\bB^{-1}$ n'apparaît que dans la résolution de systèmes linéaires. Mais dans ce contexte, on ne calcule jamais $\bB^{-1}$ pour des raisons de stabilité numérique.

\mbox{}

Reformulons le simplexe pour faire apparaître les termes linéaires.

\mbox{}

{\bf Etape 1}
\[
\bx_{\bB} = \by_0,
\]
avec
\[
\bB\by_0 = b.
\]

\end{frame}

\begin{frame}
\frametitle{Simplexe et décomposition LU}

{\bf Etape 2}
Résoudre
\[
\blambda^T \bB = \bc_{\bB}^T,
\]
et
\[
\br_{\bD}^T = \bc_{\bD}^T - \blambda^T \bD.
\]
Si $\br_{\bD} \geq 0$, stop: la solution actuelle est optimale.

\mbox{}

{\bf Etape 3}
Déterminer le vecteur $\ba_q$ qui va entrer la base en sélectionnant le coefficient de coût réduit le plus négatif, et résoudre
\[
\bB \by_q = \ba_q.
\]

\mbox{}

{\bf Etape 4}
Si aucun $\by_{iq} > 0$, stop: le problème est non borné. Sinon, calculer les rapports $\by_{i0}/\by_{iq}$ pour $\by_{iq} > 0$, et sélectionner le rapport le plus négatif pour déterminer quel vecteur sortira de la base.

\end{frame}

\begin{frame}
\frametitle{Simplexe et décomposition LU}

{\bf Etape 5}
Mise à jour de $\bB$. Retour à l'étape 1.

\mbox{}

Cette manière de formuler le simplexe offre
\begin{enumerate}
\item
une meilleure stabilité numérique,
\item
des avantages de stockage mémoire (par exemple, si $\bB$ est une matrice creuse, $\bB^{-1}$ peut être pleine).
\end{enumerate}

\mbox{}

On décompose $\bB$ comme
\[
\bB = \bL.\bU
\]
où
\begin{itemize}
\item
$\bL$ est une matrice triangulaire inférieure,
\item
$\bU$ est une matrice triangulaire supérieure.
\end{itemize}

\end{frame}

\begin{frame}
\frametitle{Simplexe et décomposition}

Alors
\begin{align*}
\bB \bx & = \bb, \\
\Leftrightarrow \bL\bU\bx &= \bb \\
\Leftrightarrow \bL \by &= \bb, \quad \by = \bU\bx
\end{align*}

\mbox{}

Résoudre un système triangulaire est immédiat!

\mbox{}

\[
\begin{pmatrix}
a_{11} & & \\
a_{21} & a_{22} & \\
a_{31} & a_{32} & a_{33}
\end{pmatrix}
\begin{pmatrix}
x_1 \\ x_2 \\ x_3
\end{pmatrix}
=
\begin{pmatrix}
b_1 \\ b_2 \\ b_3
\end{pmatrix}
\]

\end{frame}

\begin{frame}
\frametitle{Résolution de système triangulaire}

\begin{align*}
x_1 &= b_1/a_{11} \\
x_2 &= (b_2 - a_{21}x_1)/a_{22} \\
x_3 &= (b_3 - a_{31}x_1 - a_{32}x_2)/a_{33}
\end{align*}

\mbox{}

Conditions: $a_{ii} \ne 0$, $\forall \, i$.

\mbox{}

Note: on ne suppose aucun échange de ligne (parfois opéré pour préserver la précision et le caractère creux).

\mbox{}

Mise à jour:
\[
\bB =
\begin{pmatrix}
\vdots & & \vdots \\
\ba_1 & \cdots & \ba_m \\
\vdots & & \vdots
\end{pmatrix}
\]

\end{frame}

\begin{frame}
\frametitle{Résolution de système triangulaire}

Nouvelle base
\[
\overline{\bB} =
\begin{pmatrix}
\vdots & \vdots & & \vdots & \vdots & & \vdots & \vdots \\
\ba_1 & \ba_2 & \cdots & \ba_{k-1} & \ba_{k+1} & \cdots & \ba_m & \ba_q \\
\vdots & \vdots & & \vdots & \vdots & & \vdots & \vdots
\end{pmatrix}
\]

\mbox{}

Alors
\begin{align*}
\bL^{-1}\overline{\bB} &=
\begin{pmatrix}
\vdots& & \vdots & \vdots & & \vdots & \vdots \\
\bL^{-1}\ba_1 & \cdots & \bL^{-1}\ba_{k-1} & \bL^{-1}\ba_{k+1} & \cdots & \bL^{-1}\ba_m & \bL^{-1}\ba_q \\
\vdots & & \vdots & \vdots & & \vdots & \vdots
\end{pmatrix} \\
&=
\begin{pmatrix}
 u_1 & \ldots & u_{k-1} & u_{k+1} & \ldots & u_m & \bL^{-1}\ba_q
\end{pmatrix}
\\
&=
\overline{H}.
\end{align*}

\end{frame}

\begin{frame}
\frametitle{Résolution de système triangulaire}

En effet
\begin{align*}
\bB &= \bL \bU \\
\Leftrightarrow
\begin{pmatrix}
\ba_1 & \ldots & \ba_m
\end{pmatrix}
&=
\bL
\begin{pmatrix}
\bu_1 & \ldots & \bu_m
\end{pmatrix} \\
\Leftrightarrow \bL^{-1}
\begin{pmatrix}
\ba_1 & \ldots & \ba_m
\end{pmatrix}
&=
\begin{pmatrix}
\bu_1 & \ldots & \bu_m
\end{pmatrix}
\end{align*}

\mbox{}

$\overline{\bH}$ a la forme
\[
\begin{pmatrix}
\times & \times & \times & \times & \times & \times \\
0 & \times & \times & \times & \times & \times \\
0 & 0 & \times & \times & \times & \times \\
0 & 0 & \times & \times & \times & \times \\
0 & 0 & 0 & \times & \times & \times \\
0 & 0 & 0 & 0 & \times & \times
\end{pmatrix}
\]

\end{frame}

\begin{frame}
\frametitle{Résolution de système triangulaire}

$\bL^{-1}\ba_q$ est un sous-produit du calcul de $\by_q$, aussi c'est "gratuit".

\mbox{}

$\overline{\bH}$ peut être ramené à une forme triangulaire supérieure grâce à une série d'éliminations de Gauss.

\mbox{}

\[
\overline{\bU} = \bM_{m-1} \bM_{m-2} \ldots \bM_k \overline{\bH}
\]
où $\bM_i$ a la forme
\[
\bM_i =
\begin{pmatrix}
1 \\
& 1 \\
& & \ddots \\
& & & 1 \\
& & & m_i & 1 \\
& & & & & \ddots \\
& & & & & & 1
\end{pmatrix}
\]

\end{frame}

\begin{frame}
\frametitle{Résolution de système triangulaire}

\[
\overline{\bB} = \overline{\bL}\overline{\bU}
\]
avec
\[
\overline{\bL} = \bL\bM_k^{-1}\ldots\bM_{m-1}^{-1}.
\]

\mbox{}

\[
\bM_i^{-1} =
\begin{pmatrix}
1 \\
& 1 \\
& & \ddots \\
& & & 1 \\
& & & -m_i & 1 \\
& & & & & \ddots \\
& & & & & & 1
\end{pmatrix}
\]

\end{frame}

\end{document}
