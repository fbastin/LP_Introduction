\documentclass[t,usepdftitle=false]{beamer}

\usepackage[utf8]{inputenc}

\usetheme{Singapore}
\usepackage{xcolor}

\setbeamertemplate{footline}[frame number]

% \setbeamercovered{transparent}
%\usecolortheme{crane}
\title[Simplexe primal-dual]{Programmation Linéaire\\Algorithme du simplexe primal-dual}
\author[Fabian Bastin]{Fabian Bastin\\DIRO\\Université de Montréal}
\date{}

\usepackage{ulem}
\usepackage{tikz}
\newcommand*\circled[1]{\tikz[baseline=(char.base)]{
    \node[shape=circle,draw,inner sep=2pt] (char) {#1};}}

% We add the following commonly used macros:

% Note: \bsc is already in babel-french; the macros below override it

% vectors as boldsymbols:
\newcommand{\bsa}{{\boldsymbol{a}}}
\newcommand{\bsb}{{\boldsymbol{b}}}
\providecommand{\bsc}{} % if \bsc is not defined, define it
\renewcommand{\bsc}{{\boldsymbol{c}}}
\newcommand{\bsd}{{\boldsymbol{d}}}
\newcommand{\bse}{{\boldsymbol{e}}}
\newcommand{\bsf}{{\boldsymbol{f}}}
\newcommand{\bsg}{{\boldsymbol{g}}}
\newcommand{\bsh}{{\boldsymbol{h}}}
\newcommand{\bsi}{{\boldsymbol{i}}}
\newcommand{\bsj}{{\boldsymbol{j}}}
\newcommand{\bsk}{{\boldsymbol{k}}}
\newcommand{\bsl}{{\boldsymbol{l}}}
\newcommand{\bsm}{{\boldsymbol{m}}}
\newcommand{\bsn}{{\boldsymbol{n}}}
\newcommand{\bso}{{\boldsymbol{o}}}
\newcommand{\bsp}{{\boldsymbol{p}}}
\newcommand{\bsq}{{\boldsymbol{q}}}
\newcommand{\bsr}{{\boldsymbol{r}}}
\newcommand{\bss}{{\boldsymbol{s}}}
\newcommand{\bst}{{\boldsymbol{t}}}
\newcommand{\bsu}{{\boldsymbol{u}}}
\newcommand{\bsv}{{\boldsymbol{v}}}
\newcommand{\bsw}{{\boldsymbol{w}}}
\newcommand{\bsx}{{\boldsymbol{x}}}
\newcommand{\bsy}{{\boldsymbol{y}}}
\newcommand{\bsz}{{\boldsymbol{z}}}
\newcommand{\bsA}{{\boldsymbol{A}}}
\newcommand{\bsB}{{\boldsymbol{B}}}
\newcommand{\bsC}{{\boldsymbol{C}}}
\newcommand{\bsD}{{\boldsymbol{D}}}
\newcommand{\bsE}{{\boldsymbol{E}}}
\newcommand{\bsF}{{\boldsymbol{F}}}
\newcommand{\bsG}{{\boldsymbol{G}}}
\newcommand{\bsH}{{\boldsymbol{H}}}
\newcommand{\bsI}{{\boldsymbol{I}}}
\newcommand{\bsJ}{{\boldsymbol{J}}}
\newcommand{\bsK}{{\boldsymbol{K}}}
\newcommand{\bsL}{{\boldsymbol{L}}}
\newcommand{\bsM}{{\boldsymbol{M}}}
\newcommand{\bsN}{{\boldsymbol{N}}}
\newcommand{\bsO}{{\boldsymbol{O}}}
\newcommand{\bsP}{{\boldsymbol{P}}}
\newcommand{\bsQ}{{\boldsymbol{Q}}}
\newcommand{\bsR}{{\boldsymbol{R}}}
\newcommand{\bsS}{{\boldsymbol{S}}}
\newcommand{\bsT}{{\boldsymbol{T}}}
\newcommand{\bsU}{{\boldsymbol{U}}}
\newcommand{\bsV}{{\boldsymbol{V}}}
\newcommand{\bsW}{{\boldsymbol{W}}}
\newcommand{\bsX}{{\boldsymbol{X}}}
\newcommand{\bsY}{{\boldsymbol{Y}}}
\newcommand{\bsZ}{{\boldsymbol{Z}}}
% other commonly used boldsymbols:
\newcommand{\bsell}{{\boldsymbol{\ell}}}
\newcommand{\bszero}{{\boldsymbol{0}}} % vector of zeros
\newcommand{\bsone}{{\boldsymbol{1}}}  % vector of ones
% boldsymbol greeks:
\newcommand{\bsalpha}{{\boldsymbol{\alpha}}}
\newcommand{\bsbeta}{{\boldsymbol{\beta}}}
\newcommand{\bsgamma}{{\boldsymbol{\gamma}}}
\newcommand{\bsdelta}{{\boldsymbol{\delta}}}
\newcommand{\bsepsilon}{{\boldsymbol{\epsilon}}}
\newcommand{\bsvarepsilon}{{\boldsymbol{\varepsilon}}}
\newcommand{\bszeta}{{\boldsymbol{\zeta}}}
\newcommand{\bseta}{{\boldsymbol{\eta}}}
\newcommand{\bstheta}{{\boldsymbol{\theta}}}
\newcommand{\bsvartheta}{{\boldsymbol{\vartheta}}}
\newcommand{\bskappa}{{\boldsymbol{\kappa}}}
\newcommand{\bslambda}{{\boldsymbol{\lambda}}}
\newcommand{\bsmu}{{\boldsymbol{\mu}}}
\newcommand{\bsnu}{{\boldsymbol{\nu}}}
\newcommand{\bsxi}{{\boldsymbol{\xi}}}
\newcommand{\bspi}{{\boldsymbol{\pi}}}
\newcommand{\bsvarpi}{{\boldsymbol{\varpi}}}
\newcommand{\bsrho}{{\boldsymbol{\rho}}}
\newcommand{\bsvarrho}{{\boldsymbol{\varrho}}}
\newcommand{\bssigma}{{\boldsymbol{\sigma}}}
\newcommand{\bsvarsigma}{{\boldsymbol{\varsigma}}}
\newcommand{\bstau}{{\boldsymbol{\tau}}}
\newcommand{\bsupsilon}{{\boldsymbol{\upsilon}}}
\newcommand{\bsphi}{{\boldsymbol{\phi}}}
\newcommand{\bsvarphi}{{\boldsymbol{\varphi}}}
\newcommand{\bschi}{{\boldsymbol{\chi}}}
\newcommand{\bspsi}{{\boldsymbol{\psi}}}
\newcommand{\bsomega}{{\boldsymbol{\omega}}}
\newcommand{\bsGamma}{{\boldsymbol{\Gamma}}}
\newcommand{\bsDelta}{{\boldsymbol{\Delta}}}
\newcommand{\bsTheta}{{\boldsymbol{\Theta}}}
\newcommand{\bsLambda}{{\boldsymbol{\Lambda}}}
\newcommand{\bsXi}{{\boldsymbol{\Xi}}}
\newcommand{\bsPi}{{\boldsymbol{\Pi}}}
\newcommand{\bsSigma}{{\boldsymbol{\Sigma}}}
\newcommand{\bsUpsilon}{{\boldsymbol{\Upsilon}}}
\newcommand{\bsPhi}{{\boldsymbol{\Phi}}}
\newcommand{\bsPsi}{{\boldsymbol{\Psi}}}
\newcommand{\bsOmega}{{\boldsymbol{\Omega}}}

% Roman fonts:
\newcommand{\rma}{{\mathrm{a}}}
\newcommand{\rmb}{{\mathrm{b}}}
\newcommand{\rmc}{{\mathrm{c}}}
\newcommand{\rmd}{{\mathrm{d}}}
\newcommand{\rme}{{\mathrm{e}}}
\newcommand{\rmf}{{\mathrm{f}}}
\newcommand{\rmg}{{\mathrm{g}}}
\newcommand{\rmh}{{\mathrm{h}}}
\newcommand{\rmi}{{\mathrm{i}}}
\newcommand{\rmj}{{\mathrm{j}}}
\newcommand{\rmk}{{\mathrm{k}}}
\newcommand{\rml}{{\mathrm{l}}}
\newcommand{\rmm}{{\mathrm{m}}}
\newcommand{\rmn}{{\mathrm{n}}}
\newcommand{\rmo}{{\mathrm{o}}}
\newcommand{\rmp}{{\mathrm{p}}}
\newcommand{\rmq}{{\mathrm{q}}}
\newcommand{\rmr}{{\mathrm{r}}}
\newcommand{\rms}{{\mathrm{s}}}
\newcommand{\rmt}{{\mathrm{t}}}
\newcommand{\rmu}{{\mathrm{u}}}
\newcommand{\rmv}{{\mathrm{v}}}
\newcommand{\rmw}{{\mathrm{w}}}
\newcommand{\rmx}{{\mathrm{x}}}
\newcommand{\rmy}{{\mathrm{y}}}
\newcommand{\rmz}{{\mathrm{z}}}
\newcommand{\rmA}{{\mathrm{A}}}
\newcommand{\rmB}{{\mathrm{B}}}
\newcommand{\rmC}{{\mathrm{C}}}
\newcommand{\rmD}{{\mathrm{D}}}
\newcommand{\rmE}{{\mathrm{E}}}
\newcommand{\rmF}{{\mathrm{F}}}
\newcommand{\rmG}{{\mathrm{G}}}
\newcommand{\rmH}{{\mathrm{H}}}
\newcommand{\rmI}{{\mathrm{I}}}
\newcommand{\rmJ}{{\mathrm{J}}}
\newcommand{\rmK}{{\mathrm{K}}}
\newcommand{\rmL}{{\mathrm{L}}}
\newcommand{\rmM}{{\mathrm{M}}}
\newcommand{\rmN}{{\mathrm{N}}}
\newcommand{\rmO}{{\mathrm{O}}}
\newcommand{\rmP}{{\mathrm{P}}}
\newcommand{\rmQ}{{\mathrm{Q}}}
\newcommand{\rmR}{{\mathrm{R}}}
\newcommand{\rmS}{{\mathrm{S}}}
\newcommand{\rmT}{{\mathrm{T}}}
\newcommand{\rmU}{{\mathrm{U}}}
\newcommand{\rmV}{{\mathrm{V}}}
\newcommand{\rmW}{{\mathrm{W}}}
\newcommand{\rmX}{{\mathrm{X}}}
\newcommand{\rmY}{{\mathrm{Y}}}
\newcommand{\rmZ}{{\mathrm{Z}}}
% also commonly defined
\newcommand{\rd}{{\mathrm{d}}}
\newcommand{\ri}{{\mathrm{i}}}

% blackboards:
\newcommand{\bbA}{{\mathbb{A}}}
\newcommand{\bbB}{{\mathbb{B}}}
\newcommand{\bbC}{{\mathbb{C}}}
\newcommand{\bbD}{{\mathbb{D}}}
\newcommand{\bbE}{{\mathbb{E}}}
\newcommand{\bbF}{{\mathbb{F}}}
\newcommand{\bbG}{{\mathbb{G}}}
\newcommand{\bbH}{{\mathbb{H}}}
\newcommand{\bbI}{{\mathbb{I}}}
\newcommand{\bbJ}{{\mathbb{J}}}
\newcommand{\bbK}{{\mathbb{K}}}
\newcommand{\bbL}{{\mathbb{L}}}
\newcommand{\bbM}{{\mathbb{M}}}
\newcommand{\bbN}{{\mathbb{N}}}
\newcommand{\bbO}{{\mathbb{O}}}
\newcommand{\bbP}{{\mathbb{P}}}
\newcommand{\bbQ}{{\mathbb{Q}}}
\newcommand{\bbR}{{\mathbb{R}}}
\newcommand{\bbS}{{\mathbb{S}}}
\newcommand{\bbT}{{\mathbb{T}}}
\newcommand{\bbU}{{\mathbb{U}}}
\newcommand{\bbV}{{\mathbb{V}}}
\newcommand{\bbW}{{\mathbb{W}}}
\newcommand{\bbX}{{\mathbb{X}}}
\newcommand{\bbY}{{\mathbb{Y}}}
\newcommand{\bbZ}{{\mathbb{Z}}}
% commonly used shortcuts:
\newcommand{\C}{{\mathbb{C}}} % complex numbers
\newcommand{\F}{{\mathbb{F}}} % field, finite field
\newcommand{\N}{{\mathbb{N}}} % natural numbers {1, 2, ...}
\newcommand{\Q}{{\mathbb{Q}}} % rationals
\newcommand{\R}{{\mathbb{R}}} % reals
\newcommand{\Z}{{\mathbb{Z}}} % integers
% more commonly used shortcuts:
\newcommand{\CC}{{\mathbb{C}}} % complex numbers
\newcommand{\FF}{{\mathbb{F}}} % field, finite field
\newcommand{\NN}{{\mathbb{N}}} % natural numbers {1, 2, ...}
\newcommand{\QQ}{{\mathbb{Q}}} % rationals
\newcommand{\RR}{{\mathbb{R}}} % reals
\newcommand{\ZZ}{{\mathbb{Z}}} % integers
% more commonly used shortcuts:
\newcommand{\EE}{{\mathbb{E}}}
\newcommand{\PP}{{\mathbb{P}}}
\newcommand{\TT}{{\mathbb{T}}}
\newcommand{\VV}{{\mathbb{V}}}
% and even more commonly used shortcuts:
\newcommand{\Complex}{{\mathbb{C}}}
\newcommand{\Integer}{{\mathbb{Z}}}
\newcommand{\Natural}{{\mathbb{N}}}
\newcommand{\Rational}{{\mathbb{Q}}}
\newcommand{\Real}{{\mathbb{R}}}

%%% Issue with miktex
% indicator boldface 1:
%\DeclareSymbolFont{bbold}{U}{bbold}{m}{n}
%\DeclareSymbolFontAlphabet{\mathbbold}{bbold}
%\newcommand{\ind}{{\mathbbold{1}}}
%\newcommand{\bbone}{{\mathbbold{1}}}


% calligraphic letters:
\newcommand{\cala}{{\mathcal{a}}}
\newcommand{\calb}{{\mathcal{b}}}
\newcommand{\calc}{{\mathcal{c}}}
\newcommand{\cald}{{\mathcal{d}}}
\newcommand{\cale}{{\mathcal{e}}}
\newcommand{\calf}{{\mathcal{f}}}
\newcommand{\calg}{{\mathcal{g}}}
\newcommand{\calh}{{\mathcal{h}}}
\newcommand{\cali}{{\mathcal{i}}}
\newcommand{\calj}{{\mathcal{j}}}
\newcommand{\calk}{{\mathcal{k}}}
\newcommand{\call}{{\mathcal{l}}}
\newcommand{\calm}{{\mathcal{m}}}
\newcommand{\caln}{{\mathcal{n}}}
\newcommand{\calo}{{\mathcal{o}}}
\newcommand{\calp}{{\mathcal{p}}}
\newcommand{\calq}{{\mathcal{q}}}
\newcommand{\calr}{{\mathcal{r}}}
\newcommand{\cals}{{\mathcal{s}}}
\newcommand{\calt}{{\mathcal{t}}}
\newcommand{\calu}{{\mathcal{u}}}
\newcommand{\calv}{{\mathcal{v}}}
\newcommand{\calw}{{\mathcal{w}}}
\newcommand{\calx}{{\mathcal{x}}}
\newcommand{\caly}{{\mathcal{y}}}
\newcommand{\calz}{{\mathcal{z}}}
\newcommand{\calA}{{\mathcal{A}}}
\newcommand{\calB}{{\mathcal{B}}}
\newcommand{\calC}{{\mathcal{C}}}
\newcommand{\calD}{{\mathcal{D}}}
\newcommand{\calE}{{\mathcal{E}}}
\newcommand{\calF}{{\mathcal{F}}}
\newcommand{\calG}{{\mathcal{G}}}
\newcommand{\calH}{{\mathcal{H}}}
\newcommand{\calI}{{\mathcal{I}}}
\newcommand{\calJ}{{\mathcal{J}}}
\newcommand{\calK}{{\mathcal{K}}}
\newcommand{\calL}{{\mathcal{L}}}
\newcommand{\calM}{{\mathcal{M}}}
\newcommand{\calN}{{\mathcal{N}}}
\newcommand{\calO}{{\mathcal{O}}}
\newcommand{\calP}{{\mathcal{P}}}
\newcommand{\calQ}{{\mathcal{Q}}}
\newcommand{\calR}{{\mathcal{R}}}
\newcommand{\calS}{{\mathcal{S}}}
\newcommand{\calT}{{\mathcal{T}}}
\newcommand{\calU}{{\mathcal{U}}}
\newcommand{\calV}{{\mathcal{V}}}
\newcommand{\calW}{{\mathcal{W}}}
\newcommand{\calX}{{\mathcal{X}}}
\newcommand{\calY}{{\mathcal{Y}}}
\newcommand{\calZ}{{\mathcal{Z}}}

% Euler fraks:
\newcommand{\fraka}{{\mathfrak{a}}}
\newcommand{\frakb}{{\mathfrak{b}}}
\newcommand{\frakc}{{\mathfrak{c}}}
\newcommand{\frakd}{{\mathfrak{d}}}
\newcommand{\frake}{{\mathfrak{e}}}
\newcommand{\frakf}{{\mathfrak{f}}}
\newcommand{\frakg}{{\mathfrak{g}}}
\newcommand{\frakh}{{\mathfrak{h}}}
\newcommand{\fraki}{{\mathfrak{i}}}
\newcommand{\frakj}{{\mathfrak{j}}}
\newcommand{\frakk}{{\mathfrak{k}}}
\newcommand{\frakl}{{\mathfrak{l}}}
\newcommand{\frakm}{{\mathfrak{m}}}
\newcommand{\frakn}{{\mathfrak{n}}}
\newcommand{\frako}{{\mathfrak{o}}}
\newcommand{\frakp}{{\mathfrak{p}}}
\newcommand{\frakq}{{\mathfrak{q}}}
\newcommand{\frakr}{{\mathfrak{r}}}
\newcommand{\fraks}{{\mathfrak{s}}}
\newcommand{\frakt}{{\mathfrak{t}}}
\newcommand{\fraku}{{\mathfrak{u}}}
\newcommand{\frakv}{{\mathfrak{v}}}
\newcommand{\frakw}{{\mathfrak{w}}}
\newcommand{\frakx}{{\mathfrak{x}}}
\newcommand{\fraky}{{\mathfrak{y}}}
\newcommand{\frakz}{{\mathfrak{z}}}
\newcommand{\frakA}{{\mathfrak{A}}}
\newcommand{\frakB}{{\mathfrak{B}}}
\newcommand{\frakC}{{\mathfrak{C}}}
\newcommand{\frakD}{{\mathfrak{D}}}
\newcommand{\frakE}{{\mathfrak{E}}}
\newcommand{\frakF}{{\mathfrak{F}}}
\newcommand{\frakG}{{\mathfrak{G}}}
\newcommand{\frakH}{{\mathfrak{H}}}
\newcommand{\frakI}{{\mathfrak{I}}}
\newcommand{\frakJ}{{\mathfrak{J}}}
\newcommand{\frakK}{{\mathfrak{K}}}
\newcommand{\frakL}{{\mathfrak{L}}}
\newcommand{\frakM}{{\mathfrak{M}}}
\newcommand{\frakN}{{\mathfrak{N}}}
\newcommand{\frakO}{{\mathfrak{O}}}
\newcommand{\frakP}{{\mathfrak{P}}}
\newcommand{\frakQ}{{\mathfrak{Q}}}
\newcommand{\frakR}{{\mathfrak{R}}}
\newcommand{\frakS}{{\mathfrak{S}}}
\newcommand{\frakT}{{\mathfrak{T}}}
\newcommand{\frakU}{{\mathfrak{U}}}
\newcommand{\frakV}{{\mathfrak{V}}}
\newcommand{\frakW}{{\mathfrak{W}}}
\newcommand{\frakX}{{\mathfrak{X}}}
\newcommand{\frakY}{{\mathfrak{Y}}}
\newcommand{\frakZ}{{\mathfrak{Z}}}
% sets as Euler fraks:
\newcommand{\seta}{{\mathfrak{a}}}
\newcommand{\setb}{{\mathfrak{b}}}
\newcommand{\setc}{{\mathfrak{c}}}
\newcommand{\setd}{{\mathfrak{d}}}
\newcommand{\sete}{{\mathfrak{e}}}
\newcommand{\setf}{{\mathfrak{f}}}
\newcommand{\setg}{{\mathfrak{g}}}
\newcommand{\seth}{{\mathfrak{h}}}
\newcommand{\seti}{{\mathfrak{i}}}
\newcommand{\setj}{{\mathfrak{j}}}
\newcommand{\setk}{{\mathfrak{k}}}
\newcommand{\setl}{{\mathfrak{l}}}
\newcommand{\setm}{{\mathfrak{m}}}
\newcommand{\setn}{{\mathfrak{n}}}
\newcommand{\seto}{{\mathfrak{o}}}
\newcommand{\setp}{{\mathfrak{p}}}
\newcommand{\setq}{{\mathfrak{q}}}
\newcommand{\setr}{{\mathfrak{r}}}
\newcommand{\sets}{{\mathfrak{s}}}
\newcommand{\sett}{{\mathfrak{t}}}
\newcommand{\setu}{{\mathfrak{u}}}
\newcommand{\setv}{{\mathfrak{v}}}
\newcommand{\setw}{{\mathfrak{w}}}
\newcommand{\setx}{{\mathfrak{x}}}
\newcommand{\sety}{{\mathfrak{y}}}
\newcommand{\setz}{{\mathfrak{z}}}
\newcommand{\setA}{{\mathfrak{A}}}
\newcommand{\setB}{{\mathfrak{B}}}
\newcommand{\setC}{{\mathfrak{C}}}
\newcommand{\setD}{{\mathfrak{D}}}
\newcommand{\setE}{{\mathfrak{E}}}
\newcommand{\setF}{{\mathfrak{F}}}
\newcommand{\setG}{{\mathfrak{G}}}
\newcommand{\setH}{{\mathfrak{H}}}
\newcommand{\setI}{{\mathfrak{I}}}
\newcommand{\setJ}{{\mathfrak{J}}}
\newcommand{\setK}{{\mathfrak{K}}}
\newcommand{\setL}{{\mathfrak{L}}}
\newcommand{\setM}{{\mathfrak{M}}}
\newcommand{\setN}{{\mathfrak{N}}}
\newcommand{\setO}{{\mathfrak{O}}}
\newcommand{\setP}{{\mathfrak{P}}}
\newcommand{\setQ}{{\mathfrak{Q}}}
\newcommand{\setR}{{\mathfrak{R}}}
\newcommand{\setS}{{\mathfrak{S}}}
\newcommand{\setT}{{\mathfrak{T}}}
\newcommand{\setU}{{\mathfrak{U}}}
\newcommand{\setV}{{\mathfrak{V}}}
\newcommand{\setW}{{\mathfrak{W}}}
\newcommand{\setX}{{\mathfrak{X}}}
\newcommand{\setY}{{\mathfrak{Y}}}
\newcommand{\setZ}{{\mathfrak{Z}}}

% other commonly defined commands:
\newcommand{\wal}{{\rm wal}}
\newcommand{\floor}[1]{\left\lfloor #1 \right\rfloor} % floor
\newcommand{\ceil}[1]{\left\lceil #1 \right\rceil}    % ceil
\DeclareMathOperator{\cov}{Cov}
\DeclareMathOperator{\var}{Var}
\providecommand{\argmin}{\operatorname*{argmin}}
\providecommand{\argmax}{\operatorname*{argmax}}

\def\bb{\boldsymbol{b}}
\def\bc{\boldsymbol{c}}
\def\be{\boldsymbol{e}}
\def\br{\boldsymbol{r}}
\def\bu{\boldsymbol{u}}
\def\bx{\boldsymbol{x}}
\def\by{\boldsymbol{y}}
\def\bz{\boldsymbol{z}}
\def\bA{\boldsymbol{A}}
\def\bB{\boldsymbol{B}}
\def\bD{\boldsymbol{D}}
\def\bH{\boldsymbol{H}}
\def\bI{\boldsymbol{I}}
\def\bL{\boldsymbol{L}}
\def\bM{\boldsymbol{M}}
\def\bU{\boldsymbol{U}}
\def\bzero{\boldsymbol{0}}
\def\bone{\boldsymbol{1}}
\def\blambda{\boldsymbol{\lambda}}

\def\cR{\mathcal{R}}

\usepackage[french]{babel}

\begin{document}
\frame{\titlepage}

% ------------------------------------------------------------------------------------------------------------------------------------------------------\begin{frame}
\begin{frame}
\frametitle{Simplexe primal-dual}

{\bf Motivations}:
\begin{itemize}
\item
Exploiter davantage la complémentarité entre le primal et le dual.
\item
Comme pour le simplexe dual, on part d'une solution dual-réalisable.
\item
\textcolor{red}{Primal restreint}: on va forcer la condition de complémentarité
\[
x_i > 0 \Rightarrow \lambda^T a_i = c_i,
\]
en faisant entrer dans la base primale les $x_i$ correspondant aux contraintes duales actives.
\item
\textcolor{red}{Dual restreint}: on optimise le dual. Si celui-ci est réalisable, augmenter (strictement) la valeur de l'objectif dual va conduire à transformer au moins une contrainte duale inactive en contrainte duale active.
\end{itemize}

\end{frame}

\begin{frame}
\frametitle{Simplexe primal-dual}

L'idée est de travailler simultanément sur le primal et le dual.

\mbox{}

Principales idées:
\begin{itemize}
\item
trouver une solution réalisable pour le dual;
\item
l'améliorer à chaque étape en optimisant un problème primal restreint associé;
\item
essayer de satisfaire les conditions d'écart de complémentarité.
\end{itemize}

\mbox{}

Il s'agit de la variante du simplexe la plus efficace pour les problèmes de flots dans les réseaux.

\end{frame}

\begin{frame}
\frametitle{Simplexe primal-dual}

Considérons à nouveau le primal (sous forme standard)
\begin{equation}
\tag{P}
\begin{aligned}
\min_x \ & c^Tx \\
\mbox{t.q. } & Ax = b \qquad (\lambda) \\
& x \geq 0
\end{aligned}
\label{eq:P}
\end{equation}
et son dual
\begin{equation}
\tag{D}
\begin{aligned}
\max_{\lambda} \ & \lambda^T b \\
\mbox{t.q. } & \lambda^T A \leq c^T.
\end{aligned}
\label{eq:D}
\end{equation}

\mbox{}

Étant donné $\lambda$ réalisable pour le dual, définissons l'ensemble actif
\[
P = \lbrace i \,|\, \lambda^T a_i = c_i \rbrace.
\]
Vu que $\lambda$ est supposé réalisable, cela implique
\[
\lambda^T a_i < c_i,\ i \notin P.
\]

\end{frame}

\begin{frame}
\frametitle{Simplexe primal-dual}

Correspondant à $\lambda$ et $P$, nous définissons le problème {\sl primal restreint associé}
\begin{equation}
\tag{PR}
\label{eq:PR}
\begin{aligned}
\min_{x,\ y} \ & \bone^Ty \\
\mbox{t.q. } & Ax + y = b  \qquad (u)\\
& x \geq 0,\\
& x_i = 0 \mbox{ pour } i \notin P\\
& y \geq 0
\end{aligned}
\end{equation}
où $\bone$ designe the vecteur $(1, 1, \ldots, 1)$.
\eqref{eq:PR} se réécrit comme
$$
\begin{aligned}
\min_{y \geq 0,\ x_i \in P} \ & \bone^Ty \\
\mbox{t.q. } & \sum_{i \in P} a_ix_i + y = b \\
& x_i \geq 0,\  i \in P.
\end{aligned}
$$

\end{frame}

\begin{frame}
\frametitle{Simplexe primal-dual}

Le dual associé est appelé {\sl dual restreint associé}
\begin{equation}
	\tag{DR}
	\label{eq:DR}
\begin{aligned}
\max_u\ & u^T b\\
\mbox{t.q. } & u^T a_i \leq 0,\ i \in P\\
& u \leq \bone.
\end{aligned}
\end{equation}

\end{frame}

\begin{frame}
\frametitle{Théorème d'optimalité primale-duale}

{\sl Supposons que $\lambda$ est réalisable pour le dual \eqref{eq:D} et que $(x,y)$ est réalisable pour le primal restreint associé \eqref{eq:PR}, avec $y = 0$ (de sorte que $(x,y)$ est une solution optimale pour \eqref{eq:PR}). Alors, $x$ et $\lambda$ sont optimaux pour les programmes primal \eqref{eq:P} et dual \eqref{eq:D} originaux respectifs.}

\begin{proof}
$x$ est clairement réalisable pour \eqref{eq:P}: $Ax = b$.
Par définition de $P$, $\lambda^T a_i = c_i,\mbox{ si } x_i \ne 0$,
aussi
$$
c^T x = \lambda^T Ax.
$$
En combinant ces deux observations, nous avons
\[
c^T x = \lambda^T b,
\]
impliquant l'optimalité de $x$ et $\lambda$.
\end{proof}

\end{frame}

\begin{frame}
\frametitle{Algorithme primal-dual}

{\bf Étape 1}
Étant donné $\lambda_0$ réalisable pour \eqref{eq:D}, déterminer le primal restreint \eqref{eq:PR} associé.

\mbox{}

{\bf Étape 2}
Optimiser \eqref{eq:PR}.
Si la valeur optimale de \eqref{eq:PR} est nulle (impliquant $y = 0$), $x$ est optimal pour \eqref{eq:P} (théorème d'optimalité primale-duale); arrêt.


\mbox{}

{\bf Étape 3}
Si la valeur optimale de \eqref{eq:PR} est strictement positive
(i.e. if $y \ne 0$), $x$ n'est pas réalisable pour le primal \eqref{eq:P}, et on cherche à améliorer la solution réalisable $\lambda$ du dual \eqref{eq:D} avant de déterminer un nouveau primal restreint \eqref{eq:PR} associé.

Obtenir du tableau du simplexe de  \eqref{eq:PR} la solution $u_0$ du dual restreint  \eqref{eq:DR}.
Si $\nexists j$ pour lequel $u_0^T a_j > 0$, \eqref{eq:P} n'a pas de solution réalisable; arrêt.
\end{frame}

\begin{frame}
\frametitle{Algorithme primal-dual}

{\bf Etape 3 (suite)}
Sinon, calculer
$$
\epsilon_0 = \min_j \left\lbrace \frac{c_j - \lambda_0^Ta_j}{u_0^T a_j} \,\bigg|\, u_0^Ta_j > 0\right\rbrace,
$$
et poser
\[
\lambda_0 := \lambda_0 + \epsilon_0 u_0.
\]
$\lambda_0$ est réalisable pour $\eqref{eq:D}$.
Retour à l'étape 1.

\end{frame}

\begin{frame}
	\frametitle{Algorithme primal-dual: calcul de $u^Ta_i$}
	
	Soit $i \in \{1,\ldots,n\}$.
	Le coefficient de $x_i$ dans \eqref{eq:PR} est égal à zero, aussi le coût à l'optimalité de $\eqref{eq:PR}$ vaut
	$$
	0 - u^Ta_i.
	$$
	
	\ 
	
	Dès lors, les valeurs de $u^Ta_i$, peuvent être directement identifiées en prenant l'opposé des valeurs dans la ligne des coûts réduits du tableau final du primal restreint, pour les colonnes associées aux $x_i$ correspondants.
	
	\
	
	Pour les variables de base $x_j$ ($> 0$), nous avons $j \in P$, et les coûts réduits sont nuls:
	$$
	0 = 0 - u^Ta_j.
	$$
	Dès lors, $u^T a_j = 0$.
	
\end{frame}

\begin{frame}
	\frametitle{Algorithme primal-dual: calcul de $u^Ta_i$}
	
	On peut aboutir à la même observation en invoquant les écarts de complémentarité: comme $x_j > 0$, la contrainte duale associée, $u^T a_j \leq 0$ est active, i.e. $u^T a_j = 0$.
	
\end{frame}

\begin{frame}
\frametitle{Algorithme primal-dual: exemple}

\begin{align*}
\min_x \ & 2x_1 + x_2 + 4x_3 \\
\mbox{soumis à } & x_1 + x_2 + 2x_3 = 3 \\
& 2x_1 + x_2 + 3x_3 = 5 \\
& x_1 \geq 0, x_2 \geq 0, x_3 \geq 0.
\end{align*}

\mbox{}

Comme tous les coefficients dans l'objectifs sont non-négatifs, le vecteur $\lambda = (0,0)$ est réalisable pour le dual.

\mbox{}

En effet, les contraintes du dual sont
\[
\lambda^T A \leq c.
\]
Avec $\lambda = (0,0)$, aucune contrainte du dual n'est active, et donc $P = \emptyset$.

\end{frame}

\begin{frame}
\frametitle{Algorithme primal-dual}

Le primal restreint est donc
\begin{align*}
\min\ & y_1 + y_2 \\
\mbox{t.q. } & x_1 + x_2 + 2x_3 + y_1 = 3 \\
& 2x_1 + x_2 + 3x_3 + y_2 = 5 \\
& x_1 \geq 0, x_2 \geq 0, x_3 \geq 0 \\
& y_1, y_2 \geq 0 \\
& x_1 = x_2 = x_3 = 0.
\end{align*}

Tableau du simplexe pour le primal restreint associé:
\[
\begin{matrix}
& a_1 & a_2 & a_3 & y_1 & y_2 & b \\
& 1 & 1 & 2 & 1 & 0 & 3 \\
& 2 & 1 & 3 & 0 & 1 & 5 \\
& -3 & -2 & -5 & 0 & 0 & -8 \\
c_i - \lambda^Ta_i \rightarrow & 2 & 1 & 4
\end{matrix}
\]

\end{frame}

\begin{frame}
\frametitle{Algorithme primal-dual: exemple}

Comme aucune contrainte duale n'est active (il n'y a pas de zéro dans la dernière ligne), $P = \emptyset$.

\mbox{}

Dès lors, $x_1$, $x_2$ et $x_3$ sont fixées à zéro. Il n'y a pas de coût réduit négatif pour les variables restantes, $y_1$ et $y_2$. La solution
\[
(0, 0, 0, 3, 5)
\]
est donc optimale pour le primal restreint associé.

\mbox{}

Le dual restreint associé s'écrit comme
\begin{align*}
\max_u\ & u^T b\\
\mbox{t.q. } & u \leq \bone,
\end{align*}
et $u_0 = (1,1)$ est solution optimale.

\end{frame}

\begin{frame}
\frametitle{Algorithme primal-dual: exemple}

Les quantités $-u_0^Ta_i$, $i = 1, 2, 3$, sont égales aux trois premiers éléments de la troisième ligne.

\mbox{}

Pour trouver $\epsilon$, nous prenons dès lors le minimum des rapport
\[
\frac{2}{3}, \frac{1}{2}, \frac{4}{5}.
\]
Le minimum étant $1/2$, $x_2$ entre dans la base, et on annule l'entrée correspondante sur la quatrième ligne, ce qui revient à rendre la seconde contrainte duale active pour le problème dual initial.

\end{frame}

\begin{frame}
\frametitle{Algorithme primal-dual: exemple}

Pour ce faire, on ajoute $\epsilon$ fois la troisième ligne à la dernière.
En effet, le nouveau vecteur dual est $\lambda+\epsilon u$ et nous avons comme nouvelles valeurs des contraintes duales
$$
c_i - \lambda^T a_i - \epsilon u^T a_i = c_i - (\lambda + \epsilon u)^Ta_i. 
$$

\mbox{}

Ceci donne
$$
\begin{matrix}
x_1 & x_2 & x_3 & y_1 & y_2 & b \\
1 & 1 & 2 & 1 & 0 & 3 \\
2 & 1 & 3 & 0 & 1 & 5 \\
-3 & -2 & -5 & 0 & 0 & -8 \\
\frac{1}{2} & 0 & \frac{3}{2}
\end{matrix}
$$

\end{frame}

\begin{frame}
\frametitle{Algorithme primal-dual: exemple}

On doit à présent minimiser le nouveau primal restreint, avec $P = \lbrace 2 \rbrace$.
\[
\begin{matrix}
a_1 & a_2 & a_3 & y_1 & y_2 & b \\
1 & \circled{1} & 2 & 1 & 0 & 3 \\
2 & 1 & 3 & 0 & 1 & 5 \\
-3 & -2 & -5 & 0 & 0 & -8 \\
\frac{1}{2} & 0 & \frac{3}{2}
\end{matrix}
\]

\mbox{}

\[
\begin{matrix}
a_1 & a_2 & a_3 & y_1 & y_2 & b \\
1 & 1 & 2 & 1 & 0 & 3 \\
1 & 0 & 1 & -1 & 1 & 2 \\
-1 & 0 & -1 & 2 & 0 & -2 \\
\frac{1}{2} & 0 & \frac{3}{2}
\end{matrix}
\]

Note: on retrouve $u^T a_2 = 0$.
\end{frame}

\begin{frame}
\frametitle{Algorithme primal-dual: exemple}

En calculant les rapports de la dernière ligne sur l'avant-dernière, on obtient $\epsilon = 1/2$, et comme colonne entrante $a_1$.

\mbox{}

On ajoute $\epsilon$ fois la troisième ligne à la dernière, pour obtenir
\[
\begin{matrix}
a_1 & a_2 & a_3 & y_1 & y_2 & b \\
1 & 1 & 2 & 1 & 0 & 3 \\
1 & 0 & 1 & -1 & 1 & 2 \\
-1 & 0 & -1 & 2 & 0 & -2 \\
0 & 0 & 1
\end{matrix}
\]

\mbox{}

\[
P = \lbrace 1, 2 \rbrace.
\]

\end{frame}

\begin{frame}
\frametitle{Algorithme primal-dual: exemple}

Résolution du primal restreint associé.
\[
\begin{matrix}
a_1 & a_2 & a_3 & y_1 & y_2 & b \\
1 & 1 & 2 & 1 & 0 & 3 \\
\circled{1} & 0 & 1 & -1 & 1 & 2 \\
-1 & 0 & -1 & 2 & 0 & -2 \\
0 & 0 & 1
\end{matrix}
\]

\mbox{}

\[
\begin{matrix}
a_1 & a_2 & a_3 & y_1 & y_2 & b \\
0 & 1 & 1 & 2 & -1 & 1 \\
1 & 0 & 1 & -1 & 1 & 2 \\
0 & 0 & 0 & 1 & 1 & 0 \\
0 & 0 & 1
\end{matrix}
\]

Le primal est réalisable: stop. La solution est
\[
x_1 = 2,\ x_2 = 1,\ x_3 = 0.
\]

\end{frame}

\begin{frame}
\frametitle{Algorithme primal-dual: preuve}

Dans l'étape 3, il est indiqué que $u_0^T a_j \leq 0$ pour tout $j$ implique que le primal \eqref{eq:P} n'a pas de solution réalisable.

\mbox{}

Si $u_0^T a_j \leq 0$ pour tout $j$, le vecteur
$\lambda_{\epsilon} = \lambda_0 + \epsilon u_0$ conduit à
%est réalisable pour le problème dual $\forall \epsilon > 0$, comme
\[
\lambda_{\epsilon}^TA = \lambda_0^TA + \epsilon u_0^TA \leq c^T.
\]

\mbox{}

De plus, comme
\[
u_0^T b = \bone^T y > 0,
\]
nous voyons que la quantité
\[
\lambda_{\epsilon}^Tb = \lambda_0^Tb + \epsilon u_0^T b,
\]
est non bornée, lorsque nous augmentons $\epsilon$.
Du théoreme de dualité forte, le primal n'est pas réalisable.

\end{frame}

\begin{frame}
\frametitle{Algorithme primal-dual: preuve}

Supposons à présent que pour au moins un $j$, $u_0^T a_j > 0$.

\mbox{}

À nouveau, définissons
\[
\lambda_{\epsilon} = \lambda_0 + \epsilon u_0
\]
Par construction,
\[
u_0^T a_i \leq 0, \ \forall i \in P.
\]
Pour un $\epsilon$ positif assez petit, $\lambda_{\epsilon}$ est réalisable pour le dual, et nous pouvons augmenter $\epsilon$ jusqu'à transformer une des inégalités
\[
\lambda_{\epsilon}^T a_j < c_j,\ j \notin P
\]
en égalité.
Ceci détermine $\epsilon_0$ et un indice $k$ correspondant.

\end{frame}

\begin{frame}
\frametitle{Algorithme primal-dual: preuve}

Le nouveau vecteur $\lambda$ correspond à une valeur accrue de la fonction objectif duale:
\[
\lambda^T b = \lambda_0^T b + \epsilon u_0^T b.
\]
De plus, le nouvel ensemble correspondant $P$ inclut l'indice $k$.

\mbox{}

Pour tout autre indice $i$ t.q. $x_i > 0$ est dans $P$ aussi, comme en vertu de l'écart de complémentarité,
\[
u_0^T a_i = 0,
\]
pour un tel $i$, nous avons
\[
\lambda^T a_i = \lambda_0^T a_i + \epsilon_0 u_0^T a_i = c_i.
\]

\end{frame}

\begin{frame}
\frametitle{Algorithme primal-dual: preuve}

Ceci signifie que l'ancienne solution optimale (du primal restreint) est réalisable pour le nouveau problème primal restreint associé, et que $a_k$ peut entrer dans la base.
Puisque $u_0^T a_k > 0$, pivoter sur $a_k$ va décroître la valeur du primal restreint associé.

\mbox{}

Donc,
\begin{itemize}
\item
soit la valeur du primal décroît (strictement sous l'hypothèse de non-dégénérescence),
\item
soit le problème est déclaré non réalisble.
\end{itemize}

\mbox{}

Sous l'hypothèse de non-dégénérescence, l'algorithme se termine en un nombre fini d'étapes comme il y a un nombre fini de bases réalisables.

\end{frame}

\end{document}
