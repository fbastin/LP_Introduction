\documentclass[t,usepdftitle=false]{beamer}

\usepackage[utf8]{inputenc}

\usetheme{Singapore}
\usepackage{xcolor}

\setbeamertemplate{footline}[frame number]

% \setbeamercovered{transparent}
%\usecolortheme{crane}
\title[Simplexe primal-dual]{Programmation Linéaire\\Algorithme du simplexe primal-dual}
\author[Fabian Bastin]{Fabian Bastin\\DIRO\\Université de Montréal}
\date{}

\usepackage{ulem}
\usepackage{tikz}
\newcommand*\circled[1]{\tikz[baseline=(char.base)]{
    \node[shape=circle,draw,inner sep=2pt] (char) {#1};}}

\def\ba{\boldsymbol{a}}
\def\bb{\boldsymbol{b}}
\def\bc{\boldsymbol{c}}
\def\be{\boldsymbol{e}}
\def\br{\boldsymbol{r}}
\def\bu{\boldsymbol{u}}
\def\bx{\boldsymbol{x}}
\def\by{\boldsymbol{y}}
\def\bz{\boldsymbol{z}}
\def\bA{\boldsymbol{A}}
\def\bB{\boldsymbol{B}}
\def\bD{\boldsymbol{D}}
\def\bH{\boldsymbol{H}}
\def\bI{\boldsymbol{I}}
\def\bL{\boldsymbol{L}}
\def\bM{\boldsymbol{M}}
\def\bU{\boldsymbol{U}}
\def\bzero{\boldsymbol{0}}
\def\bone{\boldsymbol{1}}
\def\blambda{\boldsymbol{\lambda}}

\def\cR{\mathcal{R}}

\usepackage[frenchb]{babel}

\begin{document}
\frame{\titlepage}

% ------------------------------------------------------------------------------------------------------------------------------------------------------\begin{frame}
\begin{frame}
\frametitle{Simplexe primal-dual}

Motivations:
\begin{itemize}
\item
Exploiter d'avantage la complémentarité entre le primal et le dual.
\item
Comme pour le simplexe dual, on part d'une solution dual-réalisable.
\item
Primal restreint: on va forcer la condition de complémentarité
\[
x_i > 0 \Rightarrow \lambda^T a_i = c_i,
\]
en faisant entrer dans la base primale les $x_i$ correspondant aux contraintes duales actives.
\item
Dual restreint: on optimise le dual. Si celui-ci est réalisable, augmenter (strictement) la valeur de l'objectif dual va conduire à transformer au moins une contrainte duale inactive en contrainte duale active.
\end{itemize}

\end{frame}

\begin{frame}
\frametitle{Simplexe primal-dual}

L'idée est de travailler simultanément sur le primal et le dual.

\mbox{}

Principales idées:
\begin{itemize}
\item
trouver une solution réalisable pour le dual;
\item
l'améliorer à chaque étape en optimisant un problème primal restreint associé;
\item
essayer de satisfaire les conditions d'écart de complémentarité.
\end{itemize}

\mbox{}

Il s'agit de la variante du simplexe la plus efficace pour les problèmes de flots dans les réseaux.

\end{frame}

\begin{frame}
\frametitle{Simplexe primal-dual}

Considérons à nouveau le primal
\begin{align*}
\min_x \ & c^Tx \\
\mbox{t.q. } & Ax = b\\
& x \geq 0.
\end{align*}
et son dual
\begin{align*}
\max_{\lambda} \ & \lambda^T b \\
\mbox{t.q. } & \lambda^T A \leq c^T
\end{align*}

\mbox{}

Étant donné $\lambda$ réalisable pour le dual, définissons l'ensemble actif
\[
P = \lbrace i \,|\, \lambda^T a_i = c_i \rbrace.
\]
Vu que $\lambda$ est supposé réalisable, cela implique
\[
\lambda^T a_i < c_i,\ i \notin P.
\]

\end{frame}

\begin{frame}
\frametitle{Simplexe primal-dual}

Correspondant à $\lambda$ et $P$, nous définissons le problème {\sl primal restreint associé}
\begin{align*}
\min_{x,\ y} \ & \bone^Ty \\
\mbox{t.q. } & Ax + y = b\\
& x \geq 0,\ x_i = 0 \mbox{ pour } i \notin P\\
& y \geq 0
\end{align*}
où $\bone$ designe the vecteur $(1, 1, \ldots, 1)$.
Nous pouvons réécrire le problème comme
\begin{align*}
\min_{y \geq 0,\ x_i \in P} \ & \bone^Ty \\
\mbox{t.q. } & \sum_{i \in P} a_ix_i + y = b\\
& x_i \geq 0,\  i \in P
\end{align*}

\end{frame}

\begin{frame}
\frametitle{Simplexe primal-dual}

Le dual associé est appelé {\sl dual restreint associé}
\begin{align*}
\max_u\ & u^T b\\
\mbox{t.q. } & u^T a_i \leq 0,\ i \in P\\
& u \leq \bone.
\end{align*}

\end{frame}

\begin{frame}
\frametitle{Théorème d'optimalité primale-duale}

{\sl Supposons que $\lambda$ est réalisable pour le dual et que $(x,y)$ est réalisable pour le primal restreint associé, avec $y = 0$ (de sorte que $(x,y)$ est une solution optimale). Alors, $x$ et $\lambda$ sont optimaux pour les programmes primal et dual originaux respectifs.}

\begin{proof}
$x$ est clairement réalisable pour le primal: $Ax = b$.
Nous avons aussi, par définition de $P$,
$\lambda^T a_i = c_i,\mbox{ si } x_i \ne 0$,
de sorte que
\[
c^T x = \lambda^T Ax.
\]
En combinant ces deux observations, nous avons
\[
c^T x = \lambda^T b,
\]
impliquant l'optimalité de $x$ et $\lambda$.
\end{proof}

\end{frame}

\begin{frame}
\frametitle{Algorithme primal-dual}

{\bf Etape 1}
Étant donnée une solution réalisable $\lambda_0$ pour le dual, déterminer le primal restreint associé.

\mbox{}

{\bf Etape 2}
Optimiser le primal restreint associé.
Si la valeur optimale de ce primal restreint associé est nulle (impliquant $y = 0$), la solution correspondante est optimale pour le primal original, en vertu du théorème d'optimalité primale-duale; arrêt.


\mbox{}

{\bf Etape 3}
Si la valeur optimale du primal restreint associé est strictement positive
(i.e. if $y \ne 0$), la solution optimale de ce primal restreint associé n'est pas réalisable pour le primal, et on cherche à améliorer la solution réalisable du dual avant de déterminer un nouveau primal restreint associé.

\end{frame}

\begin{frame}
\frametitle{Algorithme primal-dual}

{\bf Etape 3 (suite)}
Obtenir du tableau du simplexe du primal restreint la solution $u_0$ du dual restreint associé.
S'il n'y a pas de $j$ pour lequel $u_0^T a_j > 0$, le primal n'a pas de solution réalisable; arrêt.
Sinon, construire le nouveau vecteur dual réalisable
\[
\lambda = \lambda_0 + \epsilon_0 u_0,
\]
où
\[
\epsilon_0 = \min_j \left\lbrace \frac{c_j - \lambda_0^Ta_j}{u_0^T a_j} \,\bigg|\, u_0^Ta_j > 0\right\rbrace.
\]
Retour à l'étape 1, en utilisant ce $\lambda$.

\end{frame}

\begin{frame}
\frametitle{Algorithme primal-dual: exemple}

\begin{align*}
\min_x \ & 2x_1 + x_2 + 4x_3 \\
\mbox{soumis à } & x_1 + x_2 + 2x_3 = 3 \\
& 2x_1 + x_2 + 3x_3 = 5 \\
& x_1 \geq 0, x_2 \geq 0, x_3 \geq 0.
\end{align*}

\mbox{}

Comme tous les coefficients dans l'objectifs sont non-négatifs, le vecteur $\lambda = (0,0)$ est réalisable pour le dual.

\mbox{}

En effet, les contraintes du dual sont
\[
\lambda^T A \leq c.
\]
Avec $\lambda = (0,0)$, aucune contrainte du dual n'est active, et donc $P = \emptyset$.

\end{frame}

\begin{frame}
\frametitle{Algorithme primal-dual}

Le primal restreint est donc
\begin{align*}
\min\ & y_1 + y_2 \\
\mbox{t.q. } & x_1 + x_2 + 2x_3 + y_1 = 3 \\
& 2x_1 + x_2 + 3x_3 + y_2 = 5 \\
& x_1 \geq 0, x_2 \geq 0, x_3 \geq 0 \\
& y_1, y_2 \geq 0 \\
& x_1 = x_2 = x_3 = 0.
\end{align*}

Tableau du simplexe pour le primal restreint associé:
\[
\begin{matrix}
& a_1 & a_2 & a_3 & y_1 & y_2 & b \\
& 1 & 1 & 2 & 1 & 0 & 3 \\
& 2 & 1 & 3 & 0 & 1 & 5 \\
& -3 & -2 & -5 & 0 & 0 & -8 \\
c_i - \lambda^Ta_i \rightarrow & 2 & 1 & 4
\end{matrix}
\]

\end{frame}

\begin{frame}
\frametitle{Algorithme primal-dual: calcul de $u^Ta_i$}

Pour $i \notin P$, $x_i$ est fixé à 0 et est hors base.
Nous pouvons calculer son coût réduit à l'optimalité du primal restreint comme
\[
0 - u^Ta_i.
\]
En effet, l'objectif du primal restreint étant $\bone^Ty$, les coefficients dans l'objectif associés aux $x_i$ sont tous nuls.
Dès lors, les valeurs de $u^Ta_i$, $i \notin P$, peuvent être directement identifiées en prenant l'opposé des valeurs dans la ligne des coûts réduits du tableau final du primal restreint, pour les colonnes associées aux $x_i$ correspondants.

\end{frame}

\begin{frame}
\frametitle{Algorithme primal-dual: calcul de $u^Ta_i$}

Pour les variables de base $x_j$ ($> 0$), nous avons $j \in P$, et les coûts réduits sont nuls.

\mbox{}

À nouveau, le coefficient de $x_j$ dans l'objectif restreint étant nul, et son coût réduit se calcule comme
\[
0 - u^Ta_j.
\]
On en tire $u^T a_j = 0$.

\mbox{}

On peut aboutir à la même observation en invoquant les écarts de complémentarité: comme $x_j > 0$, la contrainte duale associée, $u^T a_j \leq 0$ est active, i.e. $u^T a_j = 0$.

\end{frame}

\begin{frame}
\frametitle{Algorithme primal-dual: exemple}

Comme aucune contrainte duale n'est active (il n'y a pas de zéro dans la dernière ligne), $P = \emptyset$.

\mbox{}

Dès lors, $x_1$, $x_2$ et $x_3$ sont fixées à zéro. Il n'y a pas de coût réduit négatif pour les variables restantes, $y_1$ et $y_2$. La solution
\[
(0, 0, 0, 3, 5)
\]
est donc optimale pour le primal restreint associé.

\mbox{}

Le dual restreint associé s'écrit comme
\begin{align*}
\max_u\ & u^T b\\
\mbox{t.q. } & u \leq \bone,
\end{align*}
et $u_0 = (1,1)$ est solution optimale.

\end{frame}

\begin{frame}
\frametitle{Algorithme primal-dual: exemple}

Les quantités $-u_0^Ta_i$, $i = 1, 2, 3$, sont égales aux trois premiers éléments de la troisième ligne.

\mbox{}

Pour trouver $\epsilon$, nous prenons dès lors le minimum des rapport
\[
\frac{2}{3}, \frac{1}{2}, \frac{4}{5}.
\]
Le minimum étant $1/2$, $x_2$ entre dans la base, et on annule l'entrée correspondante sur la quatrième ligne, ce qui revient à rendre la seconde contrainte duale active pour le problème dual initial.

\end{frame}

\begin{frame}
\frametitle{Algorithme primal-dual: exemple}

Pour ce faire, on ajoute $\epsilon$ fois la troisième ligne à la dernière.
En effet, le nouveau vecteur dual est $\lambda+\epsilon u$ et nous avons comme nouvelles valeurs des contraintes duales
$$
c_i - \lambda^T a_i - \epsilon u^T a_i = c_i - (\lambda + \epsilon u)^Ta_i. 
$$

\mbox{}

Ceci donne
$$
\begin{matrix}
x_1 & x_2 & x_3 & y_1 & y_2 & b \\
1 & 1 & 2 & 1 & 0 & 3 \\
2 & 1 & 3 & 0 & 1 & 5 \\
-3 & -2 & -5 & 0 & 0 & -8 \\
\frac{1}{2} & 0 & \frac{3}{2}
\end{matrix}
$$

\end{frame}

\begin{frame}
\frametitle{Algorithme primal-dual: exemple}

On doit à présent minimiser le nouveau primal restreint, avec $P = \lbrace 2 \rbrace$.
\[
\begin{matrix}
a_1 & a_2 & a_3 & y_1 & y_2 & b \\
1 & \circled{1} & 2 & 1 & 0 & 3 \\
2 & 1 & 3 & 0 & 1 & 5 \\
-3 & -2 & -5 & 0 & 0 & -8 \\
\frac{1}{2} & 0 & \frac{3}{2}
\end{matrix}
\]

\mbox{}

\[
\begin{matrix}
a_1 & a_2 & a_3 & y_1 & y_2 & b \\
1 & 1 & 2 & 1 & 0 & 3 \\
1 & 0 & 1 & -1 & 1 & 2 \\
-1 & 0 & -1 & 2 & 0 & -2 \\
\frac{1}{2} & 0 & \frac{3}{2}
\end{matrix}
\]

Note: on retrouve $u^T a_2 = 0$.
\end{frame}

\begin{frame}
\frametitle{Algorithme primal-dual: exemple}

En calculant les rapports de la dernière ligne sur l'avant-dernière, on obtient $\epsilon = 1/2$, et comme colonne entrante $a_1$.

\mbox{}

On ajoute $\epsilon$ fois la troisième ligne à la dernière, pour obtenir
\[
\begin{matrix}
a_1 & a_2 & a_3 & y_1 & y_2 & b \\
1 & 1 & 2 & 1 & 0 & 3 \\
1 & 0 & 1 & -1 & 1 & 2 \\
-1 & 0 & -1 & 2 & 0 & -2 \\
0 & 0 & 1
\end{matrix}
\]

\mbox{}

\[
P = \lbrace 1, 2 \rbrace.
\]

\end{frame}

\begin{frame}
\frametitle{Algorithme primal-dual: exemple}

Résolution du primal restreint associé.
\[
\begin{matrix}
a_1 & a_2 & a_3 & y_1 & y_2 & b \\
1 & 1 & 2 & 1 & 0 & 3 \\
\circled{1} & 0 & 1 & -1 & 1 & 2 \\
-1 & 0 & -1 & 2 & 0 & -2 \\
0 & 0 & 1
\end{matrix}
\]

\mbox{}

\[
\begin{matrix}
a_1 & a_2 & a_3 & y_1 & y_2 & b \\
0 & 1 & 1 & 2 & -1 & 1 \\
1 & 0 & 1 & -1 & 1 & 2 \\
0 & 0 & 0 & 1 & 1 & 0 \\
0 & 0 & 1
\end{matrix}
\]

Le primal est réalisable: stop. La solution est
\[
x_1 = 2,\ x_2 = 1,\ x_3 = 0.
\]

\end{frame}

\begin{frame}
\frametitle{Algorithme primal-dual: preuve}

Dans l'étape 3, il est indiqué que $u_0^T a_j \leq 0$ pour tout $j$ implique que le primal n'a pas de solution réalisable.

\mbox{}

Si $u_0^T a_j \leq 0$ pour tout $j$, le vecteur
$\lambda_{\epsilon} = \lambda_0 + \epsilon u_0$ conduit à
%est réalisable pour le problème dual $\forall \epsilon > 0$, comme
\[
\lambda_{\epsilon}^TA = \lambda_0^TA + \epsilon u_0^TA \leq c^T.
\]

\mbox{}

De plus, comme
\[
u_0^T b = \bone^T y > 0,
\]
nous voyons que la quantité
\[
\lambda_{\epsilon}^Tb = \lambda_0^Tb + \epsilon u_0^T b,
\]
est non bornée, lorsque nous augmentons $\epsilon$.
Du théoreme de dualité forte, le primal n'est pas réalisable.

\end{frame}

\begin{frame}
\frametitle{Algorithme primal-dual: preuve}

Supposons à présent que pour au moins un $j$, $u_0^T a_j > 0$.

\mbox{}

À nouveau, définissons
\[
\lambda_{\epsilon} = \lambda_0 + \epsilon u_0
\]
Par construction,
\[
u_0^T a_i \leq 0, \ \forall i \in P.
\]
Pour un $\epsilon$ positif assez petit, $\lambda_{\epsilon}$ est réalisable pour le dual, et nous pouvons augmenter $\epsilon$ jusqu'à transformer une des inégalités
\[
\lambda_{\epsilon}^T a_j < c_j,\ j \notin P
\]
en égalité.
Ceci détermine $\epsilon_0$ et un indice $k$ correspondant.

\end{frame}

\begin{frame}
\frametitle{Algorithme primal-dual: preuve}

Le nouveau vecteur $\lambda$ correspond à une valeur accrue de la fonction objectif duale:
\[
\lambda^T b = \lambda_0^T b + \epsilon u_0^T b.
\]
De plus, le nouvel ensemble correspondant $P$ inclut l'indice $k$.

\mbox{}

Pour tout autre indice $i$ t.q. $x_i > 0$ est dans $P$ aussi, comme en vertu de l'écart de complémentarité,
\[
u_0^T a_i = 0,
\]
pour un tel $i$, nous avons
\[
\lambda^T a_i = \lambda_0^T a_i + \epsilon_0 u_0^T a_i = c_i.
\]

\end{frame}

\begin{frame}
\frametitle{Algorithme primal-dual: preuve}

Ceci signifie que l'ancienne solution optimale (du primal restreint) est réalisable pour le nouveau problème primal restreint associé, et que $a_k$ peut entrer dans la base.
Puisque $u_0^T a_k > 0$, pivoter sur $a_k$ va décroître la valeur du primal restreint associé.

\mbox{}

Donc,
\begin{itemize}
\item
soit la valeur du primal décroît (strictement sous l'hypothèse de non-dégénérescence),
\item
soit le problème est déclaré non réalisble.
\end{itemize}

\mbox{}

Sous l'hypothèse de non-dégénérescence, l'algorithme se termine en un nombre fini d'étapes comme il y a un nombre fini de bases réalisables.

\end{frame}

\end{document}
