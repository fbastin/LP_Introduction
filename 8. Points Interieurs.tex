\documentclass[usepdftitle=false, aspectratio=169]{beamer}
\usepackage[utf8]{inputenc}

%\usetheme{Warsaw}
%\usepackage{xcolor}

% \setbeamercovered{transparent}
%\usecolortheme{crane}
\title[Points intérieurs]{Méthodes de points intérieurs}
\author[Fabian Bastin]{Fabian Bastin\\DIRO\\Université de Montréal}
\date{}

\usetheme{Singapore}
\usepackage{xcolor}

\setbeamertemplate{footline}[frame number]

\usepackage{ulem}
\usepackage{cancel}

\usepackage{tikz}
\newcommand*\circled[1]{\tikz[baseline=(char.base)]{
    \node[shape=circle,draw,inner sep=2pt] (char) {#1};}}

\usepackage[french]{babel}

% We add the following commonly used macros:

% Note: \bsc is already in babel-french; the macros below override it

% vectors as boldsymbols:
\newcommand{\bsa}{{\boldsymbol{a}}}
\newcommand{\bsb}{{\boldsymbol{b}}}
\providecommand{\bsc}{} % if \bsc is not defined, define it
\renewcommand{\bsc}{{\boldsymbol{c}}}
\newcommand{\bsd}{{\boldsymbol{d}}}
\newcommand{\bse}{{\boldsymbol{e}}}
\newcommand{\bsf}{{\boldsymbol{f}}}
\newcommand{\bsg}{{\boldsymbol{g}}}
\newcommand{\bsh}{{\boldsymbol{h}}}
\newcommand{\bsi}{{\boldsymbol{i}}}
\newcommand{\bsj}{{\boldsymbol{j}}}
\newcommand{\bsk}{{\boldsymbol{k}}}
\newcommand{\bsl}{{\boldsymbol{l}}}
\newcommand{\bsm}{{\boldsymbol{m}}}
\newcommand{\bsn}{{\boldsymbol{n}}}
\newcommand{\bso}{{\boldsymbol{o}}}
\newcommand{\bsp}{{\boldsymbol{p}}}
\newcommand{\bsq}{{\boldsymbol{q}}}
\newcommand{\bsr}{{\boldsymbol{r}}}
\newcommand{\bss}{{\boldsymbol{s}}}
\newcommand{\bst}{{\boldsymbol{t}}}
\newcommand{\bsu}{{\boldsymbol{u}}}
\newcommand{\bsv}{{\boldsymbol{v}}}
\newcommand{\bsw}{{\boldsymbol{w}}}
\newcommand{\bsx}{{\boldsymbol{x}}}
\newcommand{\bsy}{{\boldsymbol{y}}}
\newcommand{\bsz}{{\boldsymbol{z}}}
\newcommand{\bsA}{{\boldsymbol{A}}}
\newcommand{\bsB}{{\boldsymbol{B}}}
\newcommand{\bsC}{{\boldsymbol{C}}}
\newcommand{\bsD}{{\boldsymbol{D}}}
\newcommand{\bsE}{{\boldsymbol{E}}}
\newcommand{\bsF}{{\boldsymbol{F}}}
\newcommand{\bsG}{{\boldsymbol{G}}}
\newcommand{\bsH}{{\boldsymbol{H}}}
\newcommand{\bsI}{{\boldsymbol{I}}}
\newcommand{\bsJ}{{\boldsymbol{J}}}
\newcommand{\bsK}{{\boldsymbol{K}}}
\newcommand{\bsL}{{\boldsymbol{L}}}
\newcommand{\bsM}{{\boldsymbol{M}}}
\newcommand{\bsN}{{\boldsymbol{N}}}
\newcommand{\bsO}{{\boldsymbol{O}}}
\newcommand{\bsP}{{\boldsymbol{P}}}
\newcommand{\bsQ}{{\boldsymbol{Q}}}
\newcommand{\bsR}{{\boldsymbol{R}}}
\newcommand{\bsS}{{\boldsymbol{S}}}
\newcommand{\bsT}{{\boldsymbol{T}}}
\newcommand{\bsU}{{\boldsymbol{U}}}
\newcommand{\bsV}{{\boldsymbol{V}}}
\newcommand{\bsW}{{\boldsymbol{W}}}
\newcommand{\bsX}{{\boldsymbol{X}}}
\newcommand{\bsY}{{\boldsymbol{Y}}}
\newcommand{\bsZ}{{\boldsymbol{Z}}}
% other commonly used boldsymbols:
\newcommand{\bsell}{{\boldsymbol{\ell}}}
\newcommand{\bszero}{{\boldsymbol{0}}} % vector of zeros
\newcommand{\bsone}{{\boldsymbol{1}}}  % vector of ones
% boldsymbol greeks:
\newcommand{\bsalpha}{{\boldsymbol{\alpha}}}
\newcommand{\bsbeta}{{\boldsymbol{\beta}}}
\newcommand{\bsgamma}{{\boldsymbol{\gamma}}}
\newcommand{\bsdelta}{{\boldsymbol{\delta}}}
\newcommand{\bsepsilon}{{\boldsymbol{\epsilon}}}
\newcommand{\bsvarepsilon}{{\boldsymbol{\varepsilon}}}
\newcommand{\bszeta}{{\boldsymbol{\zeta}}}
\newcommand{\bseta}{{\boldsymbol{\eta}}}
\newcommand{\bstheta}{{\boldsymbol{\theta}}}
\newcommand{\bsvartheta}{{\boldsymbol{\vartheta}}}
\newcommand{\bskappa}{{\boldsymbol{\kappa}}}
\newcommand{\bslambda}{{\boldsymbol{\lambda}}}
\newcommand{\bsmu}{{\boldsymbol{\mu}}}
\newcommand{\bsnu}{{\boldsymbol{\nu}}}
\newcommand{\bsxi}{{\boldsymbol{\xi}}}
\newcommand{\bspi}{{\boldsymbol{\pi}}}
\newcommand{\bsvarpi}{{\boldsymbol{\varpi}}}
\newcommand{\bsrho}{{\boldsymbol{\rho}}}
\newcommand{\bsvarrho}{{\boldsymbol{\varrho}}}
\newcommand{\bssigma}{{\boldsymbol{\sigma}}}
\newcommand{\bsvarsigma}{{\boldsymbol{\varsigma}}}
\newcommand{\bstau}{{\boldsymbol{\tau}}}
\newcommand{\bsupsilon}{{\boldsymbol{\upsilon}}}
\newcommand{\bsphi}{{\boldsymbol{\phi}}}
\newcommand{\bsvarphi}{{\boldsymbol{\varphi}}}
\newcommand{\bschi}{{\boldsymbol{\chi}}}
\newcommand{\bspsi}{{\boldsymbol{\psi}}}
\newcommand{\bsomega}{{\boldsymbol{\omega}}}
\newcommand{\bsGamma}{{\boldsymbol{\Gamma}}}
\newcommand{\bsDelta}{{\boldsymbol{\Delta}}}
\newcommand{\bsTheta}{{\boldsymbol{\Theta}}}
\newcommand{\bsLambda}{{\boldsymbol{\Lambda}}}
\newcommand{\bsXi}{{\boldsymbol{\Xi}}}
\newcommand{\bsPi}{{\boldsymbol{\Pi}}}
\newcommand{\bsSigma}{{\boldsymbol{\Sigma}}}
\newcommand{\bsUpsilon}{{\boldsymbol{\Upsilon}}}
\newcommand{\bsPhi}{{\boldsymbol{\Phi}}}
\newcommand{\bsPsi}{{\boldsymbol{\Psi}}}
\newcommand{\bsOmega}{{\boldsymbol{\Omega}}}

% Roman fonts:
\newcommand{\rma}{{\mathrm{a}}}
\newcommand{\rmb}{{\mathrm{b}}}
\newcommand{\rmc}{{\mathrm{c}}}
\newcommand{\rmd}{{\mathrm{d}}}
\newcommand{\rme}{{\mathrm{e}}}
\newcommand{\rmf}{{\mathrm{f}}}
\newcommand{\rmg}{{\mathrm{g}}}
\newcommand{\rmh}{{\mathrm{h}}}
\newcommand{\rmi}{{\mathrm{i}}}
\newcommand{\rmj}{{\mathrm{j}}}
\newcommand{\rmk}{{\mathrm{k}}}
\newcommand{\rml}{{\mathrm{l}}}
\newcommand{\rmm}{{\mathrm{m}}}
\newcommand{\rmn}{{\mathrm{n}}}
\newcommand{\rmo}{{\mathrm{o}}}
\newcommand{\rmp}{{\mathrm{p}}}
\newcommand{\rmq}{{\mathrm{q}}}
\newcommand{\rmr}{{\mathrm{r}}}
\newcommand{\rms}{{\mathrm{s}}}
\newcommand{\rmt}{{\mathrm{t}}}
\newcommand{\rmu}{{\mathrm{u}}}
\newcommand{\rmv}{{\mathrm{v}}}
\newcommand{\rmw}{{\mathrm{w}}}
\newcommand{\rmx}{{\mathrm{x}}}
\newcommand{\rmy}{{\mathrm{y}}}
\newcommand{\rmz}{{\mathrm{z}}}
\newcommand{\rmA}{{\mathrm{A}}}
\newcommand{\rmB}{{\mathrm{B}}}
\newcommand{\rmC}{{\mathrm{C}}}
\newcommand{\rmD}{{\mathrm{D}}}
\newcommand{\rmE}{{\mathrm{E}}}
\newcommand{\rmF}{{\mathrm{F}}}
\newcommand{\rmG}{{\mathrm{G}}}
\newcommand{\rmH}{{\mathrm{H}}}
\newcommand{\rmI}{{\mathrm{I}}}
\newcommand{\rmJ}{{\mathrm{J}}}
\newcommand{\rmK}{{\mathrm{K}}}
\newcommand{\rmL}{{\mathrm{L}}}
\newcommand{\rmM}{{\mathrm{M}}}
\newcommand{\rmN}{{\mathrm{N}}}
\newcommand{\rmO}{{\mathrm{O}}}
\newcommand{\rmP}{{\mathrm{P}}}
\newcommand{\rmQ}{{\mathrm{Q}}}
\newcommand{\rmR}{{\mathrm{R}}}
\newcommand{\rmS}{{\mathrm{S}}}
\newcommand{\rmT}{{\mathrm{T}}}
\newcommand{\rmU}{{\mathrm{U}}}
\newcommand{\rmV}{{\mathrm{V}}}
\newcommand{\rmW}{{\mathrm{W}}}
\newcommand{\rmX}{{\mathrm{X}}}
\newcommand{\rmY}{{\mathrm{Y}}}
\newcommand{\rmZ}{{\mathrm{Z}}}
% also commonly defined
\newcommand{\rd}{{\mathrm{d}}}
\newcommand{\ri}{{\mathrm{i}}}

% blackboards:
\newcommand{\bbA}{{\mathbb{A}}}
\newcommand{\bbB}{{\mathbb{B}}}
\newcommand{\bbC}{{\mathbb{C}}}
\newcommand{\bbD}{{\mathbb{D}}}
\newcommand{\bbE}{{\mathbb{E}}}
\newcommand{\bbF}{{\mathbb{F}}}
\newcommand{\bbG}{{\mathbb{G}}}
\newcommand{\bbH}{{\mathbb{H}}}
\newcommand{\bbI}{{\mathbb{I}}}
\newcommand{\bbJ}{{\mathbb{J}}}
\newcommand{\bbK}{{\mathbb{K}}}
\newcommand{\bbL}{{\mathbb{L}}}
\newcommand{\bbM}{{\mathbb{M}}}
\newcommand{\bbN}{{\mathbb{N}}}
\newcommand{\bbO}{{\mathbb{O}}}
\newcommand{\bbP}{{\mathbb{P}}}
\newcommand{\bbQ}{{\mathbb{Q}}}
\newcommand{\bbR}{{\mathbb{R}}}
\newcommand{\bbS}{{\mathbb{S}}}
\newcommand{\bbT}{{\mathbb{T}}}
\newcommand{\bbU}{{\mathbb{U}}}
\newcommand{\bbV}{{\mathbb{V}}}
\newcommand{\bbW}{{\mathbb{W}}}
\newcommand{\bbX}{{\mathbb{X}}}
\newcommand{\bbY}{{\mathbb{Y}}}
\newcommand{\bbZ}{{\mathbb{Z}}}
% commonly used shortcuts:
\newcommand{\C}{{\mathbb{C}}} % complex numbers
\newcommand{\F}{{\mathbb{F}}} % field, finite field
\newcommand{\N}{{\mathbb{N}}} % natural numbers {1, 2, ...}
\newcommand{\Q}{{\mathbb{Q}}} % rationals
\newcommand{\R}{{\mathbb{R}}} % reals
\newcommand{\Z}{{\mathbb{Z}}} % integers
% more commonly used shortcuts:
\newcommand{\CC}{{\mathbb{C}}} % complex numbers
\newcommand{\FF}{{\mathbb{F}}} % field, finite field
\newcommand{\NN}{{\mathbb{N}}} % natural numbers {1, 2, ...}
\newcommand{\QQ}{{\mathbb{Q}}} % rationals
\newcommand{\RR}{{\mathbb{R}}} % reals
\newcommand{\ZZ}{{\mathbb{Z}}} % integers
% more commonly used shortcuts:
\newcommand{\EE}{{\mathbb{E}}}
\newcommand{\PP}{{\mathbb{P}}}
\newcommand{\TT}{{\mathbb{T}}}
\newcommand{\VV}{{\mathbb{V}}}
% and even more commonly used shortcuts:
\newcommand{\Complex}{{\mathbb{C}}}
\newcommand{\Integer}{{\mathbb{Z}}}
\newcommand{\Natural}{{\mathbb{N}}}
\newcommand{\Rational}{{\mathbb{Q}}}
\newcommand{\Real}{{\mathbb{R}}}

%%% Issue with miktex
% indicator boldface 1:
%\DeclareSymbolFont{bbold}{U}{bbold}{m}{n}
%\DeclareSymbolFontAlphabet{\mathbbold}{bbold}
%\newcommand{\ind}{{\mathbbold{1}}}
%\newcommand{\bbone}{{\mathbbold{1}}}


% calligraphic letters:
\newcommand{\cala}{{\mathcal{a}}}
\newcommand{\calb}{{\mathcal{b}}}
\newcommand{\calc}{{\mathcal{c}}}
\newcommand{\cald}{{\mathcal{d}}}
\newcommand{\cale}{{\mathcal{e}}}
\newcommand{\calf}{{\mathcal{f}}}
\newcommand{\calg}{{\mathcal{g}}}
\newcommand{\calh}{{\mathcal{h}}}
\newcommand{\cali}{{\mathcal{i}}}
\newcommand{\calj}{{\mathcal{j}}}
\newcommand{\calk}{{\mathcal{k}}}
\newcommand{\call}{{\mathcal{l}}}
\newcommand{\calm}{{\mathcal{m}}}
\newcommand{\caln}{{\mathcal{n}}}
\newcommand{\calo}{{\mathcal{o}}}
\newcommand{\calp}{{\mathcal{p}}}
\newcommand{\calq}{{\mathcal{q}}}
\newcommand{\calr}{{\mathcal{r}}}
\newcommand{\cals}{{\mathcal{s}}}
\newcommand{\calt}{{\mathcal{t}}}
\newcommand{\calu}{{\mathcal{u}}}
\newcommand{\calv}{{\mathcal{v}}}
\newcommand{\calw}{{\mathcal{w}}}
\newcommand{\calx}{{\mathcal{x}}}
\newcommand{\caly}{{\mathcal{y}}}
\newcommand{\calz}{{\mathcal{z}}}
\newcommand{\calA}{{\mathcal{A}}}
\newcommand{\calB}{{\mathcal{B}}}
\newcommand{\calC}{{\mathcal{C}}}
\newcommand{\calD}{{\mathcal{D}}}
\newcommand{\calE}{{\mathcal{E}}}
\newcommand{\calF}{{\mathcal{F}}}
\newcommand{\calG}{{\mathcal{G}}}
\newcommand{\calH}{{\mathcal{H}}}
\newcommand{\calI}{{\mathcal{I}}}
\newcommand{\calJ}{{\mathcal{J}}}
\newcommand{\calK}{{\mathcal{K}}}
\newcommand{\calL}{{\mathcal{L}}}
\newcommand{\calM}{{\mathcal{M}}}
\newcommand{\calN}{{\mathcal{N}}}
\newcommand{\calO}{{\mathcal{O}}}
\newcommand{\calP}{{\mathcal{P}}}
\newcommand{\calQ}{{\mathcal{Q}}}
\newcommand{\calR}{{\mathcal{R}}}
\newcommand{\calS}{{\mathcal{S}}}
\newcommand{\calT}{{\mathcal{T}}}
\newcommand{\calU}{{\mathcal{U}}}
\newcommand{\calV}{{\mathcal{V}}}
\newcommand{\calW}{{\mathcal{W}}}
\newcommand{\calX}{{\mathcal{X}}}
\newcommand{\calY}{{\mathcal{Y}}}
\newcommand{\calZ}{{\mathcal{Z}}}

% Euler fraks:
\newcommand{\fraka}{{\mathfrak{a}}}
\newcommand{\frakb}{{\mathfrak{b}}}
\newcommand{\frakc}{{\mathfrak{c}}}
\newcommand{\frakd}{{\mathfrak{d}}}
\newcommand{\frake}{{\mathfrak{e}}}
\newcommand{\frakf}{{\mathfrak{f}}}
\newcommand{\frakg}{{\mathfrak{g}}}
\newcommand{\frakh}{{\mathfrak{h}}}
\newcommand{\fraki}{{\mathfrak{i}}}
\newcommand{\frakj}{{\mathfrak{j}}}
\newcommand{\frakk}{{\mathfrak{k}}}
\newcommand{\frakl}{{\mathfrak{l}}}
\newcommand{\frakm}{{\mathfrak{m}}}
\newcommand{\frakn}{{\mathfrak{n}}}
\newcommand{\frako}{{\mathfrak{o}}}
\newcommand{\frakp}{{\mathfrak{p}}}
\newcommand{\frakq}{{\mathfrak{q}}}
\newcommand{\frakr}{{\mathfrak{r}}}
\newcommand{\fraks}{{\mathfrak{s}}}
\newcommand{\frakt}{{\mathfrak{t}}}
\newcommand{\fraku}{{\mathfrak{u}}}
\newcommand{\frakv}{{\mathfrak{v}}}
\newcommand{\frakw}{{\mathfrak{w}}}
\newcommand{\frakx}{{\mathfrak{x}}}
\newcommand{\fraky}{{\mathfrak{y}}}
\newcommand{\frakz}{{\mathfrak{z}}}
\newcommand{\frakA}{{\mathfrak{A}}}
\newcommand{\frakB}{{\mathfrak{B}}}
\newcommand{\frakC}{{\mathfrak{C}}}
\newcommand{\frakD}{{\mathfrak{D}}}
\newcommand{\frakE}{{\mathfrak{E}}}
\newcommand{\frakF}{{\mathfrak{F}}}
\newcommand{\frakG}{{\mathfrak{G}}}
\newcommand{\frakH}{{\mathfrak{H}}}
\newcommand{\frakI}{{\mathfrak{I}}}
\newcommand{\frakJ}{{\mathfrak{J}}}
\newcommand{\frakK}{{\mathfrak{K}}}
\newcommand{\frakL}{{\mathfrak{L}}}
\newcommand{\frakM}{{\mathfrak{M}}}
\newcommand{\frakN}{{\mathfrak{N}}}
\newcommand{\frakO}{{\mathfrak{O}}}
\newcommand{\frakP}{{\mathfrak{P}}}
\newcommand{\frakQ}{{\mathfrak{Q}}}
\newcommand{\frakR}{{\mathfrak{R}}}
\newcommand{\frakS}{{\mathfrak{S}}}
\newcommand{\frakT}{{\mathfrak{T}}}
\newcommand{\frakU}{{\mathfrak{U}}}
\newcommand{\frakV}{{\mathfrak{V}}}
\newcommand{\frakW}{{\mathfrak{W}}}
\newcommand{\frakX}{{\mathfrak{X}}}
\newcommand{\frakY}{{\mathfrak{Y}}}
\newcommand{\frakZ}{{\mathfrak{Z}}}
% sets as Euler fraks:
\newcommand{\seta}{{\mathfrak{a}}}
\newcommand{\setb}{{\mathfrak{b}}}
\newcommand{\setc}{{\mathfrak{c}}}
\newcommand{\setd}{{\mathfrak{d}}}
\newcommand{\sete}{{\mathfrak{e}}}
\newcommand{\setf}{{\mathfrak{f}}}
\newcommand{\setg}{{\mathfrak{g}}}
\newcommand{\seth}{{\mathfrak{h}}}
\newcommand{\seti}{{\mathfrak{i}}}
\newcommand{\setj}{{\mathfrak{j}}}
\newcommand{\setk}{{\mathfrak{k}}}
\newcommand{\setl}{{\mathfrak{l}}}
\newcommand{\setm}{{\mathfrak{m}}}
\newcommand{\setn}{{\mathfrak{n}}}
\newcommand{\seto}{{\mathfrak{o}}}
\newcommand{\setp}{{\mathfrak{p}}}
\newcommand{\setq}{{\mathfrak{q}}}
\newcommand{\setr}{{\mathfrak{r}}}
\newcommand{\sets}{{\mathfrak{s}}}
\newcommand{\sett}{{\mathfrak{t}}}
\newcommand{\setu}{{\mathfrak{u}}}
\newcommand{\setv}{{\mathfrak{v}}}
\newcommand{\setw}{{\mathfrak{w}}}
\newcommand{\setx}{{\mathfrak{x}}}
\newcommand{\sety}{{\mathfrak{y}}}
\newcommand{\setz}{{\mathfrak{z}}}
\newcommand{\setA}{{\mathfrak{A}}}
\newcommand{\setB}{{\mathfrak{B}}}
\newcommand{\setC}{{\mathfrak{C}}}
\newcommand{\setD}{{\mathfrak{D}}}
\newcommand{\setE}{{\mathfrak{E}}}
\newcommand{\setF}{{\mathfrak{F}}}
\newcommand{\setG}{{\mathfrak{G}}}
\newcommand{\setH}{{\mathfrak{H}}}
\newcommand{\setI}{{\mathfrak{I}}}
\newcommand{\setJ}{{\mathfrak{J}}}
\newcommand{\setK}{{\mathfrak{K}}}
\newcommand{\setL}{{\mathfrak{L}}}
\newcommand{\setM}{{\mathfrak{M}}}
\newcommand{\setN}{{\mathfrak{N}}}
\newcommand{\setO}{{\mathfrak{O}}}
\newcommand{\setP}{{\mathfrak{P}}}
\newcommand{\setQ}{{\mathfrak{Q}}}
\newcommand{\setR}{{\mathfrak{R}}}
\newcommand{\setS}{{\mathfrak{S}}}
\newcommand{\setT}{{\mathfrak{T}}}
\newcommand{\setU}{{\mathfrak{U}}}
\newcommand{\setV}{{\mathfrak{V}}}
\newcommand{\setW}{{\mathfrak{W}}}
\newcommand{\setX}{{\mathfrak{X}}}
\newcommand{\setY}{{\mathfrak{Y}}}
\newcommand{\setZ}{{\mathfrak{Z}}}

% other commonly defined commands:
\newcommand{\wal}{{\rm wal}}
\newcommand{\floor}[1]{\left\lfloor #1 \right\rfloor} % floor
\newcommand{\ceil}[1]{\left\lceil #1 \right\rceil}    % ceil
\DeclareMathOperator{\cov}{Cov}
\DeclareMathOperator{\var}{Var}
\providecommand{\argmin}{\operatorname*{argmin}}
\providecommand{\argmax}{\operatorname*{argmax}}

\begin{document}
\frame{\titlepage}

\begin{frame}
\frametitle{Approche primale}

%Aspects similaires à la programmation linéaire:
%\begin{itemize}
%	\item
%	algorithmes itératifs
%	\item
%	pour un itéré donné, calcul d'une direction de recherche, puis d'une longueur de pas le long de cette direction.
%\end{itemize}

%\mbox{}

Considérons le programme
\begin{align*}
\min_x \ & c^Tx \\
\mbox{s.à. } & Ax = b\\
& x \geq 0.
\end{align*}

\mbox{}

Définissons les ensembles
\begin{align*}
\mathcal{F}_P \overset{\mbox{def}}{=} \lbrace x \,|\, Ax = b, x \geq 0 \rbrace\\
\mathring{\mathcal{F}}_P \overset{\mbox{def}}{=} \lbrace x \,|\, Ax = b, x > 0 \rbrace\\
\end{align*}

\end{frame}

\begin{frame}
\frametitle{Problème barrière}

On suppose que $\mathring{\mathcal{F}}_P$ est non vide et que l'ensemble de solutions optimales pour ce problème est borné.

\mbox{}

Soit $\mu \geq 0$. Problème barrière (PB):
\begin{align*}
\min_x \ & c^Tx - \mu \sum_{j = 1}^n \log x_j \\
\mbox{s.à. } & Ax = b \\
& x > 0.
\end{align*}

\end{frame}

\begin{frame}
\frametitle{Problème barrière vs Problème linéaire}

Si $\mu = 0$, on retrouve le problème original en permettant à $x$ d'avoir des composantes nulles:
\begin{align*}
\min_x \ & c^Tx \cancel{- \mu \sum_{j = 1}^n \log x_j} \\
\mbox{s.à. } & Ax = b \\
& x \boldsymbol{\geq} 0.
\end{align*}

\end{frame}

\begin{frame}
\frametitle{Chemin central}

Soit $x(\mu)$, la solution au PB étant donné $\mu$.
En faisant varier $\mu$ continûment vers 0, nous obtenons le chemin central primal.

\mbox{}

Si $\mu \rightarrow \infty$, la solution s'approche du centre analytique de la région réalisable, lorsque celle-ci est bornée: le terme barrière prédomine alors dans l'objectif.

\mbox{}

Comme $\mu \rightarrow 0$, ce chemin converge vers le centre analytique de la face optimale $\lbrace x \,|\, c^Tx = z^*,\ Ax = b,\ x \geq 0 \rbrace$, où $z^*$ est la valeur optimale du programme linéaire.

\mbox{}

L'idée de base est de résoudre une succession de problèmes barrières pour des valeurs décroissantes de $\mu$.

\end{frame}

\begin{frame}
\frametitle{Centre analytique}

Considérons un ensemble $\mathcal{S}$ dans un sous-ensemble $\mathcal{X}$ de $E^n$, défini comme
\[
\mathcal{S} = \lbrace \bsx \in \mathcal{X}\,:\, g_j(\bsx) \geq 0,\ j = 1,2,\ldots,m \rbrace,
\]
et supposons que les fonctions $g_j$ sont continues.

\mbox{}

Nous supposons aussi que $\mathcal{S}$ a un intérieur non-vide:
\[
\mathring{\mathcal{S}} = \lbrace \bsx \in \mathcal{X} \,:\, g_j(\bsx) > 0,\ \forall\, j \rbrace \ne \emptyset.
\]

\mbox{}

Définissons aussi sur $\mathring{\mathcal{S}}$ la {\em fonction potentiel}
\[
\psi(\bsx) = -\sum_{j = 1}^m \log g_j(\bsx).
\]

\end{frame}

\begin{frame}
\frametitle{Centre analytique}

Le {\em centre analytique} de $\mathcal{S}$ est le vecteur (ou l'ensemble de vecteurs) qui minimise le potentiel.

\mbox{}

En d'autres termes, c'est le vecteur (ou l'ensemble de vecteurs) qui résout
\[
\min_{\bsx} \psi(\bsx) =
\min_{\bsx}
\left\lbrace
-\sum_{j = 1}^m \log g_j(\bsx) \,:\, \bsx \in \mathcal{X},\ g_j(\bsx) > 0\ \forall\, j
\right\rbrace.
\]

\end{frame}

\begin{frame}
\frametitle{Exemple: un cube}

Considérons l'ensemble $\mathcal{S}$ défini par $x_i \geq 0$, $(1-x_i) \geq 0$, pour $i = 1,2,\ldots,n$:
\[
\mathcal{S} = [0,1]^n,
\]
le cube unité dans $E^n$.

\mbox{}

Cherchons à minimser la fonction potentiel correspondante
\[
\psi(\bsx) = -\sum_{i = 1}^n \log x_i - \sum_{i = 1}^n \log (1-x_i).
\]

\end{frame}

\begin{frame}
\frametitle{Exemple: un cube}

Pour ce faire, annulons le gradient de $\psi(\bsx)$
\[
\nabla_x \psi(x) =
\begin{pmatrix}
- \frac{1}{x_1} + \frac{1}{1-x_1} \\
- \frac{1}{x_2} + \frac{1}{1-x_2} \\
\vdots \\
- \frac{1}{x_n} + \frac{1}{1-x_n} \\
\end{pmatrix}
\]
(Note: $\psi(x)$ est ici une fonction convexe.)

\mbox{}

On en tire $x_i = 1/2$ pour tout $i$.
Dès lors, le centre analytique est identique à l'idée usuelle du centre du cube unité.

\mbox{}

Toutefois, en général, le centre analytique dépend de la manière dont l'ensemble est défini, en particulier les inégalités utilisées dans la définition.

\end{frame}

\begin{frame}
\frametitle{Exemple: un cube}

Par exemple, nous pourrions aussi définir le cube unité avec les inégalités
\begin{align*}
x_i &\geq 0, \ i = 1,\ldots,n\\
(1-x_i)^d &\geq 0,\ i = 1,\ldots,n,
\end{align*}
avec $d > 1$, et $d$ impair.
Dans ce cas, la solution est
\[
x_i = \frac{1}{d+1},\ \forall\, i.
\]
Pour un grand $d$, ce point est proche du coin intérieur, i.e. du point $(0,0,\ldots,0)$ du cube unité.

\mbox{}

De plus, l'ajout d'inégalités additionnelles redondantes peut aussi modifier la position du centre analytique. Répéter par exemple une inégalité donnée changera la position du centre.

\end{frame}

\begin{frame}
\frametitle{Centre analytique}

Il y a plusieurs ensembles associés avec des programmes linéaires pour lesquels le centre analytique est d'un intérêt particulier.

\mbox{}

Un tel ensemble est la région réalisable elle-même. Un autre est l'ensemble des solutions optimales. Il y a aussi les ensembles associés avec les formulations duale et primale-duale. Tous ceux-ci sont en fait reliés.

\mbox{}

Considérons par exemple le centre analytique associé à un \textcolor{red}{polytope borné} $\Omega$ dans $\RR^m$, représenté par $n$ ($> m$) inégalités linéaires:
\[
\Omega = \lbrace \bsy \in \RR^m \,:\, \bsc^T - \bsy^T \bsA \geq \bszero \rbrace,
\]
où $\bsA \in \RR^{m \times n}$ et $c \in \RR^n$ sont données, et $\bsA$ est de rang $m$.

\end{frame}

\begin{frame}
\frametitle{Centre analytique}

Dénotons l'intérieur de $\Omega$ par
\[
\mathring{\Omega} = \lbrace \bsy \in \RR^m \,:\, \bsc^T - \bsy^T\bsA > 0 \rbrace.
\]

\mbox{}

La fonction potentiel pour cet ensemble est
\[
\psi_{\Omega} (\bsy) = - \sum_{j = 1}^n \log (c_j - \bsy^T\bsa_j) = -\sum_{j = 1}^n \log s_j,
\]
où $\bss = \bsc^T - \bsy^T\bsA$ est un vecteur d'écart.

\mbox{}

Dès lors, dans le cas présent, la fonction potentiel est l'opposé de la somme des logarithmes des variables d'écart, ou encore le produit de leur inverse:
$$
-\sum_{j = 1}^n \log s_j = \log \prod_{j = 1}^n \frac{1}{s_j}
$$

\end{frame}

\begin{frame}
\frametitle{Centre analytique}

Le centre analytique de $\Omega$ est le point de $\Omega$ qui minimise la fonction potentiel. Ce point, dénoté $\bsy^a$ est associé à $\bss^a = \bsc - \bsA\bsy^a$. La paire $(\bsy^a, \bss^a)$ est définie de manière unique, vu que la fonction potentiel est strictement convexe sur un ensemble convexe borné.

\mbox{}

Notons aussi que minimiser la fonction potentiel revient ici à maximiser le produit des variables d'écart.

\mbox{}

Comme,
$$
\frac{d}{dy_i} (c_j-\bsy^T\bsa_j) = -a_{ij},
$$
nous avons
$$
\frac{d}{dy_i} \psi(\bsy) = \sum_{j = 1}^n \frac{a_{ij}}{c_j - \bsy^T\bsa_j}.
$$

\end{frame}

\begin{frame}
\frametitle{Centre analytique}

Annuler le gradient de $\psi(\bsy)$ par rapport à $\bsy$ donne
\[
\sum_{j = 1}^n \frac{a_{ij}}{c_j - \bsy^T\bsa_j} = 0,\ \forall\, i,
\]
ou encore
\[
\sum_{j = 1}^n \frac{a_{ij}}{s_j} = 0,\ \forall\, i,
\]
Définissons $x_j = 1/s_j$ pour tout $j$. Nous pouvons réécrire l'équation comme
$$
\sum_{j = 1}^n a_{ij}x_j = 0,\ \forall\, i,
$$
soit
$$
\bsA\bsx = \bszero.
$$

\end{frame}

\begin{frame}
\frametitle{Centre analytique}

Notons la multiplication composante par composante comme suit
\[
\bsx \circ \bss = (x_1s_1, x_2s_2, \ldots, x_ns_n)^T.
\]
Le \textcolor{blue}{centre analytique} est alors défini par les conditions
\begin{align*}
\bsx \circ \bss &= \bsone \\
\bsA\bsx &= \bszero \\
\bsA^T\bsy + \bss & = \bsc.
\end{align*}

\end{frame}

\begin{frame}
\frametitle{Centre analytique}

Le centre analytique peut aussi être défini quand l'intérieur est vide ou que des égalités sont présentes, par exemple avec
\[
\Omega = \lbrace \bsy \in \RR^m \,:\, \bsc^T-\bsy^T\bsA \geq 0,\, \bsB\bsy = \bsb \rbrace
\]
Dans ce cas, le centre analytique est choisi sur la surface linéaire $\lbrace \bsy \,:\, \bsB\bsy = \bsb \rbrace$ pour maximiser le produit des variables d'écart.

\mbox{}

Dans ce contexte, l'intérieur de $\Omega$ réfère à l'intérieur de l'orthant positif des variables d'écart:
\[
R^n_+ := \lbrace \bss \,:\, \bss \geq 0 \rbrace.
\]
Cette définition d'intérieur dépend seulement de la région des variables d'écart.

\mbox{}

Meme s'il y a seulement un point dans $\Omega$ avec $\bss = \bsc - \bsA^T\bsy$ pour un certain $\bsy$ où $\bsB\bsy = \bsb$ avec $\bss > 0$, nous continuerons à dire que $\mathring{\Omega}$ est non vide.

\end{frame}


\begin{frame}
\frametitle{Fonction lagrangienne}

Réécrivons la contrainte $Ax = b$ sous la forme $Ax - b = 0$, et introduisons un vecteur $y$, en associant une composante $y_i$ à la $i^e$ contrainte $\sum_{j = 1}^n a_{ij} x_j = b_i$.

\mbox{}

$y_i$ joue à peu près le même rôle que les variables duales précédentes, mais est appelé à présent multiplicateur de Lagrange.

\mbox

Lagrangien:
\[
L(x) = c^Tx - \mu \sum_{j = 1}^n \log x_j - y^T(Ax - b).
\]
On cherche à minimiser cette fonction en annulant son gradient.

\end{frame}

\begin{frame}
\frametitle{Minimisation du lagrangien}

\begin{align*}
& \nabla_x L(x) = 0 \\
\Leftrightarrow\ &
c_j - \frac{\mu}{x_j} - y^Ta_j = 0,\ j = 1,\ldots,n\\
\Leftrightarrow\ &
\mu X^{-1}\bsone + A^Ty = c,
\end{align*}
où $X = \text{diag}(x)$.

\mbox{}

En notant $s_j = \mu/x_j$, l'ensemble des conditions d'optimalité (primales, duales, minimisation du lagrangien) s'écrit:
\begin{align*}
\bsx \circ \bss &= \mu \bsone \\
\bsA\bsx &= \bsb \\
\bsA^T\bsy + \bss & = \bsc.
\end{align*}

\end{frame}

\begin{frame}
\frametitle{Lien avec le dual}

$y$ est une solution duale réalisable et $c - A^T y > 0$.

\mbox{}

En effet, le dual est
\begin{align*}
\max_y\ \ & y^T b \\
\mbox{s.à. } & A^T y \leq c
\end{align*}
Comme
\begin{align*}
A^Ty + s &= c\\
s_j &= \frac{\mu}{x_j}
\end{align*}
nous avons
\[
A^T y < c.
\]

\end{frame}

\begin{frame}
\frametitle{Exemple}

Considérons le programme
\begin{align*}
\max\ & x_1 \\
\mbox{s.à. } & 0 \leq x_1 \leq 1 \\
& 0 \leq x_2 \leq 1.
\end{align*}

\mbox{}

On réécrit ce système sous forme standard
\begin{align*}
\min\ & -x_1 \\
\mbox{s.à. } & x_1 + x_3 = 1\\
& x_2 + x_4 = 1 \\
& x_1 \geq 0, x_2 \geq 0, x_3 \geq 0, x_4 \geq 0.
\end{align*}

\end{frame}

\begin{frame}
\frametitle{Conditions d'optimalité}

\begin{align*}
x_1 s_1 &= \mu \\
x_2 s_2 &= \mu \\
x_3 s_3 &= \mu \\
x_4 s_4 &= \mu \\
x_1 + x_3 &= 1 \\
x_2 + x_4 &= 1 \\
y_1 + s_1 &= -1 \\
y_2 + s_2 &= 0 \\
y_1 + s_3 &= 0 \\
y_2 + s_4 &= 0
\end{align*}

\end{frame}

\begin{frame}
\frametitle{Conditions d'optimalité}

De
\begin{align*}
y_2 + s_2 = 0 \\
y_2 + s_4 = 0 \\
\end{align*}
on a $s_2 = s_4$.

\mbox{}

On en déduit aussi
\[
x_2 = x_4,
\]
et de là
\[
x_2 = x_4 = \frac{1}{2}.
\]

\end{frame}

\begin{frame}
\frametitle{Conditions d'optimalité}

On a aussi
\begin{align*}
& -1 = s_1 - s_3 = \frac{\mu}{x_1} - \frac{\mu}{x_3} \\
\Leftrightarrow\ & -1 = \frac{\mu}{x_1} - \frac{\mu}{1-x_1} \\
\Leftrightarrow\ & -x_1(1-x_1) = \mu(1-x_1) - \mu x_1 \\
\Leftrightarrow\ & x_1^2 - x_1 = \mu - 2\mu x_1 \\
\Leftrightarrow\ & x_1^2 - (1-2\mu) x_1 -\mu = 0.
\end{align*}
Le discriminant de cette équation quadratique est
\[
\rho = (1-2\mu)^2 + 4\mu = 1+ 4\mu^2,
\]
et de là
\[
x_1 = \frac{1-2\mu \pm \sqrt{1+4\mu^2}}{2}
\]

\end{frame}

\begin{frame}
\frametitle{Chemin central}

Pour $\mu$ grand, on doit choisir la racine correspondant à `+', car avec `-', on a
$$
1-2\mu-\sqrt{1+4\mu^2} < 1-2\mu-\sqrt{4\mu^2} = 1-4\mu
$$

D'autre part,
\[
x \rightarrow \left( 1, \frac{1}{2} \right), \mbox{ comme } \mu \rightarrow 0.
\]

\mbox{}

On converge vers le centre analytique de la face optimale
\[
\lbrace x \,|\, x_1 = 1,\ 0 \leq x_2 \leq 1\rbrace
\]
plutôt qu'un coin du carré.

\end{frame}

\begin{frame}
\frametitle{Chemin central}

De plus,
\begin{align*}
\lim_{\mu \rightarrow +\infty} \frac{1-2\mu + \sqrt{1+4\mu^2}}{2}
&= \frac{1}{2} - \frac{1}{2} \lim_{\mu \rightarrow +\infty} \left( 2\mu - \sqrt{1+4\mu^2} \right) \\
&= \frac{1}{2} - \frac{1}{2} \lim_{\mu \rightarrow +\infty} \frac{\left( 2\mu - \sqrt{1+4\mu^2} \right)\left( 2\mu + \sqrt{1+4\mu^2} \right)}{\left( 2\mu + \sqrt{1+4\mu^2} \right)} \\
&= \frac{1}{2} - \frac{1}{2} \lim_{\mu \rightarrow +\infty} \frac{4\mu^2 - 1 -4\mu^2}{\left( 2\mu + \sqrt{1+4\mu^2} \right)} \\
&= \frac{1}{2} + \frac{1}{2} \lim_{\mu \rightarrow +\infty} \frac{1}{\left( 2\mu + \sqrt{1+4\mu^2} \right)} \\
& = \frac{1}{2}.
\end{align*}

\end{frame}

\begin{frame}
\frametitle{Chemin central}

Ainsi $
x(\mu) \overset{\mu \rightarrow +\infty}{\longrightarrow} \left( \frac{1}{2}, \frac{1}{2} \right).
$

\mbox{}

Le chemin central est une ligne droite progressant du centre analytique du carré (comme $\mu \rightarrow \infty$) vers le \textcolor{blue}{centre analytique de la face optimale} (comme $\mu \rightarrow 0$).

\end{frame}

\begin{frame}
\frametitle{Chemin dual central}

Partons à présent du problème dual
\begin{align*}
\max\ & y^Tb \\
\mbox{s.à. } & y^TA + s^T = c^T \\
& s \geq 0.
\end{align*}

\mbox{}

Le problème barrière associé est
\begin{align*}
\max\ & y^Tb + \mu \sum^n_{j = 1} \log s_j \\
\mbox{s.à. } & y^TA + s^T = c^T \\
& s \geq 0.
\end{align*}

\end{frame}

\begin{frame}
\frametitle{Chemin dual central}

On suppose que $\mathring{\mathcal{F}}_D = \lbrace (y,s): y^TA + s^T = c^T,\ s > 0 \rbrace$ est non vide, et l'ensemble des solutions optimales du dual est borné.

\mbox{}

On obtient le chemin central dual en faisant tendre $\mu$ vers 0.

\mbox{}

Lagrangien:
\[
L(y,s) = y^Tb + \mu \sum_{j=1}^n \log s_j - (y^TA + s^T - c^T)x.
\]

\mbox{}

Dès lors,
\begin{align*}
\nabla_y L &= 0 \Leftrightarrow b_i - a^i x = 0,\ \forall i ,\\ 
\nabla_s L &= 0 \Leftrightarrow \frac{\mu}{s_j} - x_j = 0,\ \forall j.
\end{align*}

\end{frame}

\begin{frame}
\frametitle{Chemin dual central: conditions d'optimalité}

On obtient les conditions d'optimalité
\begin{align*}
\bsx \circ \bss &= \mu \bsone \\
\bsA\bsx &= \bsb \\
\bsA^T\bsy + \bss & = \bsc.
\end{align*}

\mbox{}

On retrouve les mêmes conditions que pour le chemin central.

\mbox{}

Par conséquent, $x$ est une solution réalisable primale et $x > 0$.

\mbox{}

Considérons l'ensemble de niveau dual
\[
\Omega(z) = \lbrace y \,|\, c^T - y^T A \geq 0, y^T b = z\rbrace,
\]
avec $z < z^*$, la valeur optimale du dual.

\end{frame}

\begin{frame}
\frametitle{Chemin dual central: conditions d'optimalité}

Le centre analytique $(y(z), s(z))$ de $\Omega$ coïncide avec le chemin central dual comme $z$ tend vers $z^*$.

%voir p.52 des notes manuscrites

\end{frame}

\begin{frame}
\frametitle{Chemin dual central: exemple}

Reprenons le problème
\begin{align*}
\min\ & -x_1 \\
\mbox{s.à. } & x_1 + x_3 = 1 \\
& x_2 + x_4 = 1 \\
& x_1 \geq 0, x_2 \geq 0, x_3 \geq 0, x_4 \geq 0.
\end{align*}

Le dual s'écrit
\begin{align*}
\max\ & y_1 + y_2 \\
\mbox{s.à. } & y_1 \leq -1 \\
& y_2 \leq 0 \\
& y_1 \leq 0 \\
& y_2 \leq 0.
\end{align*}

\end{frame}

\begin{frame}
\frametitle{Chemin dual central: exemple}

On peut réécrire le dual comme
\begin{align*}
\max\ & y_1 + y_2 \\
\mbox{s.à. } & y_1 + s_1 = -1 \\
& y_2 + s_2 = 0 \\
& s_1 \geq 0, s_2 \geq 0.
\end{align*}

\mbox{}

Les conditions d'optimalité sont les mêmes que pour le primal, aussi
\[
x_2 = x_4 = \frac{1}{2},
\]
d'où
\begin{align*}
%s_2 = s_4 = 2\mu,
s_2 &= 2\mu, \\
y_2 &= -2\mu.
\end{align*}

\end{frame}

\begin{frame}
\frametitle{Chemin dual central: exemple}

Nous avons également, en se rappelant de la résolution du primal,
\begin{align*}
y_1 & = -1 - s_1 \\
& = -1 - \frac{\mu}{x_1(\mu)} \\
& = -1 - \frac{2\mu}{1-2\mu\pm\sqrt{1+4\mu^2}}
\end{align*}

\mbox{}

Comme $\mu \rightarrow 0$, $y_1 \rightarrow -1$, $y_2 \rightarrow 0$.

\mbox{}

Il s'agit de l'unique solution du problème linéaire (les deux contraintes sont actives)

\mbox{}

Si $\mu \rightarrow \infty$, $y$ est non borné, car l'ensemble dual réalisable est non borné.

\end{frame}

\begin{frame}
\frametitle{Chemin central primal-dual}

\textcolor{blue}{Hypothèse}: la région réalisable du problème (primal) de programmation linéaire a un intérieur non vide et un ensemble borné de solutions optimales.

%\mbox{}
% Est-ce vraiment le cas? La preuve ne semble pas évidente, et si c'est le cas, pourquoi a-t-on les hypothèses au slide suivant?
% Le dual a un intérieur réalisable non vide, en vertu des conditions (d'optimalité) sur le lagrangien.

\mbox{}

Le chemin primal-dual est l'ensemble des vecteurs $(x(\mu), y(\mu), s(\mu))$ satisfaisant
\begin{align*}
\bsx \circ \bss &= \mu \bsone \\
\bsA\bsx &= \bsb \\
\bsA^T\bsy + \bss & = \bsc \\
x \geq 0,\ & s \geq 0 \\
0 \leq \mu & < \infty.
\end{align*}

\end{frame}

\begin{frame}
\frametitle{Proposition}

Si les ensembles réalisables primal et dual ont des intérieurs non vides, alors le chemin central $(x(\mu), y(\mu), s(\mu))$ existe pour tout $\mu$, $0 \leq \mu < \infty$.

\mbox{}

De plus, $x(\mu)$ est le chemin central primal, $(y(\mu), s(\mu))$ est le chemin central dual.

\mbox{}

$x(\mu)$ et $(y(\mu), s(\mu))$ convergent vers les centres analytiques des faces des solutions optimales primales et duales, respectivement, quand $\mu \rightarrow 0$.

\end{frame}

\begin{frame}
\frametitle{Saut de dualité}

Soit $(x(\mu), y(\mu), s(\mu))$ sur le chemin central primal-dual.
Nous avons
\begin{align*}
c^Tx - y^Tb &= (A^T y)^T x + s^Tx - y^T b \\
& = y^T Ax + s^Tx - y^T b \\
& = y^T b + s^Tx - y^T b \\
& = s^Tx \\
& = n\mu.
\end{align*}

\mbox{}

Comme pour la dualité faible, $c^T x \geq y^T b$, et $n\mu$ est appelé le saut de dualité.

\mbox{}

Soit $g = c^Tx - y^Tb$.

\mbox{}

Comme $y^T b \leq z^*$, $z^* \geq c^Tx - g$, et donc, étant donné $(x, y, s)$, on peut mesurer la qualité de $x$ comme $c^T x - z^* \leq g$.
\end{frame}

\begin{frame}
	\frametitle{Approche générale}
	
	Si on commence avec une valeur $\mu_0$, à l'itération $k$,
	\[
	\mu_{k} = \gamma^k \mu_0,
	\]
	
	\mbox{}
	
	Dès, réduire $\mu_k/\mu_0$ sous $\epsilon$ requiert
	\[
	k = \left\lceil \frac{\log \epsilon}{\log \gamma} \right\rceil.
	\]
	
	\mbox{}
	
	Souvent, une variante de la méthode de Newton est utilisée pour résoudre les sous-problèmes ainsi construits:
	\begin{align*}
		\bsx \circ \bss &= \mu \bsone \\
		\bsA\bsx &= \bsb \\
		\bsA^T\bsy + \bss & = \bsc
	\end{align*}
	
\end{frame}

\begin{frame}
	\frametitle{Algorithme général}
	
	\begin{description}
		\item[Etape 1]
		Choisir $\mu_0$ et un point de départ réalisable $(x_0, y_0, s_0)$, tel que
		\begin{align*}
			Ax_0 &= b \\
			A^Ty_0 + s_0 &= c^T
		\end{align*}
		et $s_0 \geq 0$, $x_0 > 0$. Poser $k = 0$, et choisir $\epsilon > 0$.
		\item[Etape 2]
		Projeter $x_k$ sur le chemin central avec la méthode de Newton: calculer le vecteur $(d_x, d_y, d_s)$ et poser
		\begin{align*}
			x_{k+1} &= x_k + d_x \\
			y_{k+1} &= y_k + d_y \\
			s_{k+1} &= s_k + d_s
		\end{align*}
		Si $\mu_k < \epsilon$, arrête.
		Sinon, définir $\mu_{k + 1} = \gamma \mu_k$, poser $k := k+1$, et retourner au début de l'étape 2.
	\end{description}
	
\end{frame}

\begin{frame}
\frametitle{Stratégies de solution en points intérieurs}

Trois grandes approches, suivant les différences dans les définitions du chemin central:
\begin{enumerate}
	\item
	barrière primale, méthode de poursuite de chemin,
	\item
	méthode primale-duale de poursuite de chemin,
	\item
	méthode primale-duale de réduction de potentiel.
\end{enumerate}

\mbox{}

Caractéristiques
\begin{tabular}{c|ccc}
	& R-P & R-D & Saut nul \\
	\hline
	simplexe primal & X & & X \\
	simplexe dual & & X & X\\
	barrière primale & X & & \\
	poursuite de chemin primale-duale & X & X & \\
	réduction de potentiel primale-duale & X & X & \\
\end{tabular}

R: réalisabilité, P: primal, D: dual

\end{frame}

\begin{frame}
\frametitle{Méthode primale barrière}

Nous partons du problème primal barrière
\begin{align*}
\min_x \ & c^Tx - \mu \sum_{j = 1}^n \log x_j \\
\mbox{s.à. } & Ax = b \\
& x \geq 0.
\end{align*}

\mbox{}

Nous voudrions le résoudre pour $\mu$ petit.

\mbox{}

Par exemple, $\mu = \epsilon/n$ permet d'obtenir un saut de dualité inférieur à $\epsilon$.

\mbox{}

Souci: difficile de résoudre pour $\mu$ proche de 0.

\end{frame}

\begin{frame}
\frametitle{Méthode primale barrière}

Une stratégie générale est de commencer avec $\mu$ modérément large (p.e. $\mu = 100$) et de résoudre le problème approximativement.

\mbox{}

La solution correspondante est approximativement sur le chemin central primal, mais probablement assez loin du point correspondant à $\mu \rightarrow 0$.

\mbox{}

Ce point ne servira que de point de départ pour le problème avec un $\mu$ plus petit.

\mbox{}

Typiquement, on mettra à jour $\mu$ de l'itération $k$ à l'itération $k+1$ comme
\[
\mu_{k+1} = \gamma \mu_k,
\]
pour $0< \gamma < 1$ fixé.

\end{frame}

\begin{frame}
\frametitle{Méthode primale barrière}

Nous pouvons en particulier appliquer la méthode de Newton pour trouver un zéro du gradient associé au lagrangien du problème de barrière logarithmique primale.

\mbox{}

Rappelons que le lagrangien du problème barrière logarithmique primale est
$$
L(x) = c^Tx - \mu \sum_{j = 1}^n \log x_j - y^T(Ax-b)
$$

\mbox{}

Il est facile d'exprimer le gradient et la matrice hessienne du lagrangien:
\begin{align*}
\nabla L(x) &= c-\mu X^{-1}\bsone - A^Ty \\
H(x) & = \mu X^{-2}
\end{align*}

\end{frame}

\begin{frame}
	\frametitle{Système d'équation non linéaires: méthode de Newton}
	
	Considérons le système d'équations non linéaires
	$$
	F(\bsx) = 0
	$$
	avec $F: \RR^n \rightarrow \RR^m$.
	
	\mbox{}
	
	En partant d'un point $\bsx_0$, la méthode de Newton construit une suite d'itérés à partir de la récurrence
	$$
	\bsx_{k+1} = \bsx_k - J^{-1}(\bsx_k) F(\bsx_k)
	$$
	où $J(\bsx_k)$ est la matrice jacobienne de $F$:
	\[
	J(\bsx) =
	\begin{pmatrix}
		\nabla^T_x f_1(x) \\
		\nabla^T_x f_2(x) \\
		\vdots \\
		\nabla^T_x f_n(x) \\
	\end{pmatrix}
	\]
	
	La méthode converge si on est suffisamment proche d'un zéro de la fonction.
	
\end{frame}

\begin{frame}
	\frametitle{Méthode de Newton en optimisation}

Supposons que $f \in C^2$.
On peut appliquer la méthode de Newton à la résolution du sytème non-linéaire
$$
\nabla f(x) = 0.
$$
Dans ce cas, la matrice jacobienne de $f$ n'est rien d'autre que la matrice hessienne:
\begin{align*}
J(\bsx) &=
\begin{pmatrix}
	\nabla^T_x \frac{d}{dx_1} f(\bsx) \\
	\nabla^T_x \frac{d}{dx_2} f(\bsx) \\
	\vdots \\
	\nabla^T_x \frac{d}{dx_n} f(\bsx)
\end{pmatrix}
=
\begin{pmatrix}
	\frac{d^2}{dx_1^2} f(\bsx) & \frac{d^2}{dx_1x_2} f(\bsx) & \cdots & \frac{d^2}{dx_1x_n} f(\bsx) \\
	\frac{d^2}{dx_1x_2} f(\bsx) & \frac{d^2}{dx_2^2} f(\bsx) & \cdots & \frac{d^2}{dx_2x_n} f(\bsx) \\
	\vdots & \vdots & \ddots & \vdots \\
	\frac{d^2}{dx_1x_n} f(\bsx) & \frac{d^2}{dx_2x_n} f(\bsx) & \cdots & \frac{d^2}{dx_n^2} f(\bsx)
\end{pmatrix} \\
&= H(\bsx)
\end{align*}
	
\end{frame}

\begin{frame}
\frametitle{Méthode de Newton}

La récurrence de Newton devient dès lors
$$
\bsx_{k+1} = \bsx_k - H^{-1}(\bsx_k) \nabla f(\bsx_k).
$$
On suppose ici que $H(\bsx_k)$ est définie positive.

\mbox{}

Note: nous ne calculons pas $H^{-1}(\bsx_k)$ pour des questions de précision numériques, mais nous obtenons le pas $\bsd_k$ en résolvant le système linéaire
$$
H(\bsx_k) \bsd_k = -\nabla f(\bsx_k).
$$
La récurrence de Newton se réécrit
$$
\bsx_{k+1} = \bsx_k + \bsd_k.
$$

\end{frame}

\begin{frame}
\frametitle{Méthode de Newton: difficultés}

La convergence n'est assurée que si le point de départ est suffisamment proche de la solution, et est alors quadratique.

\mbox{}

On peut obtenir la convergence globale en utilisant la direction de Newton comme direction de descente avec une méthode de recherche linéaire: on cherche une longueur de pas appropriée $\alpha_k$ pour produire le nouvel itéré
$$
\bsx_{k+1} := \bsx_k+\alpha_k \bsd_k, 
$$
en s'assurant que $\alpha_k$ puisse prendre la valeur 1 afin d'assurer une convergence quadratique au cours des dernières itérations.

\end{frame}

\begin{frame}
\frametitle{Méthode primale barrière}

Dès lors, la méthode de Newton pour la méthode primale barrière conduit à calculer
$$
\bsd_x = - H^{-1}\nabla L(x)
$$
ou encore
\begin{align*}
& -H\bsd_x =  c-\mu X^{-1}\bsone - A^Ty \\
\Leftrightarrow\  &  H(-\bsd_x) + A^Ty =  c-\mu X^{-1}\bsone
\end{align*}

\mbox{}

Nous devons cependant aussi nous assurer que le nouvel itéré $\bsx + \bsd_x$ est réalisable, i.e.
$$
A(\bsx + \bsd_x) =b
$$
Ceci nous mène à imposer $A\bsd_x = 0$.

\end{frame}

\begin{frame}
\frametitle{Méthode primale barrière}

Les deux équations précédentes donnent, sous forme matricielle,
$$
\begin{pmatrix}
H & A^T \\
A & 0
\end{pmatrix}
\begin{pmatrix}
-\bsd_x \\ \bsy
\end{pmatrix}
=
\begin{pmatrix}
c - \mu X^{-1}\bsone \\ 0
\end{pmatrix}.
$$

\mbox{}

Reprenons la première ligne du système
$$
-H\bsd_x + A^T\bsy = c - \mu X^{-1}\bsone
$$
et multiplions de part et d'autre part $AX^2$. Comme $H = \mu X^{-2}$, nous obtenons
$$
-\mu A\bsd_x + AX^2A^T\bsy = AX^2c - \mu AX \bsone.
$$

%Etant donné un point $x \in \mathring{\mathcal{F}}_P$, la méthode de Newton consistera à chercher des direction $\bsd_x$, $\bsd_y$ et $\bsd_s$ à partir du système
%\begin{align*}
%\mu \bX^{-2} \bsd_x + \bsd_s &= \mu \bX^{-1}\bsone - \bsc \\
%\bsA\bsd_x &= \bszero %\\
%-\bsA^T\bsd_y + \bsd_s & = 0
%\end{align*}

Comme $A\bsd_x = 0$, nous obtenons un système linéaire permettant de trouver $\bsy$:
$$
AX^2A^T\bsy = AX^2c - \mu AX \bsone.
$$

\end{frame}

\begin{frame}
\frametitle{Méthode primale barrière}

Repartons de
$$
-H\bsd_x + A^T\bsy = c - \mu X^{-1}\bsone
$$
avec $\bsy$ connu. Nous avons dès lors
\begin{align*}
\bsd_x &= \mu H^{-1}X^{-1}\bsone - H^{-1}c + H^{-1}A^T\bsy \\
&= x + \frac{1}{\mu}X^2(A^T\bsy-c).
\end{align*}

\mbox{}

Nous avons dès lors obtenu la direction $\bsd_x$.
Nous nous déplacerons le long de cette direction d'une longueur de pas $\alpha$, obtenue en minimisant le problème barrière le long de $\bsd_x$:
$$
\min_{\alpha} c^T\bsx^+ - \mu \sum_{j = 1}^n \log \bsx^+_j
$$
avec $\bsx^+ = \bsx+\alpha \bsd_x$

%On construit le nouveau point comme
%\[
%\bsx^+ = \bsx + \bsd_x.
%\]

%\mbox{}

%Si $\bsx \circ \bss = \mu \bsone$ pour un certain $\bss = \bsc - \bsA^T\bsy$, alors $\bsd \equiv (\bsd_x, \bsd_y, \bsd_s) = 0$.\\
%Si une composante de $\bsx \circ \bs$ est plus petite que $\mu$, l'approche tendera à augmenter cette composante, et inversément si la composante est plus grande que $\mu$.

\end{frame}

\begin{frame}
\frametitle{Méthode primale barrière}

La méthode marche relativement bien si $\mu$ est modérément grand, ou si l'algorithme est démarré avec un point proche de la la solution.

%\mbox{}

%Pour trouver $(\bsd_x, \bsd_y, \bsd_s)$, prémultiplions les deux côtés de la première égalité du système de Newton par $\bX^2$:
%\[
%\mu \bsd_x + \bX^2 \bsd_s = \mu \bX \bsone - \bX^2 \bsc.
%\]
%En prémultipliant par $\bsA$ et en utilisant $\bsA d_x = \bszero$, nous avons
%\[
%\bsA \bX^2 \bsd_s = \mu \bsA \bX \bsone - \bsA \bX^2 \bsc.
%\]
%Comme $\bsd_s = \bsA^T \bsd_y$, nous avons
%\[
%\bsA \bX^2 \bsA^T \bsd_y = \mu \bsA \bX \bsone - \bsA \bX^2 \bsc.
%\]
%On en tire $\bsd_y$, et de là, $\bsd_s$ puis $\bsd_x$.

\end{frame}

\begin{frame}
\frametitle{Poursuite de chemin primal-dual}

Une autre approche consiste à suivre le chemin central à partir d'une paire initiales de solutions primale et duale.
À nouveau, considérons la paire
\begin{equation}
\begin{aligned}
\min_x\ & c^T x \\
\mbox{s.c. } & Ax = b \\
& x \geq 0,
\end{aligned}
\tag{P}
\end{equation}
\begin{equation}
\begin{aligned}
\max_y \  & b^T y \\
\mbox{s.c. } & A^T y + s = c \\
& s \geq 0.
\end{aligned}
\tag{D}
\end{equation}

Nous supposons que $\mathring{\mathcal{F}} \ne \emptyset$, ou de manière équivalente
\begin{align*}
	\mathring{\mathcal{F}}_P &= \lbrace x \,|\, Ax = b, x > 0 \rbrace \ne \emptyset, \\
	\mathring{\mathcal{F}}_D &= \lbrace (y,s) \,|\, s = c-A^Ty > 0 \rbrace \ne \emptyset. \\
\end{align*}

\end{frame}

\begin{frame}
\frametitle{Conditions d'optimalité}

\begin{align*}
A^T y + s &= c \\
Ax &= b \\
XS\bsone &= 0 \\
x \geq 0,&\ s \geq 0
\end{align*}

\mbox{}

Les méthodes de points intérieurs primales-duales diffèrent des méthodes primales strictes en un point-clé: la direction de recherche est calculée tant pour $x$ que pour $(y,s)$, mais numériquement, ce changement a révolutionné la programmation linéaire, et l'approche est maintenant à la base de nombreux solveurs.

\end{frame}

\begin{frame}
\frametitle{Points intérieurs}

Nous perturbons les conditions de complémentarité pour obtenir
\begin{align*}
A^T y + s &= c \\
Ax &= b \\
XS\bsone &= \mu\bsone \\
x \geq 0,&\ s > 0
\end{align*}

On cherche dès lors un zéro du système
$$
F(x,y,s) =
\begin{pmatrix}
A^T y + s - c \\
Ax - b \\
XS\bsone - \mu\bsone
\end{pmatrix}
$$
avec $x, s > 0$.

\end{frame}

\begin{frame}
\frametitle{Points intérieurs}

Appliquons la méthode de Newton pour résoudre ce système.

\mbox{}

Remarquons que $J(Ax) = A$. En effet, la $i^e$ composante de $Ax$ vaut
$$
(Ax)_i = \sum_{j = 1}^n a_{ij} x_j.
$$
Dès lors
$$
\nabla_x (Ax)_i = a^i.
$$

\mbox{}

On peut aussi le voir en remarquant que chaque colonne de la matrice jacobienne s'obtient en dérivant par rapport à la même variable, or
$$
Ax = \sum_{i = 1}^n x_i a_i.
$$

\end{frame}

\begin{frame}
\frametitle{Points intérieurs}

En étendant le raisonnement précédent,
$$
J(F(x,y,s)) =
\begin{pmatrix}
	0 & A^T & I \\
	A & 0 & 0 \\
	S & 0 & X
\end{pmatrix}
$$

\mbox{}

Le système à résoudre pour trouver la direction de recherche est à présent
$$
\begin{pmatrix}
	0 & A^T & I \\
	A & 0 & 0 \\
	S & 0 & X
\end{pmatrix}
\begin{pmatrix}
	\bsd_x \\ \bsd_y \\ \bsd_s
\end{pmatrix}
=
-
\begin{pmatrix}
	A^Ty+s-c \\ Ax-b \\ XS\bsone - \mu\bsone
\end{pmatrix}
$$

\end{frame}

\begin{frame}
\frametitle{Direction de recherche}

En prenant $x$ et $(y,s)$ réalisables pour le problème primal et le problème dual, respectivement, le système se simplifie comme
$$
\begin{pmatrix}
0 & A^T & I \\
A & 0 & 0 \\
S & 0 & X
\end{pmatrix}
\begin{pmatrix}
\bsd_x \\ \bsd_y \\ \bsd_s
\end{pmatrix}
=
\begin{pmatrix}
0 \\ 0 \\ \mu\bsone - XS\bsone
\end{pmatrix}
$$

Le nouveau point $(x^+, y^+, s^+)$ est obtenu en se déplaçant le long de $(\bsd_x, \bsd_y, \bsd_s)$ d'une longueur de pas $\alpha$:
$$
(x^+, y^+, s^+) = (x, y, s) + \alpha (\bsd_x, \bsd_y, \bsd_s).
$$

\end{frame}

\begin{frame}
\frametitle{Squelette d'algorithme}

\begin{description}
	\item[Étape 0] Choisir $x_0 \in \mathring{\mathcal{F}}_P$, $(y_0, s_0) \in \mathring{\mathcal{F}}_D$, $\mu_0 > 0$, $\gamma \in (0,1)$, $\epsilon > 0$. Poser $k = 0$, $X_k = diag(x_k)$, $S_k = diag(s_k)$.
	\item[Étape 1] Si $\mu_k < \epsilon$, arrêt. Sinon, passer à l'étape 2.
	\item[Étape 2] Résoudre
$$
\begin{pmatrix}
	0 & A^T & I \\
	A & 0 & 0 \\
	S_k & 0 & X_k
\end{pmatrix}
\begin{pmatrix}
	\bsd_x \\ \bsd_y \\ \bsd_s
\end{pmatrix}
=
\begin{pmatrix}
	0 \\ 0 \\ \mu\bsone - X_kS_k\bsone
\end{pmatrix}
$$
\item[Étape 3] Calculer $\alpha_k$ tel que $x_k + \alpha_k\bsd_x > 0$, $s_k + \alpha_k\bsd_s > 0$ et permettant de s'approcher de la solution du système d'équations. Poser
\begin{align*}
& \mu_{k+1} = \gamma \mu_k,\quad x_{k+1} = x_k + \alpha_k\bsd_x, \\
& y_{k+1} = y_k + \alpha_k\bsd_y,\quad s_{k+1} = s_k + \alpha_k\bsd_s.
\end{align*}
Incrémenter $k$ de 1, et retourner à l'étape 1.
\end{description}

\end{frame}

%\begin{frame}
%\frametitle{Pas prédicteur}

%Nous avons
%$$
%(x^+)^Ts^+ = (1-\alpha) x^Ts 
%$$
%Le saut de dualité est dès lors réduit d'un facteur $(1-\alpha)$.

%\mbox{}

%Ce pas est souvent suivi d'une étape de correction pour rester dans un voisinage du chemin central.

%\mbox{}

%Différentes extensions existent, et sont couvertes notamment dans {\it Primal-Dual Interior-Point Methods}, Stephen J. Wright, SIAM, 1997 (\url{https://epubs.siam.org/doi/book/10.1137/1.9781611971453}).

%\end{frame}

\begin{frame}
	\frametitle{Calcul du centre analytique}

Nous avons besoin d'un point de départ réalisable!

\mbox{}
	
	Rappelons que le centre analytique peut se calculer en considérons les conditions
	\begin{align*}
		\bsx \circ \bss &= \bsone \\
		\bsA\bsx &= \bszero \\
		\bsA^T\bsy + \bss & = \bsc.
	\end{align*}

\mbox{}

De ce qui précède, le système de Newton à considérer pour résoudre ce système est
$$
\begin{pmatrix}
	0 & A^T & I \\
	A & 0 & 0 \\
	S & 0 & X
\end{pmatrix}
\begin{pmatrix}
	\Delta x \\ \Delta y \\ \Delta s
\end{pmatrix}
=
-
\begin{pmatrix}
	A^Ty+s-c \\ Ax \\ XS\bsone - \bsone
\end{pmatrix}
$$
	
\end{frame}

\begin{frame}
\frametitle{Calcul du centre analytique}

Nous pouvons réorganiser le système comme
$$
\begin{pmatrix}
	0 & 0 & A \\
	0 & X & S \\
	A^T & I & 0 \\
\end{pmatrix}
\begin{pmatrix}
	\Delta y \\ \Delta s \\ \Delta x
\end{pmatrix}
=
-
\begin{pmatrix}
	Ax \\ XS\bsone - \bsone \\ A^Ty+s-c
\end{pmatrix}
$$

\mbox {}

La contrainte $\bsx \circ \bss = \bsone$ peut se réécrire comme
$$
x-S^{-1}\bsone = 0
$$
Utilisant cette nouvelle expression, similairement à ce qui précède, nous pouvons construire le système de Newton
$$
\begin{pmatrix}
	0 & 0 & A \\
	0 & S^{-2} & I \\
	A^T & I & 0 \\
\end{pmatrix}
\begin{pmatrix}
	\Delta y \\ \Delta s \\ \Delta x
\end{pmatrix}
=
-
\begin{pmatrix}
	Ax \\ x-S^{-1}\bsone \\ A^Ty+s-c
\end{pmatrix}
$$

\end{frame}

\begin{frame}
	\frametitle{Calcul du centre analytique: pas de Newton}

En notant	
$$
r_d = s + A^Ty - c,\quad r_p = \begin{pmatrix} Ax \\ x-S^{-1}\bsone \end{pmatrix},\quad H = S^{-2}
$$
le système devient
	$$
	\begin{pmatrix}
		0 & 0 & A \\ 0 & H & I \\ A^T & I & 0
	\end{pmatrix}
\begin{pmatrix}
	\Delta y \\ \Delta s \\ \Delta x
\end{pmatrix}
	=
	-
	\begin{pmatrix}
		r_p \\ r_d
	\end{pmatrix},
	$$

\mbox{}

Après réorganisation des équations, nous pouvons obtenir
\begin{align*}
	\Delta y &= (AHA^T)^{-1} (Ax - AHr_d )\\
	\Delta s &= -A^T\Delta y - r_d \\
	\Delta x &= -H\Delta s - x + S^{-1}\bsone
\end{align*}
	
\end{frame}

\begin{frame}
	\frametitle{Calcul du centre analytique: pas de Newton}
	
	
	\mbox{}
	
	L'approche est expliquée en détails dans Goffin et Sharifi-Mokhtarian, ``Primal--Dual--Infeasible Newton Approach for the Analytic Center Deep-Cutting Plane Method'', Journal of Optimization Theory and Applications 101(1), 1999, pp. 35--58, disponible à l'adresse \url{https://link.springer.com/article/10.1023/A:1021714926231}.
	
\end{frame}

\begin{frame}
	\frametitle{Calcul du centre analytique: algorithme}
	
	Soient $y_0$, $\epsilon > 0$, $\alpha \in (0, 1/2)$, $\beta \in (0,1)$, $s_0 > 0$, $k = 0$, $k_{\max}$.
	
	$s_0$ peut être défini comme précédemment, et on prendra $\alpha = 0.01$, $\beta = 0.5$, $k_{\max} = 50$.
	
	Posons $x_0 = 0$, et calculons $r = (r_p, r_d)$.
	
	Répéter
	\begin{enumerate}
		\item Calculer le pas de Newton $(\Delta y, \Delta s, \Delta x)$.
		\item Déterminer une longueur de pas $t$.
		\item Mettre à jour la solution:
		$$
		\begin{pmatrix}
			y_{k+1} \\ s_{k+1} \\ x_{k+1}
		\end{pmatrix}
		= 
		\begin{pmatrix}
			y_{k} \\ s_{k} \\ x_{k} 
		\end{pmatrix}
		+ t
		\begin{pmatrix}
			\Delta y_{k} \\ \Delta s_{k} \\ \Delta x_{k} 
		\end{pmatrix}
		$$
		\item Poser $k :=  k+1$.
	\end{enumerate}
	Jusqu'à $s = c - A^Ty$ et $\| r(x,y,s) \|_2 \leq \epsilon$, ou $k > k_{\max}$.
	
\end{frame}

\begin{frame}
	\frametitle{Calcul du centre analytique: longueur du pas}
	
	Cela se fait par \textit{backtracking} (recherche arrière) sur $\| r \|_2$.
	
	\mbox{}
	
	Posons $t = 1$.
	
	\mbox{}
	
	Tant que des composantes de $s + t\Delta s$ sont négatives ou nulles, faire $t := \beta t$.
	
	\mbox{}
	
	Tant que
	$$
	\|r(x + t\Delta x, y + t\Delta y, s + t\Delta s) \|_2 > (1 - \alpha t) \| r(x,y,s) \|_2
	$$
	faire $t := \beta t$.
	
\end{frame}

\begin{frame}
\frametitle{Un exemple d'algorithme}

%Source: Sanjay Mehrotra, ``On the Implementation of a Primal-Dual Interior Point Method'', SIAM Journal on Optimization 2(4), pp. 575--601.

\mbox{}

\textit{Avec l'aide d'Arnaud L'Heureux}.

\mbox{}

La méthode nécessite un point de départ satisfaisant les contraintes de non-négativité, mais nous pouvons violer les autres contraintes.
	
\end{frame}

\begin{frame}
\frametitle{Point de départ}

Nous commençons par calculer
\begin{align*}
\overline{x} &= A^{T}(AA^{T})^{-1}b \\
\overline{y} &= (AA^{T})^{-1}Ac \\
\overline{s} &= c - A^T\overline{y}
\end{align*}

Ainsi,
$$
A\overline{x} = b
$$
et
$$
\overline{s} = (I-A^T(AA^{T})^{-1}A)c
$$

	
\end{frame}

\begin{frame}
\frametitle{Point de départ}

Calculons
\begin{align*}
\delta_x &= \max \{-1.5\min\{\overline{x}_i\}, 0 \} \\
\delta_s &= \max \{ -1.5\min\{\overline{s}_i\}, 0 \}
\end{align*}

Dès lors,
$$
	\overline{x} + \delta_x \bsone \geq 0, \quad \overline{s} + \delta_s \bsone \geq 0
$$
et le point est réalisable, mais pas à l'intérieur de $\mathcal{F}_P$ et $\mathcal{F}_D$.

Prenons
\begin{align*}
\overline{\delta_x} &= \delta_x + 0.5 \frac{(\overline{x}+\delta_x\bsone)^T(\overline{s}+\delta_s\bsone)}{\sum_{i=1}^n(\overline{s}_i+\delta_s)} \\
\overline{\delta_s} &= \delta_s + 0.5 \frac{(\overline{x}+\delta_x\bsone)^T(\overline{s}+\delta_s\bsone)}{\sum_{i=1}^n(\overline{x}_i+\delta_x)}
\end{align*}

\end{frame}

\begin{frame}
\frametitle{Point de départ}

Le point de départ de l'algorithme est alors calculé come
\begin{align*}
x_0 &= \overline{x} + \overline{\delta_x} \bsone \\
y_0 &= \overline{y} \\
s_0 &= \overline{s} + \overline{\delta_s} \bsone
\end{align*}

\mbox{}

Si $(\overline{x} + \delta_x \bsone)^T(\overline{s} + \delta_s \bsone) = 0$, la solution est optimale.
Sinon, $\overline{\delta_x} > 0$, $\overline{\delta_s} > 0$, et on part d'une solution intérieure.

\mbox{}

Les constantes $1.5$ et $0.5$ sont arbitraires. La même heuristique fonctionne en remplaçant $1.5$ par $k > 1$ et $0.5$ par $\ell > 0, \ell \in \mathbb{R}$.

\end{frame}

\begin{frame}
\frametitle{Point de départ: note}

Si $A \in \RR^{m \times n}$ et $rang(A) = m$, $AA^T \in \RR^{m \times m}$ et $rang(AA^T) = m$.

\mbox{}

En effet, le \textit{théorème du rang} nous dit que si $B \in \RR^{m \times n}$, alors
$$
rang(B) + \dim \left( \calN(B) \right) = n,
$$
où $\calN(A) = \{ x \,|\, Ax = 0 \}$ est le \textit{noyau} de $A$. Or $rang(A) = rang(A^T)$ et $\calN(AA^T) = \calN(A^T)$. Nous avons tout d'abord
$$
A^Tx = 0 \Rightarrow AA^Tx = 0,
$$
et donc $\calN(A^T) \subseteq \calN(AA^T)$. De plus, prenons $x \in \calN(AA^T)$ et notons $y = A^Tx \in \calN(A)$. Dès lors,
$$
0 = x^TAy = y^Ty \Rightarrow y = 0.
$$

\end{frame}

\begin{frame}
\frametitle{Point de départ: note}

Il en suit
$$
x \in \calN(AA^T) \Rightarrow x \in \calN(A^T),
$$
ou $\calN(AA^T) \subseteq \calN(A^T)$.

\mbox{}

Ainsi, $\calN(AA^T) = \calN(A^T)$ et
\begin{align*}
rang(AA^T) &= m - \dim\calN(AA^T) \\ &= m - \dim\calN(A^T) \\ &= m - (m - rang(A^T)) \\ & = rang(A^T).
\end{align*}

\end{frame}

\begin{frame}
\frametitle{Conditions d'optimalité}

Conditions d'optimalité du chemin primal-dual central:
\begin{align*}
Ax &= b \\
A^Ty + s &= c \\
XS\bsone &= \mu\bsone \\
x \geq 0,\ s & \geq 0
\end{align*}
où $X = diag(x)$, $S= diag(s)$.

\mbox{}

Soit $u = (x, y, s)$. Si $u$ n'est pas optimal, nous appliquons la méthode de Newton au système
$$
F_{\mu}(u) = \begin{pmatrix}
	A^Ty+s-c  \\
	Ax-b  \\
	XS\bsone - \mu\bsone 
\end{pmatrix} = 0, \ (x,s) \geq 0
$$
où $\mu > 0$ est le paramètre de barrière.

\end{frame}

\begin{frame}
\frametitle{Direction de Newton}

La direction $\Delta u$ est donnée par
$$
J_{\mu}(u)\Delta u = F_{\mu}(u)
$$
soit
$$
\begin{pmatrix}
	0 & A^T & I  \\
	A & 0 & 0\\
	S & 0 & X
\end{pmatrix}
\begin{pmatrix}
	\Delta x  \\
	\Delta y \\
	\Delta s
\end{pmatrix} = \begin{pmatrix}
	A^Ty+s-c  \\
	Ax-b  \\
	XS-\mu \bsone
\end{pmatrix}
=:
\begin{pmatrix}
	r_d  \\
	r_p \\
	r_c
\end{pmatrix},
$$
où $r_d$, $r_p$ et $r_c$ sont respectivement les résidus du dual, de primal et de complémentarité.
Multipliant la dernière ligne par la gauche par $S^{-1}$, nous avons
$$
\begin{pmatrix}
	0 & A^T & I  \\
	A & 0 & 0\\
	I & 0 & S^{-1}X
\end{pmatrix}
\begin{pmatrix}
	\Delta x  \\
	\Delta y \\
	\Delta s
\end{pmatrix} = \begin{pmatrix}
	r_d  \\
	r_p  \\
	S^{-1}r_c 
\end{pmatrix}.
$$

\end{frame}

\begin{frame}
\frametitle{Direction de Newton}

Notant $M = AXS^{-1}A^T$ et $t_d = r_p-AS^{-1}(r_c-Xr_d)$, nous pouvons montrer que le pas est
\begin{align*}
\Delta y &= M^{-1}t_d \\
\Delta s &= r_d - A^{T}\Delta y \\
\Delta x &= S^{-1}(r_c-X \Delta s)
\end{align*}
La solution est mise à jour avec
$$
\begin{pmatrix}
y^+ \\
s^+ \\
x^+
\end{pmatrix}
=
\begin{pmatrix}
	y \\
	s \\
	x
\end{pmatrix}
-
\alpha
\begin{pmatrix}
	\Delta y \\
	\Delta s \\
	\Delta x
\end{pmatrix}
$$
où $\alpha$ est choisi pour garantir $(x^+, s^+) \geq 0$.

\end{frame}

\begin{frame}
\frametitle{Terminaison}

$\mu$ est calculé suivant le saut de dualité à chaque itération:
$$
\mu = \frac{x^Ts}{n}.
$$

\mbox{}

L'algorithme s'arrête lors que $\max(\mu, \|r_p\|, \|r_d\|) < \epsilon$, avec $\epsilon > 0$ choisi à l'avance. Autrement dit, nous exigeons que la solution satisfasse, à une tolérance près, la faisabilité primale, duale, et les conditions de complémentarité.

\end{frame}

\begin{frame}
\frametitle{Algorithme de Mehrotra}

L'algorithme est divisé en deux étapes:
\begin{enumerate}
\item
étape prédictive,
\item
étape correctrice.
\end{enumerate}

\end{frame}

\begin{frame}
\frametitle{Étape prédictive}

L'étape prédictive consiste à calculer le pas de Newton classique, en résolvant
$$\begin{pmatrix}
	0 & A^T & I  \\
	A & 0 & 0\\
	I & 0 & S^{-1}X
\end{pmatrix}\begin{pmatrix}
	\Delta x_p  \\
	\Delta y_p \\
	\Delta s_p
\end{pmatrix} = \begin{pmatrix}
	r_d  \\
	r_p  \\
	S^{-1}r_c 
\end{pmatrix}$$
puis en limitant le longueur de pas pour satisfaire les contraintes de non-négativité.

\mbox{}

Calculons d'abord
\begin{align*}
\alpha_p^p &= \min \left\{1, \min_{\Delta (x_p)_i > 0} \frac{x_i}{\Delta (x_p)_i } \right\} \\
\alpha_d^p &= \min \left\{1, \min_{\Delta (s_p)_i > 0} \frac{s_i}{\Delta (s_p)_i } \right\}
\end{align*}

\end{frame}

\begin{frame}
\frametitle{Étape prédictive}

Le solution n'est pas nécessairement proche du chemin primal-dual.
Nous nous en rapprochons en calculant tout d'abord le paramètre
$$
\sigma = \left(\frac{(x - \alpha_p^p\Delta x_p)^T(s - \alpha_d^p\Delta s_p)}{n\mu}\right)^3,
$$
comparant la solution prédite avec le saut dualité théorique.

\mbox{}

Nous résolvons ensuite le système
$$\begin{pmatrix}
	0 & A^T & I  \\
	A & 0 & 0\\
	I & 0 & S^{-1}X
\end{pmatrix}\begin{pmatrix}
	\Delta x  \\
	\Delta y \\
	\Delta s
\end{pmatrix} = \begin{pmatrix}
	r_d  \\
	r_p  \\
	S^{-1}(Xs - \sigma \mu \bsone + \Delta x_p \Delta s_p) 
\end{pmatrix}$$

\end{frame}

\begin{frame}
\frametitle{Étape de correction}

Nous calculons finalement le pas primal et dual. Soit
\begin{align*}
\alpha_p &= \min \left\{1, \eta \min_{\Delta x_i>0} \frac{x_i}{\Delta x_i} \right\} \\
\alpha_d &= \min \left\{1, \eta \min_{\Delta s_i>0} \frac{s_i}{\Delta s_i} \right\}
\end{align*}
où $\eta = \max\{0.995, 1 - \mu\}$, le nouveau point est
\begin{align*}
x^+ &= x - \alpha_p \Delta x \\
y^+ &= y - \alpha_d \Delta y \\
s^+ &= s - \alpha_d \Delta s
\end{align*}

\end{frame}

\end{document}
