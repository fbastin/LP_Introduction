\documentclass[t, aspectratio=169,usepdftitle=false]{beamer}

\usepackage[utf8]{inputenc}
\usepackage[french]{babel}

\usetheme{Singapore}
\usepackage{xcolor}

% \setbeamercovered{transparent}
%\usecolortheme{crane}
\title[Introduction points intérieurs]{Méthode affine primale\\Introduction}
\author[Fabian Bastin]{Fabian Bastin\\DIRO\\Université de Montréal}
\date{}

\usepackage{ulem}
\usepackage{cancel}

\usepackage{amsmath}
\newtheorem{thm}{Théorème}

\setbeamertemplate{footline}[frame number]

\usepackage{tikz}
\newcommand*\circled[1]{\tikz[baseline=(char.base)]{
		\node[shape=circle,draw,inner sep=2pt] (char) {#1};}}

\include{macros}

\begin{document}
	\frame{\titlepage}

\begin{frame}
	\frametitle{Méthode affine primale}
	
	Section basée sur les notes de Gilles Savard: \url{http://www.iro.umontreal.ca/~marcotte/Ift6511/Pts_interieurs.pdf}.
	
	Voir aussi \url{https://nptel.ac.in/courses/106108056/module9/LinearProgramming_IV.pdf}
	
	\mbox{}
	
	\begin{itemize}
		\item 
		Présentée uniquement comme introduction aux méthodes de points intérieurs.
		\item
		Méthode simple et relative efficacité pratique.
		\item
		La complexité polynomiale n’est pas connue pour l'approche primale (ou duale).
		\item
		Analyse de convergence complexe.
		\item 
		Publié pour la première en 1967 par I. I. Dikin, mais ignoré jusqu'au milieu des années 1980.
	\end{itemize}
	
\end{frame}

\begin{frame}
	\frametitle{Principe}
	
	L'approche consiste à calculer une direction de descente (i.e. qui permet de réduire la valeur de l'objectif) qui n’approche pas trop rapidement
	de la frontière.
	%Cette direction de plus forte descente est calcul´ee `a partir d’un probl`eme mis `a l’´echelle qui centre la solution courante et qui est ensuite transform´ee dans l’espace original.
	
	\mbox{}
	
	Trois étapes:
	\begin{enumerate}
		\item 
		calcul de la direction de descente,
		\item
		calcul du pas,
		\item
		calcul de la transformation affine.
	\end{enumerate}
\end{frame}

\begin{frame}
	\frametitle{Intérieur de l'ensemble des contraintes}
	
	Considérons l'ensemble réalisable défini de manière général par des contraintes d'égalité et des contraintes d'inégalité:
	$$
	\calA = \left\{ x \,\bigg|\, \begin{cases}
		g_i(x) = 0,\ i = 1,\ldots,m \\
		h_j(x) \geq 0,\ j = 1,\ldots,n
	\end{cases}
	\right\}
	$$
	
	L'intérieur de $\calA$, noté notamment $int \calA$, est
	$$
	int \calA = \left\{ x \,\bigg|\, \begin{cases}
		g_i(x) = 0,\ i = 1,\ldots,m \\
		h_j(x) > 0,\ j = 1,\ldots,n
	\end{cases}
	\right\}
	$$
	
	Avec un problème linéaire sous forme standard, nous avons
	$$
	\calA = \{ x \,|\, Ax = b,\ x \geq 0 \}
	$$
	et
	$$
	int\calA = \{ x \,|\, Ax = b,\ x > 0 \}.
	$$
	
\end{frame}

\begin{frame}
	\frametitle{Formulation}
	
	Considérons à nouveau le primal
	\begin{align*}
		\min_x \ & c^Tx \\
		\mbox{t.q. } & Ax = b\\
		& x \geq 0,
	\end{align*}
	avec $A$ de rang plein. On suppose aussi que $\calA$ et $int\calA$ sont non vides.
	
\end{frame}

\begin{frame}
	\frametitle{Direction de descente}
	
	Soit $x_c$ le point courant t.q.
	$$
	Ax_c = b, \ x_c > 0.
	$$
	
	On cherche
	$$
	x^+ = x_c + \alpha \Delta x \quad \text{ t.q. }\quad 
	c^Tx^+ \leq c^T x_c,\ Ax^+ = b.
	$$
	Le déplacement doit donc vérifier
	\begin{align*}
		& c^T \Delta x \leq 0 \\
		& Ax^+ = A(x_c + \alpha \Delta x) = b
	\end{align*}
	
\end{frame}

\begin{frame}
	\frametitle{Direction de descente}
	
	Sous l'hypothèse que $\alpha > 0$, $\Delta x$ doit être dans le noyau de $A$, i.e.
	$$
	\Delta x \in \calN(A) = \{x \in \RR^n \,|\, Ax = 0 \}.
	$$
	La direction de plus forte descente est donnée par
	\begin{align*}
		\min_{\Delta x}\ & c^T\Delta x \\
		\mbox{t.q. } & A \Delta x = 0 \\
		& \| \Delta x \| = 1.
	\end{align*}
	La solution est
	$$
	\frac{\text{proj}_A(-c)}{\| \text{proj}_A (-c) \|} = \Delta x,
	$$
	$\text{proj}_A(\cdot)$ étant la matrice orthogonale de projection sur le
	noyau de $A$.
	
\end{frame}

\begin{frame}
	\frametitle{Direction de descente}
	
	Comme $A$ est de plein rang et en oubliant la normalisation, il est possible de montrer que
	$$
	\Delta x = \text{proj}_A(-c) = -(I - A^T(AA^T)^{-1}A)c.
	$$
	
	\mbox{}
	
	Note: une matrice orthogonale de projection $P$ sur le noyau de $A$ vérifie les propriétés suivantes:
	\begin{align*}
		AP &= 0 \\
		P &= P^T \\
		P^2 &= P\qquad \text{(matrice idempotente)}.
	\end{align*}
	Notons $P = I - A^T(AA^T)^{-1}A$. Alors
	$$
	c^T \Delta x = -c^T P c = -c^T P^2 c = - \| Pc \|^2_2 \leq 0.
	$$
	
\end{frame}

\begin{frame}
	\frametitle{Longueur de pas}
	
	Le taux de décroissance est constant dans la direction $\Delta x$.
	Le pas maximal est limité par les contraintes de non-négativité.
	De plus, la transformation affine qui nous permettra de centrer le point n'est pas définie sur la frontière.
	
	\mbox{}
	
	Il suffit de choisir
	$$
	\alpha = \gamma\alpha_{\max},
	$$
	avec $0 < \gamma < 1$, et
	$$
	\alpha_{\max} = \min_{\Delta x_i < 0} \frac{-(x_c)_i}{\Delta x_i}.
	$$
	En pratique, on choisit $\gamma  = 0.995$.
	Si $\Delta x \geq 0$ et $\Delta x \ne 0$, le problème est non borné.
	
\end{frame}

\begin{frame}
	\frametitle{Transformation affine}
	
	\begin{itemize}
		\item 
		Il s'agit de faire une mise à l'échelle afin que le nouveau point $x^+$ soit loin de la frontière définie par $x \geq 0$.
		\item 
		Un point idéal serait le vecteur unitaire $e$. Ainsi, on cherche une transformation
		affine qui transforme $x^+$ en $e$.
		La matrice de transformation est simplement l'inverse de la matrice diagonale dont les
		composantes sont les mêmes que celles de $x^+$ (qui sont $>0$).
		Notons cette matrice par $X$.
		On a alors $X^{-1}x^+ = e$, et notons
		$$
		X^{-1}x = \overline{x}.
		$$
		\item 
		Dans l'espace transformé, le programme devient
		\begin{align*}
			\min_{\overline{x}} \ & c^TX\overline{x} = \overline{c}^T\overline{x} \\
			\mbox{t.q. } & AX\overline{x} = \overline{A}\overline{x} = b\\
			& \overline{x} \geq 0.
		\end{align*}
	\end{itemize}
	
\end{frame}

\begin{frame}
	\frametitle{Calcul du nouveau point}
	
	Dans l'espace-$\overline{x}$, on obtient % et étant donné $\overline{x}_c e$, on obtient
	\begin{align*}
		\Delta\overline{x} &= \text{proj}_{\overline{A}}(-\overline{c}) \\
		& = -(I - \overline{A}^T(\overline{A}\overline{A}^T)^{-1}\overline{A})\overline{c} \\
		& = -(I - XA^T(AX^2A^T)^{-1}AX)Xc
	\end{align*}
	et
	$$
	\overline{x}^+ = \overline{x}_c + \alpha \Delta \overline{x},
	$$
	d'où
	\begin{align*}
		x^+ &= X\overline{x}^+ \\
		&= x_c + \alpha X \Delta \overline{x} \\
		&= x_c - \alpha X(I - XA^T(AX^2A^T)^{-1}AX)Xc.
	\end{align*}
	
\end{frame}

\begin{frame}
	\frametitle{Convergence}
	
	Supposons $A$ plein rang, qu'il existe un point strictement intérieur et que la fonction objectif est non constante sur le domaine réalisable.
	Alors
	\begin{enumerate}
		\item
		Si le problème primal et le problème dual sont non dégénérés, alors pour tout $\gamma < 1$, la suite générée par l'algorithme converge vers une solution optimale.
		\item
		Pour $\gamma \leq 2/3$, la suite produite par l’algorithme converge
		vers une solution optimale (y compris en présence de dégénérescence).
	\end{enumerate}
	
	Preuve: admis!
	
\end{frame}

\begin{frame}
	\frametitle{Critère d’arrêt}
	
	Basé sur la satisfaction des conditions d'optimalité.
	
	\mbox{}
	
	Considérons d’abord le cas non dégénéré (primal et dual).
	
	\mbox{}
	
	Programme dual:
	\begin{align*}
		\max_{y,s} \ & b^Ty \\
		\mbox{t.q. } & A^Ty + s = c \\
		& s \geq 0.
	\end{align*}
	Soit $x_k$ la solution courante et considérons le programme
	\begin{align*}
		\min_{y,s} \ & \| X_ks \| \\
		\mbox{t.q. } & A^Ty + s = c \\
		& s \geq 0.
	\end{align*}
	
\end{frame}

\begin{frame}
	\frametitle{Critère d’arrêt}
	
	Problème non linéaire!
	
	\mbox{}
	
	Il est possible de montrer que la solution est donnée par
	\begin{align*}
		y_k &= (AX^2_kA^T)^{-1}AX^2_kc \\
		s_k &= c - A^Tyk
	\end{align*}
	Dès lors
	$$
	s_k = c - A^T(AX^2_kA^T)^{-1}AX^2_kc = (I - A^T(AX^2_kA^T)^{-1}AX^2_k)c
	$$
	et
	\begin{align*}
		-X^2_ks_k &= -X^2_k(I - A^T(AX^2_kA^T)^{-1}AX^2_k)c \\
		&= -X_k(I - X_kA^T(AX^2_kA^T)^{-1}AX_k)X_kc \\
		&= \Delta x_k.
	\end{align*}
	Le vecteur dual est obtenu comme sous-produit du calcul
	de la direction de descente.
	
\end{frame}

\begin{frame}
	\frametitle{Convergence}
	
	Sous l'hypothèse de non dégénérescence primale et duale, la solution $(y_k, s_k)$ converge vers la solution optimale duale lorsque $x_k$ converge vers la solution optimale primale.
	
	\mbox{}
	
	Dans ce cas, un critère d'arrêt peut être défini sur la norme
	du vecteur de complémentarité.
	
	\mbox{}
	
	Dans le cas dégénéré, la convergence de $(y_k, s_k)$ n'est pas assurée, et un critère d'arrêt classique sur l'amélioration successive de deux itérés est utilisé.
	
\end{frame}

\begin{frame}
	\frametitle{Algorithme}
	
	Soit $\gamma \in (0,1)$, $\epsilon > 0$ et un point initial $x_0$.
	Poser
	\begin{align*}
		x_c &:= x_0 \\
		\Delta c &:= \epsilon|c^Tx_c|+2
	\end{align*}
	Tant que $\Delta c > \epsilon\max\{|c^Tx_c|,1\}$ répéter
	\begin{align*}
		X &:= X_c \\
		\Delta x &:= -X(I - XA^T(AX^2A^T)^{-1}AX)Xc \\
		\alpha &:= \gamma \min_{\Delta x_i < 0} \frac{-(x_c)_i}{\Delta x_i} \\
		x^+ &:= x_c + \alpha \Delta x \\
		% \alpha &:= \gamma \min_{\Delta x_i < 0} \frac{-1}{\Delta x_i} \\
		% x^+ &:= x_c + \alpha X \Delta x \\
		\Delta c &:= c^Tx_c - c^Tx^+ \\
		x_c &:= x^+.
	\end{align*}
	
\end{frame}

\begin{frame}
	\frametitle{Algorithme}
	
	On peut légèrement simplifier les opérations comme suit.
	
	Soit $\gamma \in (0,1)$, $\epsilon > 0$ et un point initial $x_0$.
	Poser
	\begin{align*}
		x_c &:= x_0 \\
		\Delta c &:= \epsilon|c^Tx_c|+2
	\end{align*}
	Tant que $\Delta c > \epsilon\max\{|c^Tx_c|,1\}$ répéter
	\begin{align*}
		X &:= X_c \\
		\Delta x &:= -(I - XA^T(AX^2A^T)^{-1}AX)Xc \\
		%\alpha &:= \gamma \min_{\Delta x_i < 0} \frac{-(x_c)_i}{\Delta x_i} \\
		%x^+ &:= x_c + \alpha \Delta x \\
		\alpha &:= \gamma \min_{\Delta x_i < 0} \frac{-1}{\Delta x_i} \\
		x^+ &:= x_c + \alpha X \Delta x \\
		\Delta c &:= c^Tx_c - c^Tx^+ \\
		x_c &:= x^+.
	\end{align*}
	
\end{frame}

\begin{frame}
	\frametitle{Initialisation}
	
	Une méthode de phase I peut également être développée pour trouver un point réalisable sans nécessairement qu'il soit solution de base. Voir Vanderbei, Chapitre 21, Section 5.
\end{frame}

\end{document}
