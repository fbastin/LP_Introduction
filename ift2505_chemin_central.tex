\documentclass[t,usepdftitle=false]{beamer}
\usepackage[utf8]{inputenc}
\usetheme{Warsaw}
\usepackage{xcolor}
% \setbeamercovered{transparent}
%\usecolortheme{crane}
\title[IFT2505]{IFT 2505\\Programmation Linéaire}
\author[Fabian Bastin]{Fabian Bastin\\DIRO\\Université de Montréal\\\mbox{}\\\url{http://www.iro.umontreal.ca/~bastin/ift2505.php}}
\date{Automne 2013}

\usepackage{ulem}
\usepackage{cancel}

\usepackage{tikz}
\newcommand*\circled[1]{\tikz[baseline=(char.base)]{
    \node[shape=circle,draw,inner sep=2pt] (char) {#1};}}

\def\ba{\boldsymbol{a}}
\def\bb{\boldsymbol{b}}
\def\bc{\boldsymbol{c}}
\def\be{\boldsymbol{e}}
\def\br{\boldsymbol{r}}
\def\bs{\boldsymbol{s}}
\def\bu{\boldsymbol{u}}
\def\bx{\boldsymbol{x}}
\def\by{\boldsymbol{y}}
\def\bz{\boldsymbol{z}}
\def\bA{\boldsymbol{A}}
\def\bB{\boldsymbol{B}}
\def\bD{\boldsymbol{D}}
\def\bH{\boldsymbol{H}}
\def\bI{\boldsymbol{I}}
\def\bL{\boldsymbol{L}}
\def\bM{\boldsymbol{M}}
\def\bU{\boldsymbol{U}}
\def\bzero{\boldsymbol{0}}
\def\bone{\boldsymbol{1}}
\def\blambda{\boldsymbol{\lambda}}

\def\cR{\mathcal{R}}

\usepackage[frenchb]{babel}

\begin{document}
\frame{\titlepage}

% ------------------------------------------------------------------------------------------------------------------------------------------------------\begin{frame}


\begin{frame}
\frametitle{Chemin central}

Aspects similaires à la programmation linéaire:
\begin{itemize}
\item
algorithmes itératifs
\item
pour un itéré donné, calcul d'une direction de recherche, puis d'une longueur de pas le long de cette direction.
\end{itemize}

\mbox{}

Considérons le programme
\begin{align*}
\min_x \ & c^Tx \\
\mbox{s.à. } & Ax = b\\
& x \geq 0.
\end{align*}

\mbox{}

Définissons les ensembles
\begin{align*}
\mathcal{F}_P \overset{\mbox{def}}{=} \lbrace x \,|\, Ax = b, x \geq 0 \rbrace\\
\mathring{\mathcal{F}}_P \overset{\mbox{def}}{=} \lbrace x \,|\, Ax = b, x > 0 \rbrace\\
\end{align*}

\end{frame}

\begin{frame}
\frametitle{Prbolème barrière}

On suppose que $\mathring{\mathcal{F}}_P$ est non vide et que l'ensemble de solutions optimales pour ce problème est borné.

\mbox{}

Soit $\mu \geq 0$. Problème barrière (PB):
\begin{align*}
\min_x \ & c^Tx - \mu \sum_{j = 1}^n \log x_j \\
\mbox{s.à. } & Ax = b \\
& x > 0.
\end{align*}

\end{frame}

\begin{frame}
\frametitle{Prbolème barrière vs Problème linéaire}

Si $\mu = 0$, on retrouve le problème original en permettant à $x$ d'avoir des composantes nulles:
\begin{align*}
\min_x \ & c^Tx \cancel{- \mu \sum_{j = 1}^n \log x_j} \\
\mbox{s.à. } & Ax = b \\
& x \boldsymbol{\geq} 0.
\end{align*}

\end{frame}

\begin{frame}
\frametitle{Chemin central}

Soit $x(\mu)$, la solution au PB étant donné $\mu$.
En faisant varier $\mu$ continûment vers 0, nous obtenons le chemin central primal.

\mbox{}

Si $\mu \rightarrow \infty$, la solution s'approche du centre analytique de la région réalisable, lorsque celle-ci est bornée: le terme barrière prédomine alors dans l'objectif.

\mbox{}

Comme $\mu \rightarrow 0$, ce chemin converge vers le centre analytique de la face optimale $\lbrace x \,|\, c^Tx = z^*,\ Ax = b,\ x \geq 0 \rbrace$, où $z^*$ est la valeur optimale du programme linéaire.

\mbox{}

L'idée de base est de résoudre une succession de problèmes barrières pour des valeurs décroissantes de $\mu$.

\end{frame}

\begin{frame}
\frametitle{Fonction lagrangienne}

Réécrivons la contrainte $Ax = b$ sous la forme $Ax - b = 0$, et introduisons un vecteur $y$, en associant une composante $y_i$ à une contrainte $\sum_{j = 1}^n a_{ij} x_j = b_j$.

\mbox{}

$y_i$ joue à peu près le même rôle que les variables duales précédentes, mais est appelé à présent multiplicateur de Lagrange.

\mbox

Lagrangien:
\[
L(x) = c^Tx - \mu \sum_{j = 1}^n \log x_j - y^T(Ax - b).
\]
On cherche à minimiser cette fonction en annulant son gradient.

\end{frame}

\begin{frame}
\frametitle{Minimisation du lagrangien}

\begin{align*}
& \nabla_x L(x) = 0 \\
\Leftrightarrow\ &
 c_j - \frac{\mu}{x_j} - y^Ta_j = 0,\ j = 1,\ldots,n\\
\Leftrightarrow\ &
 \mu X^{-1}\bone + A^Ty = c.
\end{align*}

En notant $s_j = \mu/x_j$, l'ensemble des conditions d'optimalité (primales, duales, minimisation du lagrangien) s'écrit:
\begin{align*}
\bx \circ \bs &= \mu \bone \\
\bA\bx &= \bb \\
\bA^T\by + \bs & = \bc.
\end{align*}

\end{frame}

\begin{frame}
\frametitle{Lien avec le dual}

$y$ est une solution duale réalisable et $c - A^T y > 0$.

\mbox{}

En effet, le dual est
\begin{align*}
\max_y\ \ & y^T b \\
\mbox{s.à. } & A^T y \leq c
\end{align*}
Comme
\begin{align*}
A^Ty + s &= c\\
s_j &= \frac{\mu}{x_j}
\end{align*}
nous avons
\[
A^T y < c.
\]

\end{frame}

\begin{frame}
\frametitle{Exemple}

Considérons le programme
\begin{align*}
\max\ & x_1 \\
\mbox{s.à. } & 0 \leq x_1 \leq 1 \\
& 0 \leq x_2 \leq 1.
\end{align*}

\mbox{}

On réécrit ce système sous forme standard
\begin{align*}
\min\ & -x_1 \\
\mbox{s.à. } & x_1 + x_3 = 1\\
& x_2 + x_4 = 1 \\
& x_1 \geq 0, x_2 \geq 0, x_3 \geq 0, x_4 \geq 0.
\end{align*}

\end{frame}

\begin{frame}
\frametitle{Conditions d'optimalité}

\begin{align*}
x_1 s_1 &= \mu \\
x_2 s_2 &= \mu \\
x_3 s_3 &= \mu \\
x_4 s_4 &= \mu \\
x_1 + x_3 &= 1 \\
x_2 + x_4 &= 1 \\
y_1 + s_1 &= -1 \\
y_2 + s_2 &= 0 \\
y_1 + s_3 &= 0 \\
y_2 + s_4 &= 0
\end{align*}

\end{frame}

\begin{frame}
\frametitle{Conditions d'optimalité}

De
\begin{align*}
y_2 + s_2 = 0 \\
y_2 + s_4 = 0 \\
\end{align*}
on a $s_2 = s_4$.

\mbox{}

On en déduit aussi
\[
x_2 = x_4,
\]
et de là
\[
x_2 = x_4 = \frac{1}{2}.
\]

\end{frame}

\begin{frame}
\frametitle{Conditions d'optimalité}

On a aussi
\begin{align*}
& -1 = s_1 - s_3 = \frac{\mu}{x_1} - \frac{\mu}{x_3} \\
\Leftrightarrow\ & -1 = \frac{\mu}{x_1} - \frac{\mu}{1-x_1} \\
\Leftrightarrow\ & -x_1(1-x_1) = \mu(1-x_1) - \mu x_1 \\
\Leftrightarrow\ & x_1^2 - x_1 = \mu - 2\mu x_1 \\
\Leftrightarrow\ & x_1^2 - (1-2\mu) x_1 -\mu = 0.
\end{align*}
Le discriminant de cette équation quadratique est
\[
\rho = (1-2\mu)^2 + 4\mu = 1+ 4\mu^2,
\]
et de là
\[
x_1 = \frac{1-2\mu \pm \sqrt{1+4\mu^2}}{2}
\]

\end{frame}

\begin{frame}
\frametitle{Chemin central}

Pour $\mu$ grand, on doit choisir la racine correspondant à '+'.

D'autre part,
\[
x \rightarrow \left( 1, \frac{1}{2} \right), \mbox{ comme } \mu \rightarrow 0.
\]

\mbox{}

On converge vers le centre analytique de la face optimale
\[
\lbrace x \,|\, x_1 = 1,\ 0 \leq x_2 \leq 1\rbrace
\]
plutôt qu'un coin du carré.

\mbox{}

De plus, quand $\mu \rightarrow +\infty$,
\[
x(\mu) \rightarrow \left( \frac{1}{2}, \frac{1}{2} \right)
\]
comme
\[
-2\mu + \sqrt{1+4\mu^2} = -2\mu + 2\mu\sqrt{\frac{1}{4\mu^2}+1}.
\]

\end{frame}

\begin{frame}
\frametitle{Chemin central}

Dès lors, le chemin central est une ligne droite progressant du centre analytique du carré (comme $\mu \rightarrow \infty$) vers le centre analytique de la face optimale (comme $\mu \rightarrow 0$).

\end{frame}

\begin{frame}
\frametitle{Chemin dual central}

Partons à présent du problème dual
\begin{align*}
\max\ & y^Tb \\
\mbox{s.à. } & y^TA + s^T = c^T \\
& s \geq 0.
\end{align*}

\mbox{}

Le problème barrière associé est
\begin{align*}
\max\ & y^Tb + \mu \sum^n_{j = 1} \log s_j \\
\mbox{s.à. } & y^TA + s^T = c^T \\
& s \geq 0.
\end{align*}

\end{frame}

\begin{frame}
\frametitle{Chemin dual central}

On suppose que $\mathring{\mathcal{F}}_P = \lbrace (y,s): y^TA + s^T = c^T,\ s > 0 \rbrace$ est non vide, et l'ensemble des solutions optimales du dual est borné.

\mbox{}

On obtient le chemin central dual en faisant tendre $\mu$ vers 0.

\mbox{}

Lagrangien:
\[
L(y) = y^Tb + \mu \sum_{j=1}^n \log s_j - (y^TA + s^T - c^T)x.
\]

\mbox{}

Dès lors,
\begin{align*}
\nabla_y L &= 0 \Leftrightarrow b_i - a^i x = 0,\ \forall i ,\\ 
\nabla_s L &= 0 \Leftrightarrow \frac{\mu}{s_j} - x_j = 0,\ \forall j.
\end{align*}

\end{frame}

\begin{frame}
\frametitle{Chemin dual central: conditions d'optimlaité}

On obtient les conditions d'optimalité
\begin{align*}
\bx \circ \bs &= \mu \bone \\
\bA\bx &= \bb \\
\bA^T\by + \bs & = \bc.
\end{align*}

\mbox{}

On retrouve les mêmes conditions que pour le chemin central.

\mbox{}

Par conséquent, $x$ est une solution réalisable primale et $x > 0$.

\mbox{}

Considérons l'ensemble de niveau dual
\[
\Omega(z) = \lbrace y \,|\, c^T - y^T A \geq 0, y^T b = z\rbrace,
\]
avec $z < z^*$, la valeur optimale du dual.

\end{frame}

\begin{frame}
\frametitle{Chemin dual central: conditions d'optimlaité}

Le centre analytique $(y(z), s(z))$ de $\Omega$ coïncide avec le chemin central dual comme $z$ tend vers $z^*$.

%voir p.52 des notes manuscrites

\end{frame}

\begin{frame}
\frametitle{Chemin dual central: exemple}

Reprenons le problème
\begin{align*}
\min\ & -x_1 \\
\mbox{s.à. } & x_1 + x_3 = 1 \\
& x_2 + x_4 = 1 \\
& x_1 \geq 0, x_2 \geq 0, x_3 \geq 0, x_4 \geq 0.
\end{align*}

Le dual s'écrit
\begin{align*}
\max\ & y_1 + y_2 \\
\mbox{s.à. } & y_1 \leq -1 \\
& y_2 \leq 0 \\
& y_1 \leq 0 \\
& y_2 \leq 0.
\end{align*}

\end{frame}

\begin{frame}
\frametitle{Chemin dual central: exemple}

On peut réécrire le dual comme
\begin{align*}
\max\ & y_1 + y_2 \\
\mbox{s.à. } & y_1 + s_1 -1 \\
& y_2 + s_2 = 0.
\end{align*}

\mbox{}

Les conditions d'optimalité sont les mêmes que pour le primal, aussi
\[
x_2 = x_4 = \frac{1}{2},
\]
d'où
\begin{align*}
s_2 = s_4 = 2\mu,
y_2 = -2\mu.
\end{align*}

\end{frame}

\begin{frame}
\frametitle{Chemin dual central: exemple}

Nous avons également, en se rappelant de la résolution du primal,
\begin{align*}
y_1 & = -1 - s_1 \\
& = -1 - \frac{\mu}{x_1(\mu)} \\
& = -1 - \frac{2\mu}{1-2\mu\pm\sqrt{1+4\mu^2}}
\end{align*}

\mbox{}

Comme $\mu \rightarrow 0$, $y_1 \rightarrow -1$, $y_2 \rightarrow 0$.

\mbox{}

Il s'agit de l'unique solution du problème linéaire (les deux contraintes sont actives)

\mbox{}

Si $\mu \rightarrow \infty$, $y$ est non borné, car l'ensemble dual réalisable est non borné.

\end{frame}

\begin{frame}
\frametitle{Chemin central primal-dual}

Hypothèse: la région réalisable du problème de programmation linéaire a un intérieur non vide et un ensemble borné de solutions optimales.

\mbox{}

Le dual a un intérieur réalisable non vide, en vertu des conditions (d'optimalité) sur le lagrangien.

\mbox{}

Le chemin primal-dual est l'ensemble des vecteurs $(x(\mu), y(\mu), s(\mu))$ satisfaisant les conditions
\begin{align*}
\bx \circ \bs &= \mu \bone \\
\bA\bx &= \bb \\
\bA^T\by + \bs & = \bc \\
x \geq 0,\ & s \geq 0 \\
0 \leq \mu & < \infty.
\end{align*}

\end{frame}

\begin{frame}
\frametitle{Proposition}

Si les ensembles réalisables primal et dual ont des intérieurs non vides, alors le chemin central $(x(\mu), y(\mu), s(\mu))$ existe pour tout $\mu$, $0 \leq \mu < \infty$.

\mbox{}

De plus, $x(\mu)$ est le chemin central primal, $(y(\mu), s(\mu))$ est le chemin central dual.

\mbox{}

$x(\mu)$ et $(y(\mu), s(\mu))$ convergent vers les centres analytiques des faces des solutions optimales primales et duales, respectivement, quand $\mu \rightarrow 0$.

\end{frame}

\begin{frame}
\frametitle{Saut de dualité}

Soit $(x(\mu), y(\mu), s(\mu))$ sur le chemin central primal-dual.
Nous avons
\begin{align*}
c^Tx - y^Tb &= (A^T y)^T x + s^Tx - y^T b \\
& = y^T Ax + s^Tx - y^T b \\
& = y^T b + s^Tx - y^T b \\
& = s^Tx \\
& = n\mu.
\end{align*}

\mbox{}

Comme pour la dualité faible, $c^T x \geq y^T b$, et $n\mu$ est appelé le saut de dualité.

\mbox{}

Soit $g = c^Tx - y^Tb$.

\mbox{}

Comme $y^T b \leq z^*$, $z^* \geq c^Tx - g$, et donc, étant donné $(x, y, s)$, on peut mesurer la qualité de $x$ comme $c^T x - z^* \leq g$.
\end{frame}

\end{document}
