\documentclass[t,usepdftitle=false]{beamer}
\usepackage[utf8]{inputenc}
\usetheme{Warsaw}
\usepackage{xcolor}
% \setbeamercovered{transparent}
%\usecolortheme{crane}
\title[IFT2505]{IFT 2505\\Programmation Linéaire}
\author[Fabian Bastin]{Fabian Bastin\\DIRO\\Université de Montréal\\\mbox{}\\\url{http://www.iro.umontreal.ca/~bastin/ift2505.php}}
\date{Automne 2012}

\usepackage{ulem}
\usepackage{cancel}

\usepackage{tikz}
\newcommand*\circled[1]{\tikz[baseline=(char.base)]{
    \node[shape=circle,draw,inner sep=2pt] (char) {#1};}}

\def\ba{\boldsymbol{a}}
\def\bb{\boldsymbol{b}}
\def\bc{\boldsymbol{c}}
\def\be{\boldsymbol{e}}
\def\br{\boldsymbol{r}}
\def\bs{\boldsymbol{s}}
\def\bu{\boldsymbol{u}}
\def\bx{\boldsymbol{x}}
\def\by{\boldsymbol{y}}
\def\bz{\boldsymbol{z}}
\def\bA{\boldsymbol{A}}
\def\bB{\boldsymbol{B}}
\def\bD{\boldsymbol{D}}
\def\bH{\boldsymbol{H}}
\def\bI{\boldsymbol{I}}
\def\bL{\boldsymbol{L}}
\def\bM{\boldsymbol{M}}
\def\bU{\boldsymbol{U}}
\def\bzero{\boldsymbol{0}}
\def\bone{\boldsymbol{1}}
\def\blambda{\boldsymbol{\lambda}}

\def\cR{\mathcal{R}}

\usepackage[frenchb]{babel}

\begin{document}
\frame{\titlepage}

% ------------------------------------------------------------------------------------------------------------------------------------------------------\begin{frame}

\begin{frame}
\frametitle{Exemple sur le simplexe dual et primal-dual}

On considère le problème
\begin{align*}
\min_{x}\ & 3x_1 + 4x_2 + 6x_3 + 7x_4 + x_5 \\
\mbox{s.à. } & 2x_1 - x_2 + x_3 +6x_4 - 5x_5 \geq 6 \\
& x_1 + x_2 + 2x_3 + x_4 + 2x_5 \geq 3 \\
& x_1, x_2, x_3, x_4, x_5 \geq 0.
\end{align*}
Le dual est
\begin{align*}
\max_{\lambda}\ & 6\lambda_1 + 3\lambda_2 \\
& 2\lambda_1 + \lambda_2 \leq 3 \\
& -\lambda_1 + \lambda_2 \leq 4 \\
& \lambda_1 + 2\lambda_2 \leq 6 \\
& 6\lambda_1 + \lambda_2 \leq 7 \\
& -5\lambda_1 + 2\lambda_2 \leq 1 \\
& \lambda_1, \lambda_2 \in \cR.
\end{align*}

\end{frame}

\begin{frame}
\frametitle{Admissibilité du dual}

Comme tous les coefficients dans la fonction objectif sont strictement positifs, une solution réalisable pour le dual est
\[
\blambda = (0,0).
\]

\mbox{}

Nous savons de plus que l'admissibilité de $\blambda$ est équivalente à l'existence d'une solution primale satisfaisant les conditions d'optimalite en termes de coûts réduits, mais violant possiblement les contraintes de non-négativités.

\mbox{}

Essayons de démarrer le simplexe.

\end{frame}

\begin{frame}
\frametitle{Forme standard}

\begin{align*}
\min_{x}\ & 3x_1 + 4x_2 + 6x_3 + 7x_4 + x_5 \\
\mbox{s.à. } & 2x_1 - x_2 + x_3 +6x_4 - 5x_5 - x_6 = 6 \\
& x_1 + x_2 + 2x_3 + x_4 + 2x_5 - x_7 = 3 \\
& x_1, x_2, x_3, x_4, x_5, x_6, x_7 \geq 0.
\end{align*}

Nous n'avons pas la forme canonique, mais nous pouvons facilement l'obtenir en multipliant les contraintes par -1:
\begin{align*}
\min_{x}\ & 3x_1 + 4x_2 + 6x_3 + 7x_4 + x_5 \\
\mbox{s.à. } & -2x_1 + x_2 - x_3 - 6x_4 + 5x_5 + x_6 = -6 \\
& -x_1 - x_2 - 2x_3 - x_4 - 2x_5 + x_7 = -3 \\
& x_1, x_2, x_3, x_4, x_5, x_6, x_7 \geq 0.
\end{align*}

\end{frame}

\begin{frame}
\frametitle{Forme standard}

Problème: la base associée à $x_6$, $x_7$ n'est pas réalisable vu que les termes non-nuls de la solution de base associée, à savoir $x_6 = -6$, $x_7 = -3$, violent les contraintes de non-négativité.

\mbox{}

Le dual, lui, devient
\begin{align*}
\max_{\lambda}\ & -6\lambda_1 -3\lambda_2 \\
& -2\lambda_1 -\lambda_2 \leq 3 \\
& \lambda_1 -\lambda_2 \leq 4 \\
& -\lambda_1 -2\lambda_2 \leq 6 \\
& -6\lambda_1 -\lambda_2 \leq 7 \\
& 5\lambda_1 -2\lambda_2 \leq 1 \\
& \lambda_1, \lambda_2 \in \cR.
\end{align*}

\end{frame}

\begin{frame}
\frametitle{Forme standard: dual}

$(\lambda_1, \lambda_2) = (0,0)$ reste réalisable, et donc nous savons qu'il existe un $x$ pour lequel les conditions d'optimalités sur les coûts réduits sont satisfaits.

\mbox{}

On peut en déduire la motivation du simplexe duale:
\begin{itemize}
\item
conserver les conditions d'optimalité et la réalisabilité duale;
\item
forcer la realisabilité primale (et partant de là, trouver une solution primale optimale).
\end{itemize}
\end{frame}

\begin{frame}
\frametitle{Simplexe dual}

Sous forme de tableau, cela donne
\[
\begin{matrix}
x_6 & -2 & 1 & -1 & -6 & 5 & 1 & 0 & -6 \\
x_7 & -1 & -1 & -2 & -1 & -2 & 0 & 1 & -3 \\
& 3 & 4 & 6 & 7 & 1 & 0 & 0 & 0
\end{matrix}
\]

On doit décider d'une variable qui va quitter la base, car de mauvais signe. Il n'y a pas de règle de sélection ici; on se contente de sélectionner une ligne avec un terme de droite strictement négatif.

\mbox{}

Supposons que nous choisissions $x_7$.

\mbox{}

On calcule les rapports entre la dernière ligne et la ligne de $x_7$, en se limitant aux entrées strictement négatives. On sélectionne le minimum en valeur absolue de ces rapports.
\end{frame}

\begin{frame}
\frametitle{Simplexe dual}

De ce qui précède, le pivot est sur $x_5$:
\[
\begin{matrix}
x_6 & -2 & 1 & -1 & -6 & 5 & 1 & 0 & -6 \\
x_7 & -1 & -1 & -2 & -1 & \circled{-2} & 0 & 1 & -3 \\
& 3 & 4 & 6 & 7 & 1 & 0 & 0 & 0
\end{matrix}
\]
conduisant au nouveau tableau
\[
\begin{matrix}
x_6 & -\frac{9}{2} & -\frac{3}{2} & -6 & -\frac{17}{2} & 0 & 1 & \frac{5}{2} & -\frac{27}{2} \\
x_5 & \frac{1}{2} & \frac{1}{2} & 1 & \frac{1}{2} & 1 & 0 & -\frac{1}{2} & \frac{3}{2} \\
& \frac{5}{2} & \frac{7}{2} & 5 & \frac{13}{2} & 0 & 0 & \frac{1}{2} & -\frac{3}{2}
\end{matrix}
\]

Il faut à présent faire sortir $x_6$. Les rapports à considérer sont $\frac{5}{9}$, $\frac{7}{3}$, $\frac{5}{6}$, $\frac{13}{17}$, et le minimum est $\frac{5}{9}$. La variable entrante est donc $x_1$.

Cela donne
\[
\begin{matrix}
x_1 & 1 & \frac{1}{3} & \frac{4}{3} & \frac{17}{9} & 0 & -\frac{2}{9} & -\frac{5}{9} & 3 \\
x_5 & 0 & \frac{1}{3} & \frac{1}{3} & -\frac{4}{9} & 1 & \frac{1}{9} & -\frac{2}{9} & 0 \\
& 0 & \frac{8}{3} & \frac{5}{3} & \frac{32}{18} & 0 & \frac{5}{9} & \frac{17}{9} & -9
\end{matrix}
\]
\end{frame}

\begin{frame}
\frametitle{Simplexe dual}

On en déduit la solution primale
\[
\bx^* = ( 3, 0, 0 ,0, 0)
\]
associée à la solution duale
\[
\blambda^* = \left( -\frac{5}{9}, -\frac{17}{9} \right).
\]

\mbox{}

Les valeurs optimales primales et duales sont égale à 9.

\mbox{}

On peut noter la correspondance entre les variables de base et les contraintes duales, comme seules les première et cinquième contrainte du dual sont actives.
\end{frame}

\begin{frame}
\frametitle{Simplexe primal-dual}

Il peut y avoir un grand nombre de rapports à calculer, ce qui peut être coûteux s'il y a un grand nombre de variables et de contraintes, et il n'est pas toujours aisé d'avoir la forme canonique.

\mbox{}

On va essayer de limiter le nombre de variables qui peuvent être candidates pour entrer dans la base, et introduire des variables artificielles pour avoir une forme canonique initiale facile à définir.

\mbox{}

On part du primal: 
\begin{align*}
\min_{x}\ & 3x_1 + 4x_2 + 6x_3 + 7x_4 + x_5 \\
\mbox{s.à. } & 2x_1 - x_2 + x_3 +6x_4 - 5x_5 - x_6 = 6 \\
& x_1 + x_2 + 2x_3 + x_4 + 2x_5 - x_7 = 3 \\
& x_1, x_2, x_3, x_4, x_5, x_6, x_7 \geq 0.
\end{align*}

\end{frame}

\begin{frame}
\frametitle{Simplexe primal-dual}

Problème artificiel: 
\begin{align*}
\min_{x}\ & 3x_1 + 4x_2 + 6x_3 + 7x_4 + x_5 \\
\mbox{s.à. } & 2x_1 - x_2 + x_3 +6x_4 - 5x_5 - x_6 + y_1 = 6 \\
& x_1 + x_2 + 2x_3 + x_4 + 2x_5 - x_7 + y_2 = 3 \\
& x_1, x_2, x_3, x_4, x_5, x_6, x_7, y_1, y_2 \geq 0.
\end{align*}

\mbox{}

Tableau:
\[
\begin{matrix}
2 & -1 & 1 & 6 & -5 & -1 & 0 & 1 & 0 & 6 \\
1 & 1 & 2 & 1 & 2 & 0 & -1 & 0 & 1 & 3 \\
-3 & 0 & -3 & -7 & -3 & 1 & 1 & 0 & 0 & -9.
\end{matrix}
\]

On pourrait continuer la phase 1, mais on veut résoudre directement le problème. 

\end{frame}

\begin{frame}
\frametitle{Simplexe primal-dual}

Regardons du côté du dual du problème initial:
\begin{align*}
\max_{\lambda}\ & 6\lambda_1 + 3\lambda_2 \\
& 2\lambda_1 + \lambda_2 \leq 3 \\
& -\lambda_1 + \lambda_2 \leq 4 \\
& \lambda_1 + 2\lambda_2 \leq 6 \\
& 6\lambda_1 + \lambda_2 \leq 7 \\
& -5\lambda_1 + 2\lambda_2 \leq 1 \\
& -\lambda_1 \leq 0 \\
& -\lambda_2 \leq 0 \\
& \lambda_1, \lambda_2 \in \cR.
\end{align*}

\mbox{}

Une solution admissible du dual est, comme précédemment, $\lambda_1 = \lambda_2 = 0$.

\end{frame}

\begin{frame}
\frametitle{Simplexe primal-dual}

Ajoutons au tableau les valeurs des contraintes du dual, sous la forme
$c_i - \lambda^T a_i$:
\[
\begin{matrix}
2 & -1 & 1 & 6 & -5 & -1 & 0 & 1 & 0 & 6 \\
1 & 1 & 2 & 1 & 2 & 0 & -1 & 0 & 1 & 3 \\
-3 & 0 & -3 & -7 & 3 & 1 & 1 & 0 & 0 & -9 \\
3 & 4 & 6 & 7 & 1 & 0 & 0
\end{matrix}
\]

\mbox{}

Soit $P$ l'ensemble défini comme
\[
P \overset{def}{=} \lbrace i \,|\, \lambda^T a_i = c_i \rbrace,
\]
i.e. l'ensemble des indices des contraintes duales actives. Ici,
\[
P = \lbrace 6, 7 \rbrace.
\]
Notons toutefois que dans le cas présent, malgré le caractère actif de deux contraintes du dual, les variables primales associées sont hors-base. Ces contraintes pourront dès lors devenir inactives plus tard.

\end{frame}

\begin{frame}
\frametitle{Simplexe primal-dual}

On va se tourner vers le dual pour décider de la variable primale à entrer dans la base, tout en maintenant l'admissibilité du dual (et donc implicitement de l'existence des conditions d'optimalité pour le primal).

\mbox{}

Dual restreint:
\begin{align*}
\max\ & 6u_1 + 3u_2 \\
\mbox{s.à. } & -u_1 \leq 0 \\
& -u_2 \leq 0 \\
& u_1 \leq 1 \\
& u_2 \leq 1 \\
& u_1, u_2 \in \mathcal{R}.
\end{align*}

\mbox{}

Ce dual a comme solution immédiate $u_1 = 1$, $u_2 = 2$, mais en fait, celle-ci ne nous intéresse pas vraiment.

\end{frame}

\begin{frame}
\frametitle{Simplexe primal-dual}

Par contre, on peut observer que la ligne des coûts réduits, limitée aux variables primales originales (hors variables artificielles), donne les valeurs
\[
-u^Ta_i, \ \forall i.
\]

\mbox{}

En effet, à l'optimalité pour le primal restreint, comprenant les contraintes $x_i = 0$ pour $i \notin P$, nous avons comme valeurs sur la ligne des coûts réduits
\[
c^T - u^T A = 0,
\]
où $B$ est la base optimale du primale restreint, hors, les composantes de $c$ correspondant aux variables primales originales sont nulles.

\mbox{}

Dès lors, il n'est \emph{jamais} nécessaire d'écrire explicitement le dual restreint, ni de chercher la solution duale explicitement.

\end{frame}

\begin{frame}
\frametitle{Simplexe primal-dual}

On voudrait rendre une des contraintes duales inactives active, autrement dit, introduire un zéro supplémentaire dans la dernière ligne.

\mbox{}

Pour ce faire, on calcule le rapport entre les deux dernières lignes, en se limitant aux éléments négatifs de l'avant-dernière ligne.

\mbox{}

Ici, on obtient les rapports 1, 2, 1. Le minimum est donc 1, conduisant à annuler $c_1 - \lambda^T a_1$ et $c_4 - \lambda^T a_4$.

\mbox{}

Le tableau devient (en ajoutant une fois la troisième ligne à la quatrième)
\[
\begin{matrix}
2 & -1 & 1 & 6 & -5 & -1 & 0 & 1 & 0 & 6 \\
1 & 1 & 2 & 1 & 2 & 0 & -1 & 0 & 1 & 3 \\
-3 & 0 & -3 & -7 & 3 & 1 & 1 & 0 & 0 & -9 \\
0 & 4 & 3 & 0 & 4 & 1 & 1
\end{matrix}
\]

\end{frame}

\begin{frame}
\frametitle{Simplexe primal-dual}

A présent, $P = \lbrace 1, 4 \rbrace$. $x_1$ et $x_4$ peuvent entrer dans la base comme les contraintes duales associées sont à présent actives. Par contre, les variables d'écart $x_6$ et $x_7$ sont à présent exclues de l'ensemble des variables candidates.

\mbox{}

Le primal restreint s'écrit
\begin{align*}
\min\ & y_1 + y_2 \\
\mbox{s.à. } & 2x_1 + 6x_4 + y_1 = 6 \\
& x_1 + x_4 + y_2 = 3 \\
& x_1, x_2, x_3, x_4, x_5, x_6, x_7, y_1, y_2 \geq 0.
\end{align*}

\end{frame}

\begin{frame}
\frametitle{Simplexe primal-dual}

On applique le simplexe primal à ce problème.
\[
\begin{matrix}
2 & -1 & 1 & \circled{6} & -5 & -1 & 0 & 1 & 0 & 6 \\
1 & 1 & 2 & 1 & 2 & 0 & -1 & 0 & 1 & 3 \\
-3 & 0 & -3 & -7 & -3 & 1 & 1 & 0 & 0 & -9 \\
0 & 4 & 3 & 0 & 4 & 1 & 1
\end{matrix}
\]
donne
\[
\begin{matrix}
\frac{1}{3} & -\frac{1}{6} & \frac{1}{6} & 1 & -\frac{5}{6} & -\frac{1}{6} & 0 & \frac{1}{6} & 0 & 1 \\
\frac{2}{3} & \frac{7}{6} & \frac{11}{6} & 0 & \frac{17}{6} & \frac{1}{6} & -1 & -\frac{1}{6}  & 1 & 2 \\
-\frac{2}{3} & -\frac{7}{6} & -\frac{11}{6} & 0 & -\frac{17}{6} & -\frac{1}{6} & 1 & \frac{7}{6} & 0 & -2 \\
0 & 4 & 3 & 0 & 4 & 1 & 1
\end{matrix}
\]
On échange à présent $x_1$ avec $y_2$ (un échange avec $x_4$ créerait un danger de cycle, et de plus, en veut annuler les variables artificielles).

\end{frame}

\begin{frame}
\frametitle{Simplexe primal-dual}

Après application du pivot, nous obtenons
\[
\begin{matrix}
0 & -\frac{3}{4} & -\frac{3}{4} & 1 & -\frac{9}{4} & -\frac{1}{4} & \frac{1}{2} & \frac{1}{4} & -\frac{1}{2} & 0 \\
1 & \frac{7}{4} & \frac{11}{4} & 0 & \frac{17}{4} & \frac{1}{4} & -\frac{3}{2} & -\frac{1}{4} & \frac{3}{2} & 3 \\
0 & 0 & 0 & 0 & 0 & 0 & 0 & 1 & 1 & 0 \\
0 & 4 & 3 & 0 & 4 & 1 & 1
\end{matrix}
\]

\mbox{}

On s'arrête comme $y_1 = y_2 = 0$.

\mbox{}

La solution optimale est (3,0,0,0,0).

\mbox{}

La solution duale (identifiée à partir des contraintes duales au niveau des variables d'écart) correspondante vaut cette fois (1,1). Les seules contraintes actives dans le duale sont les première et quatrième contraintes.

\end{frame}

\end{document}
