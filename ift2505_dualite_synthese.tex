\documentclass[t,usepdftitle=false]{beamer}
\usepackage[utf8]{inputenc}
\usetheme{Warsaw}
\usepackage{xcolor}
% \setbeamercovered{transparent}
%\usecolortheme{crane}
\title[IFT2505]{IFT 2505\\Programmation Linéaire}
\author[Fabian Bastin]{Fabian Bastin\\DIRO\\Université de Montréal\\\mbox{}\\\url{http://www.iro.umontreal.ca/~bastin/ift2505.php}}
\date{Automne 2012}

\usepackage{ulem}
\usepackage{tikz}
\newcommand*\circled[1]{\tikz[baseline=(char.base)]{
    \node[shape=circle,draw,inner sep=2pt] (char) {#1};}}

\def\ba{\boldsymbol{a}}
\def\bb{\boldsymbol{b}}
\def\bc{\boldsymbol{c}}
\def\be{\boldsymbol{e}}
\def\br{\boldsymbol{r}}
\def\bu{\boldsymbol{u}}
\def\bx{\boldsymbol{x}}
\def\by{\boldsymbol{y}}
\def\bz{\boldsymbol{z}}
\def\bA{\boldsymbol{A}}
\def\bB{\boldsymbol{B}}
\def\bD{\boldsymbol{D}}
\def\bH{\boldsymbol{H}}
\def\bI{\boldsymbol{I}}
\def\bL{\boldsymbol{L}}
\def\bM{\boldsymbol{M}}
\def\bU{\boldsymbol{U}}
\def\bzero{\boldsymbol{0}}
\def\bone{\boldsymbol{1}}
\def\blambda{\boldsymbol{\lambda}}

\def\cR{\mathcal{R}}

\usepackage[frenchb]{babel}

\begin{document}
\frame{\titlepage}

\begin{frame}
\frametitle{Simplexe primal-dual}

Motivations:
\begin{itemize}
\item
Exploiter d'avantage la complémentarité entre le primal et le dual.
\item
Comme pour le simplexe duale, on part d'une solution dual-réalisable.
\item
Primal restreint: on va forcer la condition de complémentarité
\[
x_i > 0 \Rightarrow \lambda^T a_i = c_i,
\]
en faisant entrer dans la base primale les $x_i$ correspondant aux contraintes duales actives.
\item
Dual restreint: on optimise le dual. Si celui-ci est réalisable, augmenter (strictement) la valeur de l'objectif dual va conduire à transformer au moins une contrainte duale inactive en contrainte duale active.
\end{itemize}

\end{frame}

\end{document}