\documentclass[usepdftitle=false]{beamer}
\usepackage[utf8]{inputenc}

\usetheme{Singapore}
\usepackage{xcolor}

\setbeamertemplate{footline}[frame number]

% \setbeamercovered{transparent}
%\usecolortheme{crane}
\title[IFT2505]{IFT 2505\\Programmation Linéaire\\Stratégies de points intérieurs}
\author[Fabian Bastin]{Fabian Bastin\\DIRO\\Université de Montréal\\\mbox{}}
\date{}

\usepackage{ulem}
\usepackage{cancel}

\usepackage{tikz}
\newcommand*\circled[1]{\tikz[baseline=(char.base)]{
    \node[shape=circle,draw,inner sep=2pt] (char) {#1};}}

\def\ba{\boldsymbol{a}}
\def\bb{\boldsymbol{b}}
\def\bc{\boldsymbol{c}}
\def\bd{\boldsymbol{d}}
\def\be{\boldsymbol{e}}
\def\br{\boldsymbol{r}}
\def\bs{\boldsymbol{s}}
\def\bu{\boldsymbol{u}}
\def\bx{\boldsymbol{x}}
\def\by{\boldsymbol{y}}
\def\bz{\boldsymbol{z}}
\def\bA{\boldsymbol{A}}
\def\bB{\boldsymbol{B}}
\def\bD{\boldsymbol{D}}
\def\bH{\boldsymbol{H}}
\def\bI{\boldsymbol{I}}
\def\bL{\boldsymbol{L}}
\def\bM{\boldsymbol{M}}
\def\bX{\boldsymbol{X}}
\def\bU{\boldsymbol{U}}
\def\bzero{\boldsymbol{0}}
\def\bone{\boldsymbol{1}}
\def\blambda{\boldsymbol{\lambda}}

\def\cA{\mathcal{A}}
\def\cN{\mathcal{N}}
\def\cR{\mathcal{R}}

\def\RR {\mathcal{R}}

\usepackage[french]{babel}

\begin{document}
\frame{\titlepage}

% ------------------------------------------------------------------------------------------------------------------------------------------------------\begin{frame}

\begin{frame}
\frametitle{Stratégies de solution en points intérieurs}

Trois grandes approches, suivant les différences dans les définitions du chemin central:
\begin{enumerate}
\item
barrière primale, méthode de poursuite de chemin,
\item
méthode primale-duale de poursuite de chemin,
\item
méthode primale-duale de réduction de potentiel.
\end{enumerate}

\mbox{}

Caractéristiques
\begin{tabular}{c|ccc}
& R-P & R-D & Saut nul \\
\hline
simplexe primal & X & & X \\
simplexe dual & & X & X\\
barrière primale & X & & \\
poursuite de chemin primale-duale & X & X & \\
réduction de potentiel primale-duale & X & X & \\
\end{tabular}

R: réalisabilité, P: primal, D: dual

\end{frame}

\begin{frame}
\frametitle{Méthode primale barrière}

Nous partons du problème primal barrière
\begin{align*}
\min_x \ & c^Tx - \mu \sum_{j = 1}^n \log x_j \\
\mbox{s.à. } & Ax = b \\
& x \geq 0.
\end{align*}

\mbox{}

Nous voudrions le résoudre pour $\mu$ petit.

\mbox{}

Par exemple, $\mu = \epsilon/n$ permet d'obtenir un saut de dualité inférieur à $\epsilon$.

\mbox{}

Souci: difficile de résoudre pour $\mu$ proche de 0.

\end{frame}

\begin{frame}
\frametitle{Méthode primale barrière}

Une stratégie générale est de commencer avec $\mu$ modérément large (p.e. $\mu = 100$) et de résoudre le problème approximativement.

\mbox{}

La solution correspondante est approximativement sur le chemin central primal, mais probablement assez loin du point correspondant à $\mu \rightarrow 0$.

\mbox{}

Ce point ne servira que de point de départ pour le problème avec un $\mu$ plus petit.

\mbox{}

Typiquement, on mettra à jour $\mu$ de l'itération $k$ à l'itération $k+1$ comme
\[
\mu_{k+1} = \gamma \mu_k,
\]
pour $0< \gamma < 1$ fixé.

\end{frame}

\begin{frame}
\frametitle{Méthode primale barrière}

Si on commence avec une valeur $\mu_0$, à l'itération $k$,
\[
\mu_{k} = \gamma^k \mu_0,
\]

\mbox{}

Dès, réduire $\mu_k/\mu_0$ sous $\epsilon$ requiert
\[
k = \left\lceil \frac{\log \epsilon}{\log \gamma} \right\rceil.
\]

\mbox{}

Souvent, une variante de la méthode de Newton est utilisée pour résoudre les sous-problèmes ainsi construits:
\begin{align*}
\bx \circ \bs &= \mu \bone \\
\bA\bx &= \bb \\
\bA^T\by + \bs & = \bc
\end{align*}

\end{frame}

\begin{frame}
\frametitle{Système d'équation non linéaires: méthode de Newton}

Considérons le système d'équations non linéaires
$$
F(\bx) = 0
$$
avec $F: \RR^n \rightarrow \RR^m$.

\mbox{}

En partant d'un point $\bx_0$, la méthode de Newton construit une suite d'itérés à partir de la récurrence
$$
\bx_{k+1} = \bx_k - J^{-1}(\bx_k) F(\bx_k)
$$
où $J(\bx_k)$ est le Jacobien de f:
\[
J(\bx) =
\begin{pmatrix}
\nabla^T_x f_1(x) \\
\nabla^T_x f_2(x) \\
\vdots \\
\nabla^T_x f_n(x) \\
\end{pmatrix}
\]

La méthode converge si on est suffisamment proche d'un zéro de la fonction.

\end{frame}

\begin{frame}
\frametitle{Méthode primale barrière}

Etant donné un point $x \in \mathring{\mathcal{F}}_P$, la méthode de Newton consistera à chercher des direction $\bd_x$, $\bd_y$ et $\bd_s$ à partir du système
\begin{align*}
\mu \bX^{-2} \bd_x + \bd_s &= \mu \bX^{-1}\bone - \bc \\
\bA\bd_x &= \bzero \\
-\bA^T\bd_y + \bd_s & = 0
\end{align*}

\mbox{}

On construit le nouveau point comme
\[
\bx^+ = \bx + \bd_x.
\]

\mbox{}

Si $\bx \circ \bs = \mu \bone$ pour un certain $\bs = \bc - \bA^T\by$, alors $\bd \equiv (\bd_x, \bd_y, \bd_s) = 0$.\\
Si une composante de $\bx \circ \bs$ est plus petite que $\mu$, l'approche tendera à augmenter cette composante, et inversément si la composante est plus grande que $\mu$.

\end{frame}

\begin{frame}
\frametitle{Méthode primale barrière}

La méthode marche relativement bien si $\mu$ est modérément grand, ou si l'algorithme est démarré avec un point proche de la la solution.

\mbox{}

Pour trouver $(\bd_x, \bd_y, \bd_s)$, prémultiplions les deux côtés de la première égalité du système de Newton par $\bX^2$:
\[
\mu \bd_x + \bX^2 \bd_s = \mu \bX \bone - \bX^2 \bc.
\]
En prémultipliant par $\bA$ et en utilisant $\bA d_x = \bzero$, nous avons
\[
\bA \bX^2 \bd_s = \mu \bA \bX \bone - \bA \bX^2 \bc.
\]
Comme $\bd_s = \bA^T \bd_y$, nous avons
\[
\bA \bX^2 \bA^T \bd_y = \mu \bA \bX \bone - \bA \bX^2 \bc.
\]
On en tire $\bd_y$, et de là, $\bd_s$ puis $\bd_x$.

\end{frame}

\begin{frame}
\frametitle{Algorithme général}

\begin{description}
\item[Etape 1]
Choisir $\mu_0$ et un point de départ $(x_0, y_0, s_0)$, tel que
\begin{align*}
Ax_0 &= b \\
A^Ty_0 + s_0 &= c^T
\end{align*}
et $y_0, s_0 \geq 0$, $x_0 > 0$. Poser $k = 0$, et choisir $mu_0$, $\epsilon$.
\item[Etape 2]
Projeter $x_k$ sur le chemin central avec la méthode de Newton: calculer le vecteur $(d_x, d_y, d_s)$ et poser
\begin{align*}
x_{k+1} &= x_k + d_x \\
y_{k+1} &= y_k + d_y \\
s_{k+1} &= s_k + d_s
\end{align*}
Si $\mu_k < \epsilon$, arrête.
Sinon, définir $\mu_{k + 1} = \gamma \mu_k$, poser $k := k+1$, et retourner au début de l'étape 2.
\end{description}

\end{frame}

\end{document}
